\documentclass[11pt]{article}
\usepackage[croatian]{babel}
\usepackage[utf8]{inputenc}
\usepackage[T1]{fontenc}
\usepackage{bookman}
\usepackage{microtype}
\usepackage[left=1.5cm,right=1.5cm,bottom=1.5cm,top=1.5cm]{geometry}
\usepackage[onehalfspacing]{setspace}
\usepackage{titlesec}

\titleformat{\section}{\Large\bfseries}{}{0pt}{}
\titlespacing*{\section}{0pt}{5pt}{5pt}

\pagenumbering{gobble}

\title{\bfseries \large Pristanak na sudjelovanje u istraživanju:\\
        \Large Računalno učenje i testiranje}
\date{}
\author{}

\begin{document}

\maketitle

\vspace*{-4em}

\noindent Ovo istraživanje provodi se na Katedri za eksperimentalnu psihologiju, u sklopu
projekta \textit{E-rudito: Napredni online obrazovni sustav za pametnu
    specijalizaciju i poslove budućnosti}, sufinanciranog iz Europskog fonda za
regionalni razvoj, Operativni program Konkurentnost i kohezija 2014.-2020.

\vspace{6pt}

\noindent Istraživače možete kontaktirati na mpavlic2@ffzg.hr.

\section{Svrha istraživanja}

Svrha ovog istraživanja je ispitati načine na koje ljudi uče u \textit{online}
okruženjima te mogućnosti testiranja u takvim okruženjima.

\section{Postupak}

Postupak istraživanja sastoji se od tri vremenski odvojena dijela. Prvi dio
održat će se sada, a sastoji se od čitanja tekstova i odgovaranja na neka
pitanja.

U drugom i trećem dijelu, koji će se održati unutar otprilike sljedeća četiri
tjedna, zamolit ćemo Vas da dođete u Mali praktikum na Odsjeku za psihologiju.
Tamo ćete rješavati nekoliko računalnih zadatka vezanih uz pažnju i radno
pamćenje. Za sudjelovanje u \textbf{sva tri} dijela ovog istraživanja dobit ćete
5 eksperimentalnih sati.  

Iako su nam Vaše sudjelovanje i Vaši iskreni odgovori izrazito važni za
kvalitetu istraživanja i donošenje zaključaka, u svakom trenutku možete odustati
od sudjelovanja u istraživanju, bez
straha od bilo kakvih negativnih posljedica. Ako odlučite sudjelovati,
molimo Vas da na pitanja odgovarate motivirano te da pokušate odgovoriti na svako
pitanje i da sve testove pokušate riješiti što bolje. Svaki zadatak sadrži upute
za rad koje će Vas voditi kroz aktivnosti.
Svi prikupljeni, upotrebljivi podaci bit će analizirani, osim ako eksplicitno
zatražite njihovo uništavanje.

\section{Rizici, stres, neugoda}

Tijekom istraživanja nećete biti izloženi nikakvom specifičnom riziku. Također,
razine stresa i neugode koje biste mogli doživjeti u ovom istraživanju nisu veće
od onih koje izazivaju svakodnevne situacije.

\section{Ostale informacije}

\textbf{Povjerljivost informacija o Vašem identitetu je u istraživanju
zajamčena.} Vaši će odgovori biti kodirani šifrom na temelju koje ih se neće
moći povezati s Vašim imenom. Pristup podacima imat će samo istraživači. Ako
nalazi ovog istraživanja budu objavljeni, objavit će se samo za čitavu grupu
sudionika te se ni na koji način neće moći povezati s Vama. U slučaju da će se
podaci prikupljeni u ovom istraživanju dijeliti s drugim osobama, šifre i drugi
potencijalno identificirajući podaci bit će uklonjeni prije dijeljenja.

\textbf{Naglašavamo da je sudjelovanje u ovom istraživanju dobrovoljno i da
imate pravo bez ikakvih posljedica odustati od sudjelovanja ili se iz njega
povući.} U slučaju da odustanete od sudjelovanja, molimo Vas da nam to javite
prije svog termina.
Nadalje, molimo Vas da \textbf{nikakve} detalje postupka \textbf{ne dijelite s
drugima} do završetka istraživanja. Rezultate istraživanja moći ćete doznati
kad ono završi.

\vspace{2em}

\noindent Ako imate neke pritužbe na provedeni postupak ili ste
zabrinuti zbog nečega što ste tijekom istraživačkog postupka doživjeli, molimo
Vas obratite se na etikapsi@ffzg.hr.

\vspace{2em}

\noindent \textbf{Potpisivanjem ovog dokumenta izjavljujete da ste pročitali
    gore navedeni tekst te dajete svoj informirani pristanak na sudjelovanje u
ovom istraživanju.} 
\vspace*{5em}

\noindent Potpis sudionika: \rule{6cm}{0.15mm}
\end{document}
