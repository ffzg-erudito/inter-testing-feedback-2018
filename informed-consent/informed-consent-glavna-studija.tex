\documentclass[11pt]{article}
\usepackage[croatian]{babel}
\usepackage[utf8]{inputenc}
\usepackage[T1]{fontenc}
\usepackage{bookman}
\usepackage{microtype}
\usepackage[left=1.5cm,right=1.5cm,bottom=.0cm,top=.0cm]{geometry}
\usepackage[onehalfspacing]{setspace}
\usepackage{titlesec}

\titleformat{\section}{\large\bfseries}{}{0pt}{}
\titlespacing*{\section}{0pt}{5pt}{5pt}

\pagenumbering{gobble}

\title{\bfseries \large Pristanak na sudjelovanje u istraživanju:\\
        \Large Čitanje na računalu}
\date{}
\author{}

\begin{document}

\vspace{-4em}

\maketitle

\vspace*{-4em}

\noindent Ovo istraživanje provodi se na Odsjeku za psihologiju Filozofskog 
fakulteta Sveučilišta u Zagrebu, u sklopu projekta \textit{E-rudito: Napredni 
online obrazovni sustav za pametnu specijalizaciju i poslove budućnosti}, 
sufinanciranog iz Europskog fonda za regionalni razvoj, Operativni program 
Konkurentnost i kohezija 2014.-2020.

\vspace{6pt}

\noindent Istraživače možete kontaktirati na mpavlic2@ffzg.hr i dvlasice@ffzg.hr.

\section{Svrha istraživanja}

Svrha ovog istraživanja je ispitati načine na koje ljudi čitaju u \textit{online}
okruženjima.

\section{Postupak}

Postupak istraživanja sastoji se od tri vremenski odvojena dijela.
Vaš zadatak u ovom prvom dijelu bit će pročitati jedan tekst te odgovoriti na nekoliko pitanja.
U drugom i trećem dijelu, koji će se održati unutar otprilike sljedećih mjesec dana, 
zamolit ćemo  Vas da ponovno dođete u ovu prostoriju (C-325). Tada ćete rješavati 
nekoliko računalnih zadatka vezanih uz pažnju i radno pamćenje.


Iako su nam Vaše sudjelovanje i Vaši iskreni odgovori izrazito važni za
kvalitetu istraživanja i donošenje zaključaka, u svakom trenutku možete odustati
od sudjelovanja u istraživanju, bez straha od bilo kakvih negativnih posljedica. 
\textbf{Ako odlučite sudjelovati, molimo Vas da istraživanju pristupite motivirano.}
Svaki zadatak sadrži upute za rad koje će Vas voditi kroz aktivnosti.
Svi prikupljeni, upotrebljivi podaci bit će analizirani, osim ako eksplicitno
zatražite njihovo uništavanje.

\section{Rizici, stres, neugoda}

Tijekom istraživanja nećete biti izloženi nikakvom specifičnom riziku.
Razine stresa i neugode koje biste mogli doživjeti u ovom istraživanju nisu veće
od onih koje izazivaju svakodnevne situacije.

\section{Ostale informacije}

\textbf{Povjerljivost informacija o Vašem identitetu je u istraživanju
zajamčena.} Vaši će odgovori biti kodirani šifrom na temelju koje ih se neće
moći povezati s Vašim imenom. Pristup podacima imat će samo istraživači. Ako
nalazi ovog istraživanja budu objavljeni, objavit će se samo za čitavu grupu
sudionika te se ni na koji način neće moći povezati s Vama. U slučaju da će se
podaci prikupljeni u ovom istraživanju dijeliti s drugim osobama, šifre i drugi
potencijalno identificirajući podaci bit će uklonjeni prije dijeljenja.

\textbf{Naglašavamo da je sudjelovanje u ovom istraživanju dobrovoljno i da
imate pravo bez ikakvih posljedica odustati od sudjelovanja ili se iz njega
povući.} U slučaju da odustanete od sudjelovanja, molimo Vas da nam to javite
prije svog termina.
Nadalje, molimo Vas da \textbf{nikakve} detalje postupka \textbf{ne dijelite s
drugima} do završetka istraživanja. Rezultate istraživanja moći ćete doznati
kad ono završi.

\vspace{1em}

\noindent Ako imate neke pritužbe na provedeni postupak ili ste
zabrinuti zbog nečega što ste tijekom istraživačkog postupka doživjeli, molimo
Vas obratite se na etikapsi@ffzg.hr.

\vspace{1em}

\noindent \textbf{Potpisivanjem ovog dokumenta izjavljujete da ste pročitali
    gore navedeni tekst te dajete svoj informirani pristanak na sudjelovanje u
ovom istraživanju.} 
\vspace*{1em}

\noindent Potpis sudionika: \rule{6cm}{0.15mm}
\end{document}
