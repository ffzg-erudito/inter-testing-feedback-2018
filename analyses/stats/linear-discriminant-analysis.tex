\documentclass[11pt,]{article}
\usepackage{lmodern}
\usepackage{amssymb,amsmath}
\usepackage{ifxetex,ifluatex}
\usepackage{fixltx2e} % provides \textsubscript
\ifnum 0\ifxetex 1\fi\ifluatex 1\fi=0 % if pdftex
  \usepackage[T1]{fontenc}
  \usepackage[utf8]{inputenc}
\else % if luatex or xelatex
  \ifxetex
    \usepackage{mathspec}
  \else
    \usepackage{fontspec}
  \fi
  \defaultfontfeatures{Ligatures=TeX,Scale=MatchLowercase}
\fi
% use upquote if available, for straight quotes in verbatim environments
\IfFileExists{upquote.sty}{\usepackage{upquote}}{}
% use microtype if available
\IfFileExists{microtype.sty}{%
\usepackage{microtype}
\UseMicrotypeSet[protrusion]{basicmath} % disable protrusion for tt fonts
}{}
\usepackage[margin=1.3cm]{geometry}
\usepackage{hyperref}
\hypersetup{unicode=true,
            pdftitle={Supplementary material - linear discriminant analysis},
            pdfauthor={DV},
            pdfborder={0 0 0},
            breaklinks=true}
\urlstyle{same}  % don't use monospace font for urls
\usepackage{graphicx,grffile}
\makeatletter
\def\maxwidth{\ifdim\Gin@nat@width>\linewidth\linewidth\else\Gin@nat@width\fi}
\def\maxheight{\ifdim\Gin@nat@height>\textheight\textheight\else\Gin@nat@height\fi}
\makeatother
% Scale images if necessary, so that they will not overflow the page
% margins by default, and it is still possible to overwrite the defaults
% using explicit options in \includegraphics[width, height, ...]{}
\setkeys{Gin}{width=\maxwidth,height=\maxheight,keepaspectratio}
\IfFileExists{parskip.sty}{%
\usepackage{parskip}
}{% else
\setlength{\parindent}{0pt}
\setlength{\parskip}{6pt plus 2pt minus 1pt}
}
\setlength{\emergencystretch}{3em}  % prevent overfull lines
\providecommand{\tightlist}{%
  \setlength{\itemsep}{0pt}\setlength{\parskip}{0pt}}
\setcounter{secnumdepth}{0}
% Redefines (sub)paragraphs to behave more like sections
\ifx\paragraph\undefined\else
\let\oldparagraph\paragraph
\renewcommand{\paragraph}[1]{\oldparagraph{#1}\mbox{}}
\fi
\ifx\subparagraph\undefined\else
\let\oldsubparagraph\subparagraph
\renewcommand{\subparagraph}[1]{\oldsubparagraph{#1}\mbox{}}
\fi

%%% Use protect on footnotes to avoid problems with footnotes in titles
\let\rmarkdownfootnote\footnote%
\def\footnote{\protect\rmarkdownfootnote}

%%% Change title format to be more compact
\usepackage{titling}

% Create subtitle command for use in maketitle
\providecommand{\subtitle}[1]{
  \posttitle{
    \begin{center}\large#1\end{center}
    }
}

\setlength{\droptitle}{-2em}

  \title{Supplementary material - linear discriminant analysis}
    \pretitle{\vspace{\droptitle}\centering\huge}
  \posttitle{\par}
    \author{DV}
    \preauthor{\centering\large\emph}
  \postauthor{\par}
    \date{}
    \predate{}\postdate{}
  
\usepackage{booktabs}
\usepackage{longtable}
\usepackage{array}
\usepackage{multirow}
\usepackage{wrapfig}
\usepackage{float}
\usepackage{colortbl}
\usepackage{pdflscape}
\usepackage{tabu}
\usepackage{threeparttable}
\usepackage{threeparttablex}
\usepackage[normalem]{ulem}
\usepackage{makecell}
\usepackage{xcolor}

\usepackage[natbibapa, sectionbib, tocbib]{apacite}
\usepackage[utf8]{inputenc}
\usepackage{caption}
\usepackage{bookman}
\usepackage{multirow}
\usepackage{array}
\usepackage[htt]{hyphenat}
\usepackage{booktabs}
\usepackage[euler]{textgreek}
\usepackage{float}
\usepackage[onehalfspacing]{setspace}
\captionsetup[table]{width=\textwidth}
\hypersetup{colorlinks = true, linkcolor = blue, urlcolor = black, citecolor = black}

\begin{document}
\maketitle

{
\setcounter{tocdepth}{4}
\tableofcontents
}
In the results section of the paper, Roy-Bargmann stepdown analyses are
used to dive deeper into the nature of the differences identified by the
MANOVAs. As some authors
\citep[eg.][]{tabachnickUsingMultivariateStatistics2012,salkindEncyclopediaMeasurementStatistics2007,
fieldDiscoveringStatisticsUsing2012} recommend conducting a linear
discriminant analysis as a follow-up to MANOVA, we are reporting these
results here.

\hypertarget{hard-exclusion-criteria}{%
\section{Hard exclusion criteria}\label{hard-exclusion-criteria}}

As in the paper itself, the following analyses are going to be conducted
on a subset of the collected data which contains 203 cases. We will
conduct the analyses specified in the \texttt{analysis-plan.md} file and
follow them up with linear discriminant analyses.

\hypertarget{interpolated-activity-effect}{%
\section{Interpolated activity
effect}\label{interpolated-activity-effect}}

Again, we will first conduct a MANOVA with the total number of correct
answers and total number of intrusive distractors chosen as dependent
variables and the type of interpolated activity as the independent
variable.

\hypertarget{note}{%
\subsection{Note}\label{note}}

A decision was made not to check the univariate and multivariate
outliers at this point. Regarding the univariate outliers - the boxplots
point to only one case which could be an outlier. The scatterplots show
no point that's obviously different from the rest. As for the
multivariate outliers, \citet{tabachnickUsingMultivariateStatistics2012}
warn that the Mahalanobis distance can produce false negatives or false
positives. Furthermore, deleting a set of outliers and rerunning the
analysis can reveal yet another set of outliers --- without a clear-cut
and absolute criterion, exclusions are somewhat arbitrary. Finally,
cases were excluded based on criteria that are more or less
substantivelly meaningful in the context of the conducted study. Given
the above, no statistical criteria is used for exclusion at this point.

\hypertarget{manova}{%
\subsection{MANOVA}\label{manova}}

Here's the otuput of \texttt{R}'s \texttt{manova} function:

\begin{verbatim}
## 
## Type II MANOVA Tests:
## 
## Sum of squares and products for error:
##                totalCorrect totalIntrusors
## totalCorrect            993           -417
## totalIntrusors         -417            435
## 
## ------------------------------------------
##  
## Term: as.factor(activityFactor) 
## 
## Sum of squares and products for the hypothesis:
##                totalCorrect totalIntrusors
## totalCorrect          125.9          -70.5
## totalIntrusors        -70.5           41.0
## 
## Multivariate Tests: as.factor(activityFactor)
##                  Df test stat approx F num Df den Df  Pr(>F)    
## Pillai            2     0.126     3.99      4    238 0.00376 ** 
## Wilks             2     0.875     4.07      4    236 0.00327 ** 
## Hotelling-Lawley  2     0.142     4.16      4    234 0.00285 ** 
## Roy               2     0.137     8.13      2    119 0.00049 ***
## ---
## Signif. codes:  0 '***' 0.001 '**' 0.01 '*' 0.05 '.' 0.1 ' ' 1
\end{verbatim}

As can be seen from the resulting output, Pillai's V indicates that the
three groups differ significantly along the linear combination of the
two DVs. The other three reported statistics point to the same
conclusion. Therefore, we'll proceed with conducting a linear
discriminant analyis.

We can look at \(1 - \Lambda\) as an extension to the univariate
\(\eta^2\) \citep{hubertyAppliedMANOVADiscriminant2006}. In our case,
the multivariate \(\eta^2\) is 0.125, which represents the proportion of
total variance associated with the activity type IV. Further, we can
calculate the effect size index \(\xi^2\), which is based on Pillai's
test statistic, and represents the mean squared canonical correlation
\citep{hubertyAppliedMANOVADiscriminant2006}: \[\xi^2 = \frac{U}{r},\]
where \(r\) is the number of variates (2, in our case). Therefore,
\(\xi^2 = 0.063\). Finally, we will calculate Tatsuoka's
\citep[1970; according to][]{hubertyAppliedMANOVADiscriminant2006}
extension of the \(\omega^2\) to the multivariate case. In this case,
\(\omega^2_{mult} = 0.109\). The adjusted value of the \(\xi^2\)
statistic is \(\xi^2_{adj} = 0.047\)

Now, let's take a closer look at the nature of our effect, using linear
discriminant analysis.

\hypertarget{linear-discriminant-analysis}{%
\subsection{Linear discriminant
analysis}\label{linear-discriminant-analysis}}

We are using the \texttt{candisc} function from the eponymous package
\citep{friendlyCandiscVisualizingGeneralized2017} to conduct the LDA.

\begin{verbatim}
## 
## Canonical Discriminant Analysis for as.factor(activityFactor):
## 
##    CanRsq Eigenvalue Difference Percent Cumulative
## 1 0.12018     0.1366      0.131   96.13       96.1
## 2 0.00547     0.0055      0.131    3.87      100.0
## 
## Test of H0: The canonical correlations in the 
## current row and all that follow are zero
## 
##   LR test stat approx F numDF denDF Pr(> F)   
## 1        0.875     4.07     4   236  0.0033 **
## 2        0.995     0.65     1   119  0.4202   
## ---
## Signif. codes:  0 '***' 0.001 '**' 0.01 '*' 0.05 '.' 0.1 ' ' 1
\end{verbatim}

\begin{verbatim}
## Class means
\end{verbatim}

\begin{verbatim}
##             Can1    Can2
## content    0.501 -0.0104
## general   -0.309 -0.0845
## rereading -0.217  0.0954
\end{verbatim}

\begin{verbatim}
## Raw coefficients
\end{verbatim}

\begin{verbatim}
##                  Can1  Can2
## totalCorrect    0.255 0.368
## totalIntrusors -0.186 0.651
\end{verbatim}

\begin{verbatim}
## Standardized coefficients
\end{verbatim}

\begin{verbatim}
##                  Can1 Can2
## totalCorrect    0.736 1.06
## totalIntrusors -0.355 1.24
\end{verbatim}

From the above output we can see that the first variate explains most of
the variance. Furthermore, Wilks' lambda values inform us that the
groups are separated only on the first variate, so that's the only one
we'll interpret. Also, we can see that the variation in the grouping
variable is almost exclusively explained by the first variate.

Looking at the structure scores, we can see that both the total number
of correct answers and the total number of intrusive distractors chosen
share a lot of variance with the first variate. The first variate is
almost completely defined by the total number of correct answers, but
the contribution of the number of chosen intrusors in also considerable.
This could be due to the relatively high correlation between those two
variables.

\begin{figure*}
\includegraphics{linear-discriminant-analysis_files/figure-latex/ldaPlot-1} \caption{\label{ldaPlot1}Plot showing the cases' location on the two variates. Group means on the variates are marked by crosses. The vertical line marks the 0 on the first variate.}\label{fig:ldaPlot}
\end{figure*}

To assess the ability of the LDA model to discriminate group membership
based on the number of correct answers to the questions and the number
of chosen instrusive distractors, we'll re-train the model and evaluate
it's error rate using the leave-one-out cross-validation technique.

\begin{verbatim}
## Confusion Matrix and Statistics
## 
##            Reference
## Prediction  content general rereading
##   content        27      16        17
##   general        11      19        13
##   rereading       4       5        10
## 
## Overall Statistics
##                                         
##                Accuracy : 0.459         
##                  95% CI : (0.368, 0.552)
##     No Information Rate : 0.344         
##     P-Value [Acc > NIR] : 0.00572       
##                                         
##                   Kappa : 0.185         
##  Mcnemar's Test P-Value : 0.00577       
## 
## Statistics by Class:
## 
##                      Class: content Class: general Class: rereading
## Sensitivity                   0.643          0.475            0.250
## Specificity                   0.588          0.707            0.890
## Pos Pred Value                0.450          0.442            0.526
## Neg Pred Value                0.758          0.734            0.709
## Precision                     0.450          0.442            0.526
## Recall                        0.643          0.475            0.250
## F1                            0.529          0.458            0.339
## Prevalence                    0.344          0.328            0.328
## Detection Rate                0.221          0.156            0.082
## Detection Prevalence          0.492          0.352            0.156
## Balanced Accuracy             0.615          0.591            0.570
\end{verbatim}

As can be seen from the table, the total LOOCV accuracy is 0.459, which
is significantly above the no information rate (which is taken to be the
largest class percentage in the data). According to the Landis \& Koch
\citep[1977; as reported in][]{salkindEncyclopediaMeasurementStatistics2007}
guidlines, this represents only a slight agreement between the predicted
and actual classes. Next, we'll drill into the individual predictors to
see which are useful for discriminating between different groups.

\hypertarget{evaluating-individual-predictors}{%
\subsection{Evaluating individual
predictors}\label{evaluating-individual-predictors}}

\citet{tabachnickUsingMultivariateStatistics2012} describe the process
of sequential discriminant analysis, where predictors are entered
one-by-one, and the improvement in classification accuracy is monitored.
Therefore, we'll fit an LDA model containing only the number of correct
answers as a predictor. Then, we will compare this model's LOOCV
accuracy to that of the full model (reported at the end of the previous
section).

\begin{verbatim}
## Confusion Matrix and Statistics
## 
##            Reference
## Prediction  content general rereading
##   content        29      15        16
##   general        11      22        18
##   rereading       2       3         6
## 
## Overall Statistics
##                                        
##                Accuracy : 0.467        
##                  95% CI : (0.376, 0.56)
##     No Information Rate : 0.344        
##     P-Value [Acc > NIR] : 0.00334      
##                                        
##                   Kappa : 0.198        
##  Mcnemar's Test P-Value : 5.87e-05     
## 
## Statistics by Class:
## 
##                      Class: content Class: general Class: rereading
## Sensitivity                   0.690          0.550           0.1500
## Specificity                   0.613          0.646           0.9390
## Pos Pred Value                0.483          0.431           0.5455
## Neg Pred Value                0.790          0.746           0.6937
## Precision                     0.483          0.431           0.5455
## Recall                        0.690          0.550           0.1500
## F1                            0.569          0.484           0.2353
## Prevalence                    0.344          0.328           0.3279
## Detection Rate                0.238          0.180           0.0492
## Detection Prevalence          0.492          0.418           0.0902
## Balanced Accuracy             0.651          0.598           0.5445
\end{verbatim}

As can be seen from the second confusion matrix, the accuracy of this
model is actually somewhat higher than in the full model, as is Cohen's
\textkappa. Importantly, we notice that adding the total number of
intrusors to the model doesn't significantly increase the accuracy of
the model (the 95\% confidence intervals for the accuracies of the two
models completely overlap).

\hypertarget{multivariate-contrasts}{%
\subsection{Multivariate contrasts}\label{multivariate-contrasts}}

We've planned to contrast the two test groups with the rereading group,
and the two test groups with each other. That's what we'll do here.

\begin{verbatim}
##           test vs rereading content vs general
## content                   1                  1
## general                   1                 -1
## rereading                -2                  0
\end{verbatim}

\begin{verbatim}
##                                  totalCorrect totalIntrusors
## (Intercept)                            11.379          4.194
## activityFactortest vs rereading         0.252         -0.216
## activityFactorcontent vs general        1.155         -0.597
\end{verbatim}

Now that we've set up the model, let's run the contrasts. The first
contrast is between the two test groups (content and general knowledge)
and the rereading group.

\begin{verbatim}
## 
## Sum of squares and products for the hypothesis:
##                totalCorrect totalIntrusors
## totalCorrect           15.3          -13.1
## totalIntrusors        -13.1           11.3
## 
## Sum of squares and products for error:
##                totalCorrect totalIntrusors
## totalCorrect            993           -417
## totalIntrusors         -417            435
## 
## Multivariate Tests: 
##                  Df test stat approx F num Df den Df Pr(>F)
## Pillai            1     0.026     1.57      2    118   0.21
## Wilks             1     0.974     1.57      2    118   0.21
## Hotelling-Lawley  1     0.027     1.57      2    118   0.21
## Roy               1     0.027     1.57      2    118   0.21
\end{verbatim}

As can be seen from the test statistics, no significant difference is
found between the two test groups and the rereading group. Next, we'll
look at the contrast between the content test group and the general
knowledge test group.

\begin{verbatim}
## 
## Sum of squares and products for the hypothesis:
##                totalCorrect totalIntrusors
## totalCorrect          109.4          -56.5
## totalIntrusors        -56.5           29.2
## 
## Sum of squares and products for error:
##                totalCorrect totalIntrusors
## totalCorrect            993           -417
## totalIntrusors         -417            435
## 
## Multivariate Tests: 
##                  Df test stat approx F num Df den Df Pr(>F)   
## Pillai            1     0.102     6.73      2    118 0.0017 **
## Wilks             1     0.898     6.73      2    118 0.0017 **
## Hotelling-Lawley  1     0.114     6.73      2    118 0.0017 **
## Roy               1     0.114     6.73      2    118 0.0017 **
## ---
## Signif. codes:  0 '***' 0.001 '**' 0.01 '*' 0.05 '.' 0.1 ' ' 1
\end{verbatim}

This contrast is statistically significant, indicating that the two
groups differ on the linear combination of the number of correct answers
and number of intrusive distractors chosen. We'll calculate the same
effect size indices as for the omnibus model.

The multivariate \(\eta^2\) is 0.102. The effect size index \(\xi^2\) is
0.051. Finally, we will calculate Tatsuoka's
\citep[1970; according to][]{hubertyAppliedMANOVADiscriminant2006}
extension of the \(\omega^2\) to the multivariate case. In this case,
\(\omega^2_{mult} = 0.087\). The adjusted value of the \(\xi^2\)
statistic is \(\xi^2_{adj} = 0.035\)

\hypertarget{contrast-lda}{%
\subsubsection{Contrast LDA}\label{contrast-lda}}

Again, to further investigate the nature of the difference between the
content and general knowledge test group, we'll conduct a linear
discriminant analysis to try and find the variate that best
discriminates these two groups.

\begin{verbatim}
## 
## Type II MANOVA Tests: Pillai test statistic
##                Df test stat approx F num Df den Df Pr(>F)   
## activityFactor  1     0.148     6.85      2     79 0.0018 **
## ---
## Signif. codes:  0 '***' 0.001 '**' 0.01 '*' 0.05 '.' 0.1 ' ' 1
\end{verbatim}

\begin{verbatim}
## 
## Canonical Discriminant Analysis for activityFactor:
## 
##   CanRsq Eigenvalue Difference Percent Cumulative
## 1 0.1479     0.1735                100        100
## 
## Class means:
## 
## [1]  0.4015 -0.4216
## 
##  std coefficients:
##   totalCorrect totalIntrusors 
##         0.7210        -0.3722
\end{verbatim}

\begin{verbatim}
##                  Can1
## totalCorrect    0.964
## totalIntrusors -0.851
\end{verbatim}

Again, we see that both predictors are highly correlated with the
discriminant function, albeit with different signs. Let's look at the
LOOCV prediction accuracy.

\begin{verbatim}
## Confusion Matrix and Statistics
## 
##           Reference
## Prediction content general
##    content      28      16
##    general      14      24
##                                        
##                Accuracy : 0.634        
##                  95% CI : (0.52, 0.738)
##     No Information Rate : 0.512        
##     P-Value [Acc > NIR] : 0.0175       
##                                        
##                   Kappa : 0.267        
##  Mcnemar's Test P-Value : 0.8551       
##                                        
##             Sensitivity : 0.667        
##             Specificity : 0.600        
##          Pos Pred Value : 0.636        
##          Neg Pred Value : 0.632        
##              Prevalence : 0.512        
##          Detection Rate : 0.341        
##    Detection Prevalence : 0.537        
##       Balanced Accuracy : 0.633        
##                                        
##        'Positive' Class : content      
## 
\end{verbatim}

With both predictors, the prediction accuracy is significantlly above
the no information rate.

\begin{verbatim}
## Confusion Matrix and Statistics
## 
##           Reference
## Prediction content general
##    content      29      15
##    general      13      25
##                                        
##                Accuracy : 0.659        
##                  95% CI : (0.546, 0.76)
##     No Information Rate : 0.512        
##     P-Value [Acc > NIR] : 0.00523      
##                                        
##                   Kappa : 0.316        
##  Mcnemar's Test P-Value : 0.85011      
##                                        
##             Sensitivity : 0.690        
##             Specificity : 0.625        
##          Pos Pred Value : 0.659        
##          Neg Pred Value : 0.658        
##              Prevalence : 0.512        
##          Detection Rate : 0.354        
##    Detection Prevalence : 0.537        
##       Balanced Accuracy : 0.658        
##                                        
##        'Positive' Class : content      
## 
\end{verbatim}

Furthermore, the prediction accuracy doesn't drop significantly when we
omit the total number of intrusors.

\setstretch{1}
\bibliographystyle{apacite}
\bibliography{../../paper/reference.bib}


\end{document}
