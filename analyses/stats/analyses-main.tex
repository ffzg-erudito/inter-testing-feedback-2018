\documentclass[11pt,]{article}
\usepackage{lmodern}
\usepackage{amssymb,amsmath}
\usepackage{ifxetex,ifluatex}
\usepackage{fixltx2e} % provides \textsubscript
\ifnum 0\ifxetex 1\fi\ifluatex 1\fi=0 % if pdftex
  \usepackage[T1]{fontenc}
  \usepackage[utf8]{inputenc}
\else % if luatex or xelatex
  \ifxetex
    \usepackage{mathspec}
  \else
    \usepackage{fontspec}
  \fi
  \defaultfontfeatures{Ligatures=TeX,Scale=MatchLowercase}
\fi
% use upquote if available, for straight quotes in verbatim environments
\IfFileExists{upquote.sty}{\usepackage{upquote}}{}
% use microtype if available
\IfFileExists{microtype.sty}{%
\usepackage{microtype}
\UseMicrotypeSet[protrusion]{basicmath} % disable protrusion for tt fonts
}{}
\usepackage[margin=1.8cm]{geometry}
\usepackage{hyperref}
\hypersetup{unicode=true,
            pdfborder={0 0 0},
            breaklinks=true}
\urlstyle{same}  % don't use monospace font for urls
\usepackage{graphicx,grffile}
\makeatletter
\def\maxwidth{\ifdim\Gin@nat@width>\linewidth\linewidth\else\Gin@nat@width\fi}
\def\maxheight{\ifdim\Gin@nat@height>\textheight\textheight\else\Gin@nat@height\fi}
\makeatother
% Scale images if necessary, so that they will not overflow the page
% margins by default, and it is still possible to overwrite the defaults
% using explicit options in \includegraphics[width, height, ...]{}
\setkeys{Gin}{width=\maxwidth,height=\maxheight,keepaspectratio}
\IfFileExists{parskip.sty}{%
\usepackage{parskip}
}{% else
\setlength{\parindent}{0pt}
\setlength{\parskip}{6pt plus 2pt minus 1pt}
}
\setlength{\emergencystretch}{3em}  % prevent overfull lines
\providecommand{\tightlist}{%
  \setlength{\itemsep}{0pt}\setlength{\parskip}{0pt}}
\setcounter{secnumdepth}{0}
% Redefines (sub)paragraphs to behave more like sections
\ifx\paragraph\undefined\else
\let\oldparagraph\paragraph
\renewcommand{\paragraph}[1]{\oldparagraph{#1}\mbox{}}
\fi
\ifx\subparagraph\undefined\else
\let\oldsubparagraph\subparagraph
\renewcommand{\subparagraph}[1]{\oldsubparagraph{#1}\mbox{}}
\fi

%%% Use protect on footnotes to avoid problems with footnotes in titles
\let\rmarkdownfootnote\footnote%
\def\footnote{\protect\rmarkdownfootnote}

%%% Change title format to be more compact
\usepackage{titling}

% Create subtitle command for use in maketitle
\providecommand{\subtitle}[1]{
  \posttitle{
    \begin{center}\large#1\end{center}
    }
}

\setlength{\droptitle}{-2em}

  \title{}
    \pretitle{\vspace{\droptitle}}
  \posttitle{}
    \author{}
    \preauthor{}\postauthor{}
    \date{}
    \predate{}\postdate{}
  
\usepackage{booktabs}
\usepackage{longtable}
\usepackage{array}
\usepackage{multirow}
\usepackage{wrapfig}
\usepackage{float}
\usepackage{colortbl}
\usepackage{pdflscape}
\usepackage{tabu}
\usepackage{threeparttable}
\usepackage{threeparttablex}
\usepackage[normalem]{ulem}
\usepackage{makecell}
\usepackage{xcolor}

\begin{document}

\twocolumn

\hypertarget{results}{%
\subsection{Results}\label{results}}

\hypertarget{exclusion-criteria}{%
\subsubsection{Exclusion criteria}\label{exclusion-criteria}}

Prior to analysing the data, we have excluded participants based on a
priori set criteria. Participants who have spent less than or equal to
90 seconds on the practice text were excluded (1 exclusion). Further, we
wanted to exclude participants who have had no correct answers on the
final test (0 exclusions). Finally, we have excluded participants who
have stated that they have reading deficits (3 exclusions). This left us
with a total sample of 203 participants. There is another set of
exclusion criteria based on the number of times the participants have
read each of the three texts. These are used in robustness check
analyses (see suplementary materials).

\hypertarget{interpolated-activity-effect}{%
\subsubsection{Interpolated activity
effect}\label{interpolated-activity-effect}}

Our first two hypotheses are concerned with the effects of different
interpolated activities on the total number of correct answers and total
number of intrusive distractors chosen. To test these hypotheses, we
have focused only on the groups which have not received feedback, since
there was no feedback option for the rereading group. We conducted a
one-way MANOVA with interpolated activity as the independent variable
and the total number of correct and intrusive options chosen as
dependent variables. The correlation between our DVs calculated on the
whole sample is -0.71 (95\% CI: {[}-0.77, -0.63{]}, \(p\) =
\(4.79255\times 10^{-32}\)).

Pillai's V for the analysis is 0.12565, \(p\) = 0.00376 (Wilks'
\(\Lambda\) = 0.875, \(p\) = 0.00327; Hotelling-Lawley's trace = 0.1421,
\(p\) = 0.00285; Roy's largest root = 0.1366, \(p\) = 0.00049)

\onecolumn

\hypertarget{references}{%
\subsection{References}\label{references}}

\bibliographystyle{apacite}
\bibliography{reference.bib}


\end{document}
