\documentclass[11pt,]{article}
\usepackage{lmodern}
\usepackage{amssymb,amsmath}
\usepackage{ifxetex,ifluatex}
\usepackage{fixltx2e} % provides \textsubscript
\ifnum 0\ifxetex 1\fi\ifluatex 1\fi=0 % if pdftex
  \usepackage[T1]{fontenc}
  \usepackage[utf8]{inputenc}
\else % if luatex or xelatex
  \ifxetex
    \usepackage{mathspec}
  \else
    \usepackage{fontspec}
  \fi
  \defaultfontfeatures{Ligatures=TeX,Scale=MatchLowercase}
\fi
% use upquote if available, for straight quotes in verbatim environments
\IfFileExists{upquote.sty}{\usepackage{upquote}}{}
% use microtype if available
\IfFileExists{microtype.sty}{%
\usepackage{microtype}
\UseMicrotypeSet[protrusion]{basicmath} % disable protrusion for tt fonts
}{}
\usepackage[margin=1.8cm]{geometry}
\usepackage{hyperref}
\hypersetup{unicode=true,
            pdfborder={0 0 0},
            breaklinks=true}
\urlstyle{same}  % don't use monospace font for urls
\usepackage{graphicx,grffile}
\makeatletter
\def\maxwidth{\ifdim\Gin@nat@width>\linewidth\linewidth\else\Gin@nat@width\fi}
\def\maxheight{\ifdim\Gin@nat@height>\textheight\textheight\else\Gin@nat@height\fi}
\makeatother
% Scale images if necessary, so that they will not overflow the page
% margins by default, and it is still possible to overwrite the defaults
% using explicit options in \includegraphics[width, height, ...]{}
\setkeys{Gin}{width=\maxwidth,height=\maxheight,keepaspectratio}
\IfFileExists{parskip.sty}{%
\usepackage{parskip}
}{% else
\setlength{\parindent}{0pt}
\setlength{\parskip}{6pt plus 2pt minus 1pt}
}
\setlength{\emergencystretch}{3em}  % prevent overfull lines
\providecommand{\tightlist}{%
  \setlength{\itemsep}{0pt}\setlength{\parskip}{0pt}}
\setcounter{secnumdepth}{0}
% Redefines (sub)paragraphs to behave more like sections
\ifx\paragraph\undefined\else
\let\oldparagraph\paragraph
\renewcommand{\paragraph}[1]{\oldparagraph{#1}\mbox{}}
\fi
\ifx\subparagraph\undefined\else
\let\oldsubparagraph\subparagraph
\renewcommand{\subparagraph}[1]{\oldsubparagraph{#1}\mbox{}}
\fi

%%% Use protect on footnotes to avoid problems with footnotes in titles
\let\rmarkdownfootnote\footnote%
\def\footnote{\protect\rmarkdownfootnote}

%%% Change title format to be more compact
\usepackage{titling}

% Create subtitle command for use in maketitle
\providecommand{\subtitle}[1]{
  \posttitle{
    \begin{center}\large#1\end{center}
    }
}

\setlength{\droptitle}{-2em}

  \title{}
    \pretitle{\vspace{\droptitle}}
  \posttitle{}
    \author{}
    \preauthor{}\postauthor{}
    \date{}
    \predate{}\postdate{}
  
\usepackage{booktabs}
\usepackage{longtable}
\usepackage{array}
\usepackage{multirow}
\usepackage{wrapfig}
\usepackage{float}
\usepackage{colortbl}
\usepackage{pdflscape}
\usepackage{tabu}
\usepackage{threeparttable}
\usepackage{threeparttablex}
\usepackage[normalem]{ulem}
\usepackage{makecell}
\usepackage{xcolor}

\usepackage[natbibapa, sectionbib, tocbib]{apacite}
\usepackage{microtype}
\usepackage[utf8]{inputenc}
\usepackage{caption}
\usepackage{lmodern}
\usepackage{multirow}
\usepackage{array}
\usepackage{float}
\usepackage[htt]{hyphenat}
\usepackage{booktabs}
\usepackage[euler]{textgreek}
\usepackage{float}
\usepackage[onehalfspacing]{setspace}
\captionsetup[table]{width=\textwidth}
\hypersetup{colorlinks = true, linkcolor = blue, urlcolor = red}

\begin{document}

\twocolumn

\hypertarget{results}{%
\subsection{Results}\label{results}}

\hypertarget{exclusion-criteria}{%
\subsubsection{Exclusion criteria}\label{exclusion-criteria}}

\begin{table*}[t]

\caption{\label{tab:descTable}\label{descTable} Descriptive statistics for the
                     number of correct answers and chosen intrusors broken down
                     by experimental condition.}
\centering
\resizebox{\linewidth}{!}{
\begin{tabular}{llrrrrrrrr}
\toprule
Measure & Condition & $n$ & $M$ & $SE$ & $SD$ & min & max & skew & kurtosis\\
\midrule
 & Content, feedback & 41 & 13.220 & 0.508 & 3.252 & 2 & 19 & -0.800 & 1.503\\

 & Content, no feedback & 42 & 12.786 & 0.465 & 3.017 & 7 & 19 & 0.039 & -0.775\\

 & General, feedback & 40 & 10.975 & 0.533 & 3.370 & 1 & 17 & -0.481 & 0.462\\

 & General, no feedback & 40 & 10.475 & 0.449 & 2.837 & 5 & 16 & -0.053 & -0.986\\

\multirow{-5}{*}{\raggedright\arraybackslash Total correct} & Rereading & 40 & 10.875 & 0.443 & 2.803 & 4 & 17 & -0.141 & -0.253\\
\cmidrule{1-10}
 & Content, feedback & 41 & 3.146 & 0.258 & 1.652 & 0 & 7 & 0.292 & -0.351\\

 & Content, no feedback & 42 & 3.381 & 0.257 & 1.667 & 0 & 7 & 0.203 & -0.385\\

 & General, feedback & 40 & 4.175 & 0.318 & 2.011 & 0 & 8 & 0.024 & -1.124\\

 & General, no feedback & 40 & 4.575 & 0.288 & 1.824 & 1 & 9 & 0.328 & -0.484\\

\multirow{-5}{*}{\raggedright\arraybackslash Total intrusors} & Rereading & 40 & 4.625 & 0.350 & 2.215 & 1 & 10 & 0.272 & -0.537\\
\bottomrule
\end{tabular}}
\end{table*}

Prior to analysing the data, we have excluded participants based on a
priori set criteria. Participants who have spent less than or equal to
90 seconds on the practice text were excluded (1 exclusion). Further, we
wanted to exclude participants who have had no correct answers on the
final test (0 exclusions). Finally, we have excluded participants who
have stated that they have reading deficits (3 exclusions). This left us
with a total sample of 203 participants. The descriptives for the sample
are shown in Table \ref{descTable}. There is another set of exclusion
criteria based on the number of times the participants have read each of
the three texts. These are used in robustness check analyses (see
suplementary materials).

\hypertarget{interpolated-activity-effect}{%
\subsubsection{Interpolated activity
effect}\label{interpolated-activity-effect}}

Our first two hypotheses are concerned with the effects of different
interpolated activities on the total number of correct answers and total
number of intrusive distractors chosen. To test these hypotheses, we
have focused only on the groups which have not received feedback, since
there was no feedback option for the rereading group (\(n\) = 122). We
conducted a one-way MANOVA with interpolated activity as the independent
variable and the total number of correct and intrusive options chosen as
dependent variables. The correlation between our DVs calculated on the
whole sample is -0.71 (95\% CI: {[}-0.77, -0.63{]}, \(p\) =
\(4.79255\times 10^{-32}\)). Boxplots for the groups in this analysis
are shown in Figure \ref{box1}.

\begin{figure*}[h]

{\centering \includegraphics{analyses-main_files/figure-latex/boxPlot-1} 

}

\caption{\label{box1} Boxplots broken down by experimental condition and dependent variable, with overlayed raw scores.}\label{fig:boxPlot}
\end{figure*}

Pillai's V for the analysis is 0.12565, \(p\) = 0.00376 (Wilks'
\(\Lambda\) = 0.875, \(p\) = 0.00327; Hotelling-Lawley's trace = 0.1421,
\(p\) = 0.00285; Roy's largest root = 0.1366, \(p\) = 0.00049). The
effect size, calculated as \(\omega^2_{mult}\) = 0.10949. To further
inspect the relationship of the interpolated activities with our
dependent variables, we have conducted a Roy-Bargmann stepdown analysis,
as suggested by \citeauthor{tabachnick_using_2012}
(\citeyear{tabachnick_using_2012}; a linear discriminant analysis with
the same aim is available in the supplementary materials). The total
number of correct answers was a priori chosen to be the higher priority
variable. Therefore, we first conducted an ANOVA with interpolated
activity type as the indepedent variable and the total number of correct
answers as the dependent variable.

As could be expected, the ANOVA points to an interpolated activity
effect, with \(F(2, 119)\) = 7.54055, \(p\) = 0.00083. Following the
ANOVA, we conducted an ANCOVA, with the total number of correct answers
as the covariate, and the total number of intrusors as the dependent
variable. As could be expected, the results imply a main effect of the
total number of correct answers (\(F(1, 118)\) = 79.67428,
\(p = 6.87333\times 10^{-15}\)). After correcting for the number of
correct answers, there is no evidence for an effect of interpolated
activity on the total number of chosen intrusors (\(F (2, 118)\) =
0.84413, \(p\) = 0.43251). For now, we may claim that we do not have any
evidence to support our second hypothesis that the type of interpolated
activity will have an effect on the number of intrusors.

In order to test our first hypothesis, we have contrasted (i) the
rereading group with the two test groups, and (ii) the two test groups
with each other, taking only the total number of correct answers as the
DV. The first contrast finds no evidence of a difference between the
rereading group and the two test groups (\(t\) = 1.35542, \(p\) =
0.17785, \(g_s\) = 0.19, 95\% CI = {[}-0.19, 0.57{]}, Cohens's
\(U_{3, g_s}\) = 57.6\%, probability of superiority = 55.39\%). However,
there is a difference between the two test groups (\(t\) = 3.61993,
\(p\) = 0.00043 \(g_s\) = 0.66, 95\% CI = {[}0.21, 1.1{]}, Cohens's
\(U_{3, g_s}\) = 74.43\%, probability of superiority = 67.88\%). These
two findings are not in line with our predictions.

\hypertarget{the-interaction-between-feedback-and-interpolated-activity-type}{%
\subsubsection{The interaction between feedback and interpolated
activity
type}\label{the-interaction-between-feedback-and-interpolated-activity-type}}

The remaining hypotheses deal with the effect of feedback on the total
number of correct answers and the total number of intrusors. Therefore,
these analyses are carried out only on the data from participants in the
general and content related test conditions (\(n\) = 203). To test these
hypotheses, we first conducted a two-way MANOVA with interpolated
activity and feedback as independent variables, and total number of
correct answers and total number of intrusors as the dependent
variables.

\onecolumn

\bibliographystyle{apacite}
\bibliography{../../paper/reference.bib}


\end{document}
