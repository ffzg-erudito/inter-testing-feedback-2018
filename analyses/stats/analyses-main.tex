% Options for packages loaded elsewhere
\PassOptionsToPackage{unicode}{hyperref}
\PassOptionsToPackage{hyphens}{url}
%
\documentclass[
  11pt,
]{article}
\usepackage{lmodern}
\usepackage{amssymb,amsmath}
\usepackage{ifxetex,ifluatex}
\ifnum 0\ifxetex 1\fi\ifluatex 1\fi=0 % if pdftex
  \usepackage[T1]{fontenc}
  \usepackage[utf8]{inputenc}
  \usepackage{textcomp} % provide euro and other symbols
\else % if luatex or xetex
  \usepackage{unicode-math}
  \defaultfontfeatures{Scale=MatchLowercase}
  \defaultfontfeatures[\rmfamily]{Ligatures=TeX,Scale=1}
\fi
% Use upquote if available, for straight quotes in verbatim environments
\IfFileExists{upquote.sty}{\usepackage{upquote}}{}
\IfFileExists{microtype.sty}{% use microtype if available
  \usepackage[]{microtype}
  \UseMicrotypeSet[protrusion]{basicmath} % disable protrusion for tt fonts
}{}
\makeatletter
\@ifundefined{KOMAClassName}{% if non-KOMA class
  \IfFileExists{parskip.sty}{%
    \usepackage{parskip}
  }{% else
    \setlength{\parindent}{0pt}
    \setlength{\parskip}{6pt plus 2pt minus 1pt}}
}{% if KOMA class
  \KOMAoptions{parskip=half}}
\makeatother
\usepackage{xcolor}
\IfFileExists{xurl.sty}{\usepackage{xurl}}{} % add URL line breaks if available
\IfFileExists{bookmark.sty}{\usepackage{bookmark}}{\usepackage{hyperref}}
\hypersetup{
  hidelinks,
  pdfcreator={LaTeX via pandoc}}
\urlstyle{same} % disable monospaced font for URLs
\usepackage[margin=1.8cm]{geometry}
\usepackage{graphicx}
\makeatletter
\def\maxwidth{\ifdim\Gin@nat@width>\linewidth\linewidth\else\Gin@nat@width\fi}
\def\maxheight{\ifdim\Gin@nat@height>\textheight\textheight\else\Gin@nat@height\fi}
\makeatother
% Scale images if necessary, so that they will not overflow the page
% margins by default, and it is still possible to overwrite the defaults
% using explicit options in \includegraphics[width, height, ...]{}
\setkeys{Gin}{width=\maxwidth,height=\maxheight,keepaspectratio}
% Set default figure placement to htbp
\makeatletter
\def\fps@figure{htbp}
\makeatother
\setlength{\emergencystretch}{3em} % prevent overfull lines
\providecommand{\tightlist}{%
  \setlength{\itemsep}{0pt}\setlength{\parskip}{0pt}}
\setcounter{secnumdepth}{-\maxdimen} % remove section numbering
\usepackage[natbibapa, sectionbib, tocbib]{apacite}
\usepackage[utf8]{inputenc}
\usepackage[singlelinecheck = off]{caption}
\usepackage{lmodern}
\usepackage{microtype}
\usepackage{multirow}
\usepackage[inline]{enumitem}
\usepackage{array}
\usepackage[htt]{hyphenat}
\usepackage{booktabs}
\usepackage[euler]{textgreek}
\usepackage{float}
\usepackage[doublespacing]{setspace}
\usepackage{fancyhdr}
\captionsetup[table]{width=\textwidth}
\setlength{\parindent}{2em}
\fancyhf{}
\fancyhead[RH]{\thepage}
\renewcommand{\headrulewidth}{0pt}
\pagestyle{fancy}
\hypersetup{colorlinks = true, linkcolor = blue, urlcolor = black, citecolor = blue}
\DeclareCaptionFormat{apa}{#1#2\\[1em]#3}
\captionsetup*[table]{labelsep = none, textfont = it, format = apa, width = .8\textwidth}
\captionsetup*[figure]{labelsep = period, labelfont = it, position = below}
\usepackage{booktabs}
\usepackage{longtable}
\usepackage{array}
\usepackage{multirow}
\usepackage{wrapfig}
\usepackage{float}
\usepackage{colortbl}
\usepackage{pdflscape}
\usepackage{tabu}
\usepackage{threeparttable}
\usepackage{threeparttablex}
\usepackage[normalem]{ulem}
\usepackage{makecell}
\usepackage{xcolor}

\author{}
\date{\vspace{-2.5em}}

\begin{document}

\hypertarget{exclusion-criteria}{%
\subsection{Exclusion criteria}\label{exclusion-criteria}}

\begin{table*}

\caption{\label{tab:descTable}\label{descTable}Descriptive statistics for the DVs broken down
                     by experimental condition.}
\centering
\begin{tabular}[t]{llrrrrrr}
\toprule
Measure & Condition & $n$ & $M$ & $SE_M$ & $SD$ & min & max\\
\midrule
 & Content, feedback & 41 & 13.22 & 0.508 & 3.25 & 2 & 19\\

 & Content, no feedback & 42 & 12.79 & 0.465 & 3.02 & 7 & 19\\

 & General, feedback & 40 & 10.97 & 0.533 & 3.37 & 1 & 17\\

 & General, no feedback & 40 & 10.47 & 0.449 & 2.84 & 5 & 16\\

\multirow{-5}{*}{\raggedright\arraybackslash Total correct} & Rereading & 40 & 10.88 & 0.443 & 2.80 & 4 & 17\\
\cmidrule{1-8}
 & Content, feedback & 41 & 3.15 & 0.258 & 1.65 & 0 & 7\\

 & Content, no feedback & 42 & 3.38 & 0.257 & 1.67 & 0 & 7\\

 & General, feedback & 40 & 4.17 & 0.318 & 2.01 & 0 & 8\\

 & General, no feedback & 40 & 4.58 & 0.288 & 1.82 & 1 & 9\\

\multirow{-5}{*}{\raggedright\arraybackslash Total intrusors} & Rereading & 40 & 4.62 & 0.350 & 2.21 & 1 & 10\\
\bottomrule
\end{tabular}
\end{table*}

Prior to analysing the data, we excluded participants based on a priori
set criteria. Participants who spent less than or equal to 90 seconds on
the practice text were excluded (1 exclusion). Further, we wanted to
exclude participants who had no correct answers on the final test (0
exclusions). Finally, we excluded participants who had stated that they
had reading deficits (3 exclusions). This left us with a total sample of
203 participants. The descriptives for the sample are shown in Table
\ref{descTable}.

\hypertarget{interpolated-activity-effect}{%
\subsection{Interpolated activity
effect}\label{interpolated-activity-effect}}

Our first two hypotheses are concerned with the effects of different
interpolated activities on the total number of correct answers and total
number of intrusive distractors chosen. To test these hypotheses, we
focused only on the groups which did not receive feedback (\(n\) = 122).
This was done because there was no feedback option for the rereading
group, and we did not want to treat the feedback and no-feedback
general-knowledge and content-related testing groups as equivalent
without strong evidence supporting that assumption.

The correlation between our DVs calculated on the whole sample is
\(r(201) =\) -.707 (95\% CI: {[}-.77, -.63{]}, \(p\) \textless{} .0001).
Given that we have two dependent variables, which are highly correlated,
we have decided to conduct a one-way MANOVA. According to
\citet{tabachnickUsingMultivariateStatistics2012}, conducting a MANOVA
instead of multiple ANOVAs increases the chance of discovering the
effects of different treatments. Furthermore conducting a MANOVA guards
against the inflation of Type 1 errors due to multiple tests of
correlated dependent variables
\citep{tabachnickUsingMultivariateStatistics2012,fieldDiscoveringStatisticsUsing2012}.
Finally, conducting separate ANOVAs would disregard the correlation
between our two dependent variables
\citep{fieldDiscoveringStatisticsUsing2012}. Therefore, we conducted a
one-way MANOVA with interpolated activity as the independent variable
and the total number of correct and intrusive options chosen as
dependent variables.

A power analysis conducted prior to analyzing the data
\citep[using the G*Power software by][]{faulStatisticalPowerAnalyses2009}
has shown that we should have above 80\% power to detect effects which
fall between small and medium (Cohen's \(f^2\ \gtrsim 0.06\)), at an
\(\alpha\) level of .025, with a sample size of 110 participants. Note
that larger effects are expected based on prior studies.

Pillai's V for the analysis is .126, \(p = .004\) (Wilks' \(\Lambda\) =
.875, \(p = .003\)). The effect size, calculated as
\(\omega^2_{mult} = .109\) (bootstrap
median\footnote{All bootstrap estimates taken from 10000 replications.}
= .132, \(BC_\alpha\) 95\% CI = {[}.011, .202{]})\footnote{
Cohen's \(f^2 = 0.051\) (calculated according to Equation 12 in
\citealp{steynjrEstimatingEffectSize2009}).
}. To further inspect the relationship of the interpolated activities
with our dependent variables, we conducted a Roy-Bargmann stepdown
analysis, as suggested by
\citeauthor{tabachnickUsingMultivariateStatistics2012}
(\citeyear{tabachnickUsingMultivariateStatistics2012}; a linear
discriminant analysis with the same aim is available in the
supplementary materials). According to
\citet{tabachnickUsingMultivariateStatistics2012}, the higher priority
variable can be chosen based on theoretical or practical grounds. Since
the total number of correct answers is the criterion that determines a
student's success in a testing context, we chose this dependent variable
as the higher priority one. Therefore, we first conducted an ANOVA with
interpolated activity type as the independent variable and the total
number of correct answers as the dependent variable.

As could be expected, the ANOVA points to a differential effect of our
conditions on the total number of correct answers, with \(F(2, 119)\) =
7.541, \(p = .001\). Following the ANOVA, we conducted an ANCOVA, with
the total number of correct answers as the covariate, and the total
number of intrusors as the dependent variable. The results imply a main
effect of the total number of correct answers (\(F(1, 118)\) = 79.674,
\(p < .0001\)), but after we took into account the number of correct
answers, we found no evidence for an effect of interpolated activity
type on the total number of chosen intrusors (\(F (2, 118)\) = 0.844,
\(p = .433\)). Thus far, results point to a lack of evidence to support
our second hypothesis that the type of interpolated activity will have
an effect on the number of intrusors.

In order to test our first hypothesis, we contrasted (i) the rereading
group with the two test groups, and (ii) the two test groups with each
other, taking only the total number of correct answers as the DV. The
first contrast found no evidence of a difference between the rereading
group and the two test groups (\(t(119)\) = 1.355, \(p = .178\), \(g_s\)
= 0.19, 95\% CI = {[}-0.19, 0.57{]}, Cohen's \(U_{3, g_s}\) = 57.6\%,
probability of superiority = 55.39\%). Therefore, we cannot conclude
that being in the rereading condition, as opposed to being in one of the
two test groups, leads to different learning outcomes. However, there
was a difference between the two test groups (\(t(119)\) = 3.62,
\(p = .0004\), \(g_s\) = 0.66, 95\% CI = {[}0.21, 1.1{]}, Cohen's
\(U_{3, g_s}\) = 74.43\%, probability of superiority = 67.88\%).
Participants in the content related test group scored higher on the
final test than participants in the general knowledge test condition.
These two findings are not in line with our predictions.

\hypertarget{the-interaction-between-feedback-and-interpolated-activity-type}{%
\subsection{The interaction between feedback and interpolated activity
type}\label{the-interaction-between-feedback-and-interpolated-activity-type}}

The remaining hypotheses deal with the effect of feedback on the total
number of correct answers and the total number of intrusors. Therefore,
these analyses were carried out on the data from participants in the
general and content related test conditions only (\(n\) = 163). To test
these hypotheses, we first conducted a two-way MANOVA with interpolated
activity and feedback as independent variables, and total number of
correct answers and total number of intrusors as the dependent
variables. Again, a power analysis conducted before analysing the data
has shown that we should have above 80\% power to detect effects which
fall between small and medium (Cohen's \(f^2\ \gtrsim 0.05\)), at an
\(\alpha\) level of .025, with a sample size of 145 participants.

Pillai's V for the interpolated activity effect (calculated with type
III sums of squares) is .071, \(p = .003\) (Wilks' \(\Lambda\) = .929,
\(p = .003\)) confirming the main effect of interpolated activity type.
The effect size \(\omega^2_{mult}\) = .065 (bootstrap median = .072,
\(BC_\alpha\) 95\% CI = {[}.008, .140{]}).

On the other hand, we found no evidence for an effect of giving feedback
on the linear combination of our two dependent variables --- Pillai's V
= .003, \(p = .800\) (Wilks' \(\Lambda\) = .997, \(p = .800\)). The
effect size is \(\omega^2_{mult}\) = -.003 (bootstrap median =
.003\footnote{
The \(BC_\alpha\) 95\% CI for this estimate is \([-.006,
.004]\).
\label{bca-ref}}).

Furthermore, we found no evidence for an interaction effect between
activity type and feedback --- Pillai's V = .001, \(p = .941\) (Wilks'
\(\Lambda\) = .999, \(p = .941\)). The effect size \(\omega^2_{mult}\) =
-.005 (bootstrap median = .003\footnote{
The \(BC_\alpha\) 95\% CI = \([-.006,
-.005]\).
Our guess is that this odd result is due to the fact that most of the density is concentrated
around 0, causing an unreliable estimate. The same could be said for the CI in
footnote \ref{bca-ref}.}). Both the feedback and the interaction
estimates of \(\omega^2_{mult}\) are to be considered to be zero, given
their negative values.

Again, we conducted a follow-up Roy-Bargmann stepdown analysis. In the
ANOVA model with the total number of correct answers as the dependent
variable and the type of interpolated activity, feedback and their
interaction as predictors, only the type of activity seems to be
relevant (\(F(1, 159) = 11.2, p = .001\)). This result also shows that
participants in the content related test condition scored higher on the
final test than the participants in the general knowledge test
condition, which should be no surprise given the results of the first
stepdown analysis. In the second step, we fit an ANCOVA model with the
total number of correct answers as the covariate. In this model, the
type of interpolated activity ceases to be a relevant predictor
(\(F(1, 155) = 0.175, p = .676\)). The full models are shown in Table
\ref{rb2-table}.

To summarise, contrary to our expectations, we find no evidence of an
effect of feedback on the total number of correctly answered questions.
Also, we found no evidence for an interaction effect of feedback and
type of interpolated activity on the total number of correct answers.
The same findings apply to the predictions regarding the total number of
intrusors chosen.

\begin{table*}

\caption{\label{tab:rb2Table}\label{rb2-table}ANOVA and ANCOVA models for the second Roy-Bargmann
                     procedure.}
\centering
\begin{tabular}[t]{lrrrr}
\toprule
Term & $SS$ & $df$ & $F$ & $p$\\
\midrule
\addlinespace[0.3em]
\multicolumn{5}{l}{\textbf{ANOVA}}\\
\hspace{1em}Activity & 109.393 & 1 & 11.200 & .001\\
\hspace{1em}Feedback & 3.904 & 1 & 0.400 & .528\\
\hspace{1em}Activity x Feedback & 0.045 & 1 & 0.005 & .946\\
\hspace{1em}Residuals & 1553.046 & 159 &  & \\
\addlinespace[0.3em]
\multicolumn{5}{l}{\textbf{ANCOVA}}\\
\hspace{1em}Activity & 0.301 & 1 & 0.175 & .676\\
\hspace{1em}Feedback & 0.173 & 1 & 0.100 & .752\\
\hspace{1em}Total correct & 63.216 & 1 & 36.760 & < .0001\\
\hspace{1em}Activity x Feedback & 0.813 & 1 & 0.473 & .493\\
\hspace{1em}Activity x Total correct & 0.862 & 1 & 0.501 & .480\\
\hspace{1em}Feedback x Total correct & 0.130 & 1 & 0.075 & .784\\
\hspace{1em}Activity x Feedback x Total correct & 1.229 & 1 & 0.715 & .399\\
\hspace{1em}Residuals & 266.551 & 155 &  & \\
\bottomrule
\end{tabular}
\end{table*}

\hypertarget{deviations-from-the-preregistered-analysis-plan}{%
\subsection{Deviations from the preregistered analysis
plan}\label{deviations-from-the-preregistered-analysis-plan}}

Initially, we had planned to do a robustness check of our findings using
data with an additional exclusion criterion, based on the number of
times each participant had read each of the three parts of the main
text. This analysis was never conducted because (i) applying this
criterion would have lead to unacceptably low power and (ii) the
participants' estimates of the number of times they had read each part
were similarly distributed across all conditions. Further, we had
planned to conduct a TOST procedure to test whether there is no
difference between the content-related and general-knowledge testing
groups. This analysis was not conducted because we did find a
difference. A Bayesian t-test was also considered for the same
comparison, but was dropped early on due to some conceptual concerns.

\end{document}
