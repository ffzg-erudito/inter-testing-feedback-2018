\documentclass[11pt,]{article}
\usepackage{lmodern}
\usepackage{amssymb,amsmath}
\usepackage{ifxetex,ifluatex}
\usepackage{fixltx2e} % provides \textsubscript
\ifnum 0\ifxetex 1\fi\ifluatex 1\fi=0 % if pdftex
  \usepackage[T1]{fontenc}
  \usepackage[utf8]{inputenc}
\else % if luatex or xelatex
  \ifxetex
    \usepackage{mathspec}
  \else
    \usepackage{fontspec}
  \fi
  \defaultfontfeatures{Ligatures=TeX,Scale=MatchLowercase}
\fi
% use upquote if available, for straight quotes in verbatim environments
\IfFileExists{upquote.sty}{\usepackage{upquote}}{}
% use microtype if available
\IfFileExists{microtype.sty}{%
\usepackage{microtype}
\UseMicrotypeSet[protrusion]{basicmath} % disable protrusion for tt fonts
}{}
\usepackage[margin=1.8cm]{geometry}
\usepackage{hyperref}
\hypersetup{unicode=true,
            pdfborder={0 0 0},
            breaklinks=true}
\urlstyle{same}  % don't use monospace font for urls
\usepackage{graphicx,grffile}
\makeatletter
\def\maxwidth{\ifdim\Gin@nat@width>\linewidth\linewidth\else\Gin@nat@width\fi}
\def\maxheight{\ifdim\Gin@nat@height>\textheight\textheight\else\Gin@nat@height\fi}
\makeatother
% Scale images if necessary, so that they will not overflow the page
% margins by default, and it is still possible to overwrite the defaults
% using explicit options in \includegraphics[width, height, ...]{}
\setkeys{Gin}{width=\maxwidth,height=\maxheight,keepaspectratio}
\IfFileExists{parskip.sty}{%
\usepackage{parskip}
}{% else
\setlength{\parindent}{0pt}
\setlength{\parskip}{6pt plus 2pt minus 1pt}
}
\setlength{\emergencystretch}{3em}  % prevent overfull lines
\providecommand{\tightlist}{%
  \setlength{\itemsep}{0pt}\setlength{\parskip}{0pt}}
\setcounter{secnumdepth}{0}
% Redefines (sub)paragraphs to behave more like sections
\ifx\paragraph\undefined\else
\let\oldparagraph\paragraph
\renewcommand{\paragraph}[1]{\oldparagraph{#1}\mbox{}}
\fi
\ifx\subparagraph\undefined\else
\let\oldsubparagraph\subparagraph
\renewcommand{\subparagraph}[1]{\oldsubparagraph{#1}\mbox{}}
\fi

%%% Use protect on footnotes to avoid problems with footnotes in titles
\let\rmarkdownfootnote\footnote%
\def\footnote{\protect\rmarkdownfootnote}

%%% Change title format to be more compact
\usepackage{titling}

% Create subtitle command for use in maketitle
\providecommand{\subtitle}[1]{
  \posttitle{
    \begin{center}\large#1\end{center}
    }
}

\setlength{\droptitle}{-2em}

  \title{}
    \pretitle{\vspace{\droptitle}}
  \posttitle{}
    \author{}
    \preauthor{}\postauthor{}
    \date{}
    \predate{}\postdate{}
  
\usepackage{booktabs}
\usepackage{longtable}
\usepackage{array}
\usepackage{multirow}
\usepackage{wrapfig}
\usepackage{float}
\usepackage{colortbl}
\usepackage{pdflscape}
\usepackage{tabu}
\usepackage{threeparttable}
\usepackage{threeparttablex}
\usepackage[normalem]{ulem}
\usepackage{makecell}
\usepackage{xcolor}

\usepackage[natbibapa, sectionbib, tocbib]{apacite}
\usepackage[utf8]{inputenc}
\usepackage{caption}
\usepackage{lmodern}
\usepackage{multirow}
\usepackage[inline]{enumitem}
\usepackage{array}
\usepackage[htt]{hyphenat}
\usepackage{booktabs}
\usepackage[euler]{textgreek}
\usepackage{float}
\usepackage[onehalfspacing]{setspace}
\captionsetup[table]{width=\textwidth}
\hypersetup{colorlinks = true, linkcolor = blue, urlcolor = red}

\begin{document}

\hypertarget{participants-and-design}{%
\subsection{Participants and design}\label{participants-and-design}}

Undergraduate and graduate phontecis and psychology students (80.8\%
female, median age = 21, IQR = 3, range = {[}18, 31{]}, total \(N\) =
207) from the University of Zagreb participated in the study in exchange
for course credit. Participants were randomly assigned to one of five
groups which differed in the type of activity they engaged in between
parts of the text they have read and in whether they received feedback
on their intermittent test achievement or not.

\hypertarget{materials-and-procedure}{%
\subsection{Materials and procedure}\label{materials-and-procedure}}

\hypertarget{materials}{%
\subsubsection{Materials}\label{materials}}

Participants read a text on the evolution, ecological and biological
characteristics of weeds. The text was taken from a chapter in a
Croatian university-level textbook. Some sentences and passages were
slightly modified, so as to avoid odd language constructions; Latin
plant names were translated to Croatian, and some plants were removed
from the text to make it less difficult for the target participant
population. The text was divided into three parts of 874, 754, and 835
words, respectively. Additionaly, there was a practice text taken from
the same chapter, but unrelated to any of the other three parts of the
text (768 words).

Forty-four content related questions with four response options were
generated from the presented texts. Four questions were presented after
the practice text, ten after each of the first two parts (only to the
participatns in the content related test condition), and twenty after
the third part of the text (to all participants). Starting from the
second ten question set, the distractor options were chosen so that (a)
two distractors were plausible, but unrelated to the text, and (b) one
distractor was a term or concept mentioned in the previous part of the
text --- this was considered to be the ``intrusive'' option.

An example question is:

\begin{quotation}
\noindent Compared to younger weeds, older weeds:
\begin{enumerate}[label = (\alph*)]
\item have a stronger allelopathic effect
\item contain more phytotoxins
\item \label{optCorr} \textbf{contain less inhibitory matter}
\item \label{optIntr} \textit{show greater plasticity.}
\end{enumerate}
\end{quotation}

Option \ref{optCorr} is the correct answer, and option \ref{optIntr} is
the intrusive distractor.

Further, twenty-four general knowledge questions were generated. These
questions were presented to participants in the general knowledge test
condition, after the first two parts of the text. An example general
knowledge question is:

\begin{quotation}
\noindent The name of Kurt Vonnegut's famous anti-war novel is:
\begin{enumerate}[label = (\alph*)]
\item \textbf{Slaughterhouse Five}
\item All Quiet on the Western Front
\item A Farewell to Arms
\item Journey to the End of the Night.
\end{enumerate}
\end{quotation}

At the beginning of the session, participants' ID, age and sex
information was collected. At the end of the session, participants were
asked to estimate how much of each text they have read. The texts and
questions were presented on a personal computer, in an application
constructed using the open source \texttt{oTree} framework
\citep[version 2.1.35,][]{chen_otreeopen-source_2016} for the Python
programming language (version 3.6.4, October 20, 2018).

\hypertarget{procedure}{%
\subsubsection{Procedure}\label{procedure}}

Participants were first given a brief introduction to the study, and
were encouraged to carefully read and follow the written instructions.
Then, they were led to one of six compartments containing a computer,
which was running a fullscreen instance of the \texttt{oTree}
application with a randomly chosen experimental condition. There,
participants read the informed consent form and, in case there were no
questions, started the experiment.

After entering their personal information, participants were presented
with instructions for their first task, which was to read the practice
text at a speed that comes naturally to them. They were to click a
button at the bottom of the text when they have finished reading it.
Unbeknownst to the participants, the time they took to read the practice
text was recorded, and used as the basis for determining the reading
time limits for the remaining texts. Results of a pilot study using
different participants have shown that most participants found 4 minutes
to be too short, and 9 minutes too long, so we have set the lowest
possible limit to 5 minutes, and the longest to 8 minutes.

Next, participants were familiarised with the interpolated activity they
were going to perform during the main part of the procedure. The
rereading group reread the practice text (this time with the time limit
applied), the general knowledge test group answered four general
knowledge questions, and the content related test group answered four
questions based on the practice text.

Participants assigned to the feedback condition also received feedback
on their practice test achievement. Feedback was presented on a separate
screen, which listed the questions, the participant's answers, and the
correct answers in tabular format. Incorrectly answered questions were
highlighted in red. After 40 seconds elapsed, a ``Next'' button
appeared, allowing participants to proceed to the next text. This way,
we wanted to prevent participants from simply clicking through the
feedback, hoping that they will spend their time examining it. The
feedback was presented for maximally 60 seconds, after which the
application proceeded to the next text.

Subjects in the rereading and general knowledge conditions also answered
the four questions related to the practice text, so as to familiarise
them with the scope and specificity level of questions that they will
receive after reading the final text. All participants were told that
there would be a cumulative test after the final text, examining their
knowledge of the three texts following the practice text. In reality,
the final test examined only the knowledge of the final text.

After the practice round, participants proceeded to read the main texts,
engaging in the interpolated activities they were assigned. After the
third text, all participatns were presented with twenty questions
examining their knowledge of the third text. The computer recorded
whether a participant correctly answered a question and whether the
participant chose an intrusive distractor. This allowed us to compute
our dependent variables --- the total number of correct answers and the
total number of intrusive distractors chosen.


\end{document}
