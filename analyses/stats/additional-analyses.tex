\documentclass[11pt,]{article}
\usepackage{lmodern}
\usepackage{amssymb,amsmath}
\usepackage{ifxetex,ifluatex}
\usepackage{fixltx2e} % provides \textsubscript
\ifnum 0\ifxetex 1\fi\ifluatex 1\fi=0 % if pdftex
  \usepackage[T1]{fontenc}
  \usepackage[utf8]{inputenc}
\else % if luatex or xelatex
  \ifxetex
    \usepackage{mathspec}
  \else
    \usepackage{fontspec}
  \fi
  \defaultfontfeatures{Ligatures=TeX,Scale=MatchLowercase}
\fi
% use upquote if available, for straight quotes in verbatim environments
\IfFileExists{upquote.sty}{\usepackage{upquote}}{}
% use microtype if available
\IfFileExists{microtype.sty}{%
\usepackage{microtype}
\UseMicrotypeSet[protrusion]{basicmath} % disable protrusion for tt fonts
}{}
\usepackage[margin=1.8cm]{geometry}
\usepackage{hyperref}
\hypersetup{unicode=true,
            pdfborder={0 0 0},
            breaklinks=true}
\urlstyle{same}  % don't use monospace font for urls
\usepackage{graphicx,grffile}
\makeatletter
\def\maxwidth{\ifdim\Gin@nat@width>\linewidth\linewidth\else\Gin@nat@width\fi}
\def\maxheight{\ifdim\Gin@nat@height>\textheight\textheight\else\Gin@nat@height\fi}
\makeatother
% Scale images if necessary, so that they will not overflow the page
% margins by default, and it is still possible to overwrite the defaults
% using explicit options in \includegraphics[width, height, ...]{}
\setkeys{Gin}{width=\maxwidth,height=\maxheight,keepaspectratio}
\IfFileExists{parskip.sty}{%
\usepackage{parskip}
}{% else
\setlength{\parindent}{0pt}
\setlength{\parskip}{6pt plus 2pt minus 1pt}
}
\setlength{\emergencystretch}{3em}  % prevent overfull lines
\providecommand{\tightlist}{%
  \setlength{\itemsep}{0pt}\setlength{\parskip}{0pt}}
\setcounter{secnumdepth}{0}
% Redefines (sub)paragraphs to behave more like sections
\ifx\paragraph\undefined\else
\let\oldparagraph\paragraph
\renewcommand{\paragraph}[1]{\oldparagraph{#1}\mbox{}}
\fi
\ifx\subparagraph\undefined\else
\let\oldsubparagraph\subparagraph
\renewcommand{\subparagraph}[1]{\oldsubparagraph{#1}\mbox{}}
\fi

%%% Use protect on footnotes to avoid problems with footnotes in titles
\let\rmarkdownfootnote\footnote%
\def\footnote{\protect\rmarkdownfootnote}

%%% Change title format to be more compact
\usepackage{titling}

% Create subtitle command for use in maketitle
\providecommand{\subtitle}[1]{
  \posttitle{
    \begin{center}\large#1\end{center}
    }
}

\setlength{\droptitle}{-2em}

  \title{}
    \pretitle{\vspace{\droptitle}}
  \posttitle{}
    \author{}
    \preauthor{}\postauthor{}
    \date{}
    \predate{}\postdate{}
  
\usepackage{booktabs}
\usepackage{longtable}
\usepackage{array}
\usepackage{multirow}
\usepackage{wrapfig}
\usepackage{float}
\usepackage{colortbl}
\usepackage{pdflscape}
\usepackage{tabu}
\usepackage{threeparttable}
\usepackage{threeparttablex}
\usepackage[normalem]{ulem}
\usepackage{makecell}
\usepackage{xcolor}

\usepackage[natbibapa, sectionbib, tocbib]{apacite}
\usepackage[utf8]{inputenc}
\usepackage[singlelinecheck = off]{caption}
\usepackage{lmodern}
\usepackage{microtype}
\usepackage{multirow}
\usepackage[inline]{enumitem}
\usepackage{array}
\usepackage[htt]{hyphenat}
\usepackage{booktabs}
\usepackage[euler]{textgreek}
\usepackage{float}
\usepackage[doublespacing]{setspace}
\usepackage{fancyhdr}
\captionsetup[table]{width=\textwidth}
\setlength{\parindent}{2em}
\fancyhf{}
\fancyhead[RH]{\thepage}
\renewcommand{\headrulewidth}{0pt}
\pagestyle{fancy}
\hypersetup{colorlinks = true, linkcolor = blue, urlcolor = black, citecolor = blue}
\DeclareCaptionFormat{apa}{#1#2\\[1em]#3}
\captionsetup*[table]{labelsep = none, textfont = it, format = apa, width = .8\textwidth}
\captionsetup*[figure]{labelsep = period, labelfont = it, position = below}

\begin{document}

\hypertarget{additional-analyses}{%
\subsection{Additional analyses}\label{additional-analyses}}

Because it is theoretically interesting to see whether there is evidence
for absence of a difference between certain conditions, or no effect of
certain manipulations, we conducted a Bayesian reanalysis of the two
Roy-Bargmann stepdown procedures. Since these analyses had not been
planned, we decided to use the default priors provided in the
\textit{BayesFactor} \citep{moreyBayesFactorComputationBayes2018}
package.\footnote{All posteriors obtained from 6000 simulations.}

\hypertarget{bayesian-reanalysis-of-the-first-roy-bargmann-procedure}{%
\subsubsection{Bayesian reanalysis of the first Roy-Bargmann
procedure}\label{bayesian-reanalysis-of-the-first-roy-bargmann-procedure}}

As was earlier done in a frequentist setting, we first fit an ANOVA
model with the total number of correct answers as the dependent
variable, and the type of interpolated activity as the predictor. All
effects are expressed as deviations from the estimated posterior
subsample mean of 11.381. The estimated mean of the effect of content
related testing is 1.254 (95\% HDI = {[}0.553, 2.005{]}). The 95\%
highest density interval of the posterior indicates that there is a fair
amount of uncertainty around the exact magnitude of the effect of
content-related testing. However, most of the probability density is
quite far above zero, implying that there really is a positive effect.
The means of the posterior distributions for the general-knowledge-test
and rereading conditions \(b\)s are -0.805 (95\% HDI = {[}-1.549,
-0.116{]}) and -0.449, (95\% HDI = {[}-1.125, 0.257{]}) respectively.
Most of the posterior distribution for the effect of general knowledge
testing lies below zero, pointing to a negative effect on the total
number of correct answers, although the distance is not as marked as in
the content-related condition. On the other hand, there is a lot of
uncertainty about the effect of rereading, compared to the other two
estimates. Still, 89.8\% of the posterior lies below zero, which lead us
to believe that the effect is most likely negative.

Furthermore, we wanted to explore the difference between the rereading
and general-knowledge-test conditions, given their somewhat similar
coefficient and HDI estimates, as well as sample means. To do this, we
conducted a Bayesian t-test, again with the \textit{BayesFactor}
package's default priors. The estimated posterior mean of the difference
in the total number of correct answers between the two groups is -0.362
(95\% HDI = {[}-1.49, 0.856{]}). As can be seen from the HDI, there is a
lot of uncertainty around the estimate of the difference, which points
to a lack of evidence for any claim regarding the effect.

In the second step of the Roy-Bargmann procedure, we fit an ANCOVA model
with the total number of correct answers as the covariate and the total
number of intrusive options chosen as the dependent variable. Effects
are again expressed relative to the estimated posterior subsample mean
of 4.193. There is uncertainty around the estimates of the effects of
the different experimental conditions --- content related testing \(b\)
= -0.214 (95\% HDI = {[}-0.583, 0.146{]}), general-knowledge testing
\(b\) = 0.072 (95\% HDI = {[}-0.288, 0.424{]}), rereading \(b\) = 0.142
(95\% HDI = {[}-0.216, 0.494{]}). The HDIs show that there could be
either a slight increase or a slight decrease in the number of
intrusors, which prevented us from making a conclusion about the nature
of the effects. However, given the current data and priors, we find the
following --- 87.43\% of the posterior for the effect of content related
testing falls below zero; 65.57\% of the posterior for the effect of
general knowledge testing falls above zero; 77.68\% of the posterior for
the effect of rereading falls above zero. Given the stated, there is
some evidence implying that content related testing decreases the number
of intrusors chosen, after controlling for the effect of the total
number of correct answers. Further, there is some, albeit weaker
evidence that rereading leads to an increase in the number of chosen
intrusive distractors. Lastly, the posterior of the general knowledge
testing effect points to no particular direction. A stronger test of
these claims is desired.

\hypertarget{bayesian-reanalysis-of-the-second-roy-bargmann-procedure}{%
\subsubsection{Bayesian reanalysis of the second Roy-Bargmann
procedure}\label{bayesian-reanalysis-of-the-second-roy-bargmann-procedure}}

In the second Roy-Bargmann analysis, we wanted to test whether there is
an effect of the type of interpolated activity, receiving feedback, and
their interaction on the total number of correct answers and chosen
intrusors. Again, we first fit an ANOVA model with the two predictors
and the total number of correct answers as the dependent variable.

Effects are expressed relative to the estimated posterior subsample mean
of 11.868. We found that content related testing leads to an increase in
the total number of correct answers, \(b\) = 1.086 (95\% HDI = {[}0.589,
1.559{]}), compared to the general knowledge testing. This is aligned
with the finding obtained in the frequentist setting. The mean of the
posterior for the effect of receiving feedback is 0.218 (95\% HDI =
{[}-0.251, 0.679{]}). The HDI around the estimate precludes any firm
conclusions regarding the effect of receiving feedback. However, we will
mention that 82.25\% of the posterior lies above zero, implying a
possible positive effect on learning. Finally, the estimate for the
interaction effect (being in the content condition and receiving
feedback) is -0.013 (95\% HDI = {[}-0.46, 0.432{]}). This could point to
there not being a relevant interaction effect. According to the
collected data and the priors, we could claim that the effect is
practically equivalent to zero if we were not interested in a half-point
increase or decrease in the average scores (i.e.~defining a region of
practical equivalence (ROPE) between {[}-0.5, 0.5{]}). Still, greater
precision, which would require further data collection, is desired.

We continue with the ANCOVA model, taking the total number of correct
answers as the covariate. The estimate of the intercept is 3.821 (95\%
HDI = {[}3.6, 4.03{]}). The estimate for the effect of content related
testing on the total number of intrusive distractors chosen is \(b\) =
-0.118 (95\% HDI = {[}-0.325, 0.092{]}), compared to general knowledge
testing. There is some evidence for a slight decrease in the number of
intrusive distractors chosen in the content related testing condition.
However, an increase is also possible, but less likely and negligibly
small. The estimate for the effect of receiving feedback is -0.091 (95\%
HDI = {[}-0.302, 0.121{]}). Although the mean of the posterior is close
to zero, the lower bound of the HDI shows that values which may be
considered non-negligible are still somewhat probable. Therefore, we
shall refrain from making a judgement regarding the effect of feedback
on choosing intrusive distractors. Finally, the estimate of the
interaction effect is \(b\) = 0.047 (95\% HDI = {[}-0.153, 0.244{]}).
The mean of the posterior is close to zero, and we could declare the
effect to be practically equivalent to zero with a ROPE of approximately
{[}-0.25, 0.25{]}.

As previously stated, all these analyses were not planned a priori. This
warrants certain caveats. The \textit{BayesFactor} package's default
priors were used. The appropriateness of these priors should certainly
be questioned. However, we decided to use them because we did not want
to choose priors after already seeing the data, which would have been
more problematic. Further, the statements about effects made in this
section are noncommittal. Whether a 0.5 increase or decrease in the
total number of correct answers is practically equivalent to zero or not
is left to the reader.

\bibliographystyle{apacite}
\bibliography{../../paper/reference}


\end{document}
