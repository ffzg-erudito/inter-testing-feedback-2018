\documentclass[11pt,]{article}
\usepackage{lmodern}
\usepackage{amssymb,amsmath}
\usepackage{ifxetex,ifluatex}
\usepackage{fixltx2e} % provides \textsubscript
\ifnum 0\ifxetex 1\fi\ifluatex 1\fi=0 % if pdftex
  \usepackage[T1]{fontenc}
  \usepackage[utf8]{inputenc}
\else % if luatex or xelatex
  \ifxetex
    \usepackage{mathspec}
  \else
    \usepackage{fontspec}
  \fi
  \defaultfontfeatures{Ligatures=TeX,Scale=MatchLowercase}
\fi
% use upquote if available, for straight quotes in verbatim environments
\IfFileExists{upquote.sty}{\usepackage{upquote}}{}
% use microtype if available
\IfFileExists{microtype.sty}{%
\usepackage{microtype}
\UseMicrotypeSet[protrusion]{basicmath} % disable protrusion for tt fonts
}{}
\usepackage[margin=1.8cm]{geometry}
\usepackage{hyperref}
\hypersetup{unicode=true,
            pdfborder={0 0 0},
            breaklinks=true}
\urlstyle{same}  % don't use monospace font for urls
\usepackage{graphicx,grffile}
\makeatletter
\def\maxwidth{\ifdim\Gin@nat@width>\linewidth\linewidth\else\Gin@nat@width\fi}
\def\maxheight{\ifdim\Gin@nat@height>\textheight\textheight\else\Gin@nat@height\fi}
\makeatother
% Scale images if necessary, so that they will not overflow the page
% margins by default, and it is still possible to overwrite the defaults
% using explicit options in \includegraphics[width, height, ...]{}
\setkeys{Gin}{width=\maxwidth,height=\maxheight,keepaspectratio}
\IfFileExists{parskip.sty}{%
\usepackage{parskip}
}{% else
\setlength{\parindent}{0pt}
\setlength{\parskip}{6pt plus 2pt minus 1pt}
}
\setlength{\emergencystretch}{3em}  % prevent overfull lines
\providecommand{\tightlist}{%
  \setlength{\itemsep}{0pt}\setlength{\parskip}{0pt}}
\setcounter{secnumdepth}{0}
% Redefines (sub)paragraphs to behave more like sections
\ifx\paragraph\undefined\else
\let\oldparagraph\paragraph
\renewcommand{\paragraph}[1]{\oldparagraph{#1}\mbox{}}
\fi
\ifx\subparagraph\undefined\else
\let\oldsubparagraph\subparagraph
\renewcommand{\subparagraph}[1]{\oldsubparagraph{#1}\mbox{}}
\fi

%%% Use protect on footnotes to avoid problems with footnotes in titles
\let\rmarkdownfootnote\footnote%
\def\footnote{\protect\rmarkdownfootnote}

%%% Change title format to be more compact
\usepackage{titling}

% Create subtitle command for use in maketitle
\providecommand{\subtitle}[1]{
  \posttitle{
    \begin{center}\large#1\end{center}
    }
}

\setlength{\droptitle}{-2em}

  \title{}
    \pretitle{\vspace{\droptitle}}
  \posttitle{}
    \author{}
    \preauthor{}\postauthor{}
    \date{}
    \predate{}\postdate{}
  
\usepackage{booktabs}
\usepackage{longtable}
\usepackage{array}
\usepackage{multirow}
\usepackage{wrapfig}
\usepackage{float}
\usepackage{colortbl}
\usepackage{pdflscape}
\usepackage{tabu}
\usepackage{threeparttable}
\usepackage{threeparttablex}
\usepackage[normalem]{ulem}
\usepackage{makecell}
\usepackage{xcolor}

\usepackage[natbibapa, sectionbib, tocbib]{apacite}
\usepackage[utf8]{inputenc}
\usepackage{caption}
\usepackage{lmodern}
\usepackage{multirow}
\usepackage[inline]{enumitem}
\usepackage{array}
\usepackage[htt]{hyphenat}
\usepackage{booktabs}
\usepackage[euler]{textgreek}
\usepackage{float}
\usepackage[onehalfspacing]{setspace}
\captionsetup[table]{width=\textwidth}
\hypersetup{colorlinks = true, linkcolor = blue, urlcolor = black, citecolor = blue}

\begin{document}

\hypertarget{participants-and-design}{%
\subsection{Participants and design}\label{participants-and-design}}

Undergraduate and graduate phontecis and psychology students (80.8\%
female, median age = 21, IQR = 3, range = {[}18, 31{]}, total \(N\) =
207) from the University of Zagreb participated in the study in exchange
for course credit. Participants were randomly assigned to one of five
groups which differed in the type of activity they engaged in between
parts of the text they have read and in whether they received feedback
on their intermittent test achievement or not.

\hypertarget{materials-and-procedure}{%
\subsection{Materials and procedure}\label{materials-and-procedure}}

\hypertarget{materials}{%
\subsubsection{Materials}\label{materials}}

Participants read a text on the evolution, ecological and biological
characteristics of weeds. The text was taken from a chapter in a
Croatian university-level textbook. Some sentences and passages were
slightly modified, so as to avoid odd language constructions; Latin
plant names were translated to Croatian, and some plants were removed
from the text to make it less difficult for the target participant
population. The text was divided into three parts of 874, 754, and 835
words, respectively. Additionaly, there was a practice text taken from
the same chapter, but unrelated to any of the other three parts of the
text (768 words).

Forty-four content related questions with four response options were
generated from the presented texts. Four questions were presented after
the practice text, ten after each of the first two parts (only to the
participatns in the content related test condition), and twenty after
the third part of the text (to all participants). Starting from the
second ten-question-set, the distractor options were chosen so that (a)
two distractors were plausible, but unrelated to the text, and (b) one
distractor was a term or concept mentioned in the previous part of the
text --- this was considered to be the ``intrusive'' option.

An example question is:

\begin{quotation}
\noindent Compared to younger weeds, older weeds:
\begin{enumerate}[label = (\alph*)]
\item have a stronger allelopathic effect
\item contain more phytotoxins
\item \label{optCorr} \textbf{contain less inhibitory matter}
\item \label{optIntr} \textit{show greater plasticity.}
\end{enumerate}
\end{quotation}

Option \ref{optCorr} is the correct answer, and option \ref{optIntr} is
the intrusive distractor.

Further, twenty-four general knowledge questions were generated. These
questions were presented to participants in the general knowledge test
condition, after the first two parts of the text. An example general
knowledge question is:

\begin{quotation}
\noindent The name of Kurt Vonnegut's famous anti-war novel is:
\begin{enumerate}[label = (\alph*)]
\item \textbf{Slaughterhouse Five}
\item All Quiet on the Western Front
\item A Farewell to Arms
\item Journey to the End of the Night.
\end{enumerate}
\end{quotation}

At the beginning of the session, participants' ID, age and sex
information was collected. At the end of the session, participants were
asked to estimate how much of each text they have read. The texts and
questions were presented on a personal computer, in an application
constructed using the open source \texttt{oTree} framework
\citep[version 2.1.35,][]{chen_otreeopen-source_2016} for the Python
programming language (version 3.6.4, October 20, 2018).

\hypertarget{procedure}{%
\subsubsection{Procedure}\label{procedure}}

Participants were first given a brief introduction to the study, and
were encouraged to carefully read and follow the written instructions.
Then, they were led to one of six compartments containing a computer,
which was running a fullscreen instance of the \texttt{oTree}
application with a randomly chosen experimental condition. There,
participants read the informed consent form and, in case there were no
questions, started the experiment.

After entering their personal information, participants were presented
with instructions for their first task, which was to read the practice
text at a speed that comes naturally to them. They were to click a
button at the bottom of the text when they have finished reading it.
Unbeknownst to the participants, the time they took to read the practice
text was recorded, and used as the basis for determining the reading
time limits for the remaining texts. Results of a pilot study using
different participants have shown that most participants found 4 minutes
to be too short, and 9 minutes too long, so we have set the lowest
possible limit to 5 minutes, and the longest to 8 minutes.

Next, participants were familiarised with the interpolated activity they
were going to perform during the main part of the procedure. The
rereading group reread the practice text (this time with the time limit
applied), the general knowledge test group answered four general
knowledge questions, and the content related test group answered four
questions based on the practice text.

Participants assigned to the feedback condition also received feedback
on their practice test achievement. Feedback was presented on a separate
screen, which listed the questions, the participant's answers, and the
correct answers in tabular format. Incorrectly answered questions were
highlighted in red. After 40 seconds elapsed, a ``Next'' button
appeared, allowing participants to proceed to the next text. This way,
we wanted to prevent participants from simply clicking through the
feedback, hoping that they will spend their time examining it. The
feedback was presented for maximally 60 seconds, after which the
application proceeded to the next text.

Subjects in the rereading and general knowledge conditions also answered
the four questions related to the practice text, so as to familiarise
themselves with the scope and specificity level of questions that they
will receive after reading the final text. All participants were told
that there would be a cumulative test after the final text, examining
their knowledge of the three texts following the practice text. In
reality, the final test examined only the knowledge of the final text.

After the practice round, participants proceeded to the main texts,
engaging in the interpolated activities they were assigned. After the
third text, all participatns were presented with twenty questions
examining their knowledge of the third text. The computer recorded
whether a participant correctly answered a question and whether the
participant chose an intrusive distractor. This allowed us to compute
our dependent variables --- the total number of correct answers and the
total number of intrusive distractors chosen.

\hypertarget{results}{%
\subsection{Results}\label{results}}

\hypertarget{exclusion-criteria}{%
\subsubsection{Exclusion criteria}\label{exclusion-criteria}}

\begin{table*}[t]

\caption{\label{tab:descTable}\label{descTable} Descriptive statistics for the
                     number of correct answers and chosen intrusors broken down
                     by experimental condition.}
\centering
\resizebox{\linewidth}{!}{
\begin{tabular}{llrrrrrrrr}
\toprule
Measure & Condition & $n$ & $M$ & $SE$ & $SD$ & min & max & skew & kurtosis\\
\midrule
 & Content, feedback & 41 & 13.22 & 0.508 & 3.25 & 2 & 19 & -0.800 & 1.503\\

 & Content, no feedback & 42 & 12.79 & 0.465 & 3.02 & 7 & 19 & 0.039 & -0.775\\

 & General, feedback & 40 & 10.97 & 0.533 & 3.37 & 1 & 17 & -0.481 & 0.462\\

 & General, no feedback & 40 & 10.47 & 0.449 & 2.84 & 5 & 16 & -0.053 & -0.986\\

\multirow{-5}{*}{\raggedright\arraybackslash Total correct} & Rereading & 40 & 10.88 & 0.443 & 2.80 & 4 & 17 & -0.141 & -0.253\\
\cmidrule{1-10}
 & Content, feedback & 41 & 3.15 & 0.258 & 1.65 & 0 & 7 & 0.292 & -0.351\\

 & Content, no feedback & 42 & 3.38 & 0.257 & 1.67 & 0 & 7 & 0.203 & -0.385\\

 & General, feedback & 40 & 4.17 & 0.318 & 2.01 & 0 & 8 & 0.024 & -1.124\\

 & General, no feedback & 40 & 4.58 & 0.288 & 1.82 & 1 & 9 & 0.328 & -0.484\\

\multirow{-5}{*}{\raggedright\arraybackslash Total intrusors} & Rereading & 40 & 4.62 & 0.350 & 2.21 & 1 & 10 & 0.272 & -0.537\\
\bottomrule
\end{tabular}}
\end{table*}

Prior to analysing the data, we have excluded participants based on a
priori set criteria. Participants who have spent less than or equal to
90 seconds on the practice text were excluded (1 exclusion). Further, we
wanted to exclude participants who have had no correct answers on the
final test (0 exclusions). Finally, we have excluded participants who
have stated that they have reading deficits (3 exclusions). This left us
with a total sample of 203 participants. The descriptives for the sample
are shown in Table \ref{descTable}. There is another set of exclusion
criteria based on the number of times the participants have read each of
the three texts. These are used in robustness check analyses (see
suplementary materials).

\hypertarget{interpolated-activity-effect}{%
\subsubsection{Interpolated activity
effect}\label{interpolated-activity-effect}}

Our first two hypotheses are concerned with the effects of different
interpolated activities on the total number of correct answers and total
number of intrusive distractors chosen. To test these hypotheses, we
have focused only on the groups which have not received feedback, since
there was no feedback option for the rereading group (\(n\) = 122). We
conducted a one-way MANOVA with interpolated activity as the independent
variable and the total number of correct and intrusive options chosen as
dependent variables. The correlation between our DVs calculated on the
whole sample is -0.71 (95\% CI: {[}-0.77, -0.63{]}, \(p\) =
\(4.793\times 10^{-32}\)). Boxplots for the groups in this analysis are
shown in Figure \ref{box1}.

\begin{figure*}[h]

{\centering \includegraphics{methods-results_files/figure-latex/boxPlot-1} 

}

\caption{\label{box1} Boxplots broken down by experimental conditions included in the first MANOVA, and dependent variable, with overlayed raw scores.}\label{fig:boxPlot}
\end{figure*}

Pillai's V for the analysis is 0.126, \(p = 0.004\) (Wilks' \(\Lambda\)
= 0.875, \(p = 0.003\); Hotelling-Lawley's trace = 0.142, \(p = 0.003\);
Roy's largest root = 0.137, \(p = 4.912\times 10^{-4}\)). The effect
size, calculated as \(\omega^2_{mult} = 0.109\)
(bootstrap\textsubscript{R = 10000} median = 0.132, \(BC_\alpha\) 95\%
CI = {[}0.012, 0.201{]}). To further inspect the relationship of the
interpolated activities with our dependent variables, we have conducted
a Roy-Bargmann stepdown analysis, as suggested by
\citeauthor{tabachnick_using_2012} (\citeyear{tabachnick_using_2012}; a
linear discriminant analysis with the same aim is available in the
supplementary materials). The total number of correct answers was a
priori chosen to be the higher priority variable. Therefore, we first
conducted an ANOVA with interpolated activity type as the indepedent
variable and the total number of correct answers as the dependent
variable.

As could be expected, the ANOVA points to an interpolated activity
effect, with \(F(2, 119)\) = 7.541, \(p = 8.254\times 10^{-4}\).
Following the ANOVA, we conducted an ANCOVA, with the total number of
correct answers as the covariate, and the total number of intrusors as
the dependent variable. The results imply a main effect of the total
number of correct answers (\(F(1, 118)\) = 79.674,
\(p = 6.873\times 10^{-15}\)), but after taking into acount the number
of correct answers, there is no evidence for an effect of interpolated
activity on the total number of chosen intrusors (\(F (2, 118)\) =
0.844, \(p = 0.433\). For now, we may claim that we do not have any
evidence to support our second hypothesis that the type of interpolated
activity will have an effect on the number of intrusors.

In order to test our first hypothesis, we have contrasted (i) the
rereading group with the two test groups, and (ii) the two test groups
with each other, taking only the total number of correct answers as the
DV. The first contrast finds no evidence of a difference between the
rereading group and the two test groups (\(t\) = 1.355, \(p = 0.178\),
\(g_s\) = 0.19, 95\% CI = {[}-0.19, 0.57{]}, Cohens's \(U_{3, g_s}\) =
57.6\%, probability of superiority = 55.39\%). However, there is a
difference between the two test groups (\(t\) = 3.62,
\(p = 4.34\times 10^{-4}\), \(g_s\) = 0.66, 95\% CI = {[}0.21, 1.1{]},
Cohens's \(U_{3, g_s}\) = 74.43\%, probability of superiority =
67.88\%). Participants in the content related test group scored higher
on the final test than participants in the general knowledge test
condition. These two findings are not in line with our predictions.

\hypertarget{the-interaction-between-feedback-and-interpolated-activity-type}{%
\subsubsection{The interaction between feedback and interpolated
activity
type}\label{the-interaction-between-feedback-and-interpolated-activity-type}}

The remaining hypotheses deal with the effect of feedback on the total
number of correct answers and the total number of intrusors. Therefore,
these analyses are carried out only on the data from participants in the
general and content related test conditions (\(n\) = 163). Boxplots for
these groups are shown in Figure \ref{box2}. To test these hypotheses,
we first conducted a two-way MANOVA with interpolated activity and
feedback as independent variables, and total number of correct answers
and total number of intrusors as the dependent variables.

\begin{figure*}[h]

{\centering \includegraphics{methods-results_files/figure-latex/boxPlot2-1} 

}

\caption{\label{box2} Boxplots broken down by experimental conditions included in the second MANOVA, and dependent variable, with overlayed raw scores.}\label{fig:boxPlot2}
\end{figure*}

Pillai's V for the interpolated activity effect (calculated with type
III sums of squares) is 0.071, \(p = 0.003\) (Wilks' \(\Lambda\) =
0.929, \(p = 0.003\); Hotelling-Lawley's trace = 0.08, \(p = 0.003\);
Roy's largest root = 0.08, \(p = 0.003\)) confirming the main effect of
interpolated activity type. The effect size \(\omega^2_{mult}\) = 0.065
(bootstrap\textsubscript{R = 10000} median = 0.072, \(BC_\alpha\) 95\%
CI = {[}0.007, 0.139{]}).

On the other hand, we find no evidence for an effect of giving feedback
on the linear combination of our two dependent variables --- Pillai's V
= 0.003, \(p = 0.8\) (Wilks' \(\Lambda\) = 0.997, \(p = 0.8\);
Hotelling-Lawley's trace \(\approx\) 0, \(p = 0.8\); Roy's largest root
\(\approx\) 0, \(p = 0.8\)). The effect size is \(\omega^2_{mult}\) =
-0.003 (bootstrap\textsubscript{R = 10000} median = 0.003\footnote{
The \(BC_\alpha\) 95\% CI for this estimate is \([-0.006, 0.004]\).
\label{bca-ref}}).

Furthermore, we find no evidence for an interaction effect between
activity type and feedback --- Pillai's V = 0.001, \(p = 0.941\) (Wilks'
\(\Lambda\) = 0.999, \(p = 0.941\); Hotelling-Lawley's trace \(\approx\)
0, \(p = 0.941\); Roy's largest root \(\approx\) 0, \(p = 0.941\)). The
effect size \(\omega^2_{mult}\) = -0.005
(bootstrap\textsubscript{R = 10000} median = 0.003\footnote{
The \(BC_\alpha\) 95\% CI = \([-0.006, -0.005]\).
Our guess is that this odd result is due to the fact that most of the density is concentrated
around 0, causing an unreliable estimate. The same could be said for the CI in
footnote \ref{bca-ref}.}). Both the feedback and the interaction
estimates of \(\omega^2_{mult}\) are to be considered to be zero, given
their negative values.

Again, we have conducted a follow-up Roy-Bargmann stepdown analysis. In
the ANOVA model with the total number of correct answers as the
dependent variable and the type of interpolated activity, feedback and
their interaction as predictors, only the type of activity seems to be
relevant (\(F(1, 159) = 11.2, p = 0.001\)). This result also shows that
participants in the content related test condition scored higher on the
final test than the participants in the general knowledge test
condition, which should be no suprise given the results of the first
stepdown analysis. In the second step, we fit an ANCOVA model with the
total number of correct answers as the covariate. In this model, the
type of interpolated activity ceases to be a relevant predictor
(\(F(1, 155) = 0.175, p = 0.676\)). The full models are shown in Table
\ref{rb2-table}.

\begin{table*}[t]

\caption{\label{tab:rb2Table}Full ANOVA and ANCOVA models for the second Roy-Bargmann
                     stepdown analysis.\label{rb2-table}}
\centering
\begin{tabular}{lrrrr}
\toprule
Term & $SS$ & $df$ & $F$ & $p$\\
\midrule
\addlinespace[0.3em]
\multicolumn{5}{l}{\textbf{ANOVA}}\\
\hspace{1em}Activity & 109.393 & 1 & 11.200 & 0.001\\
\hspace{1em}Feedback & 3.904 & 1 & 0.400 & 0.528\\
\hspace{1em}Activity x Feedback & 0.045 & 1 & 0.005 & 0.946\\
\hspace{1em}Residuals & 1553.046 & 159 &  & \\
\addlinespace[0.3em]
\multicolumn{5}{l}{\textbf{ANCOVA}}\\
\hspace{1em}Activity & 0.301 & 1 & 0.175 & 0.676\\
\hspace{1em}Feedback & 0.173 & 1 & 0.100 & 0.752\\
\hspace{1em}Total correct & 63.216 & 1 & 36.760 & 0.000\\
\hspace{1em}Activity x Feedback & 0.813 & 1 & 0.473 & 0.493\\
\hspace{1em}Activity x Total correct & 0.862 & 1 & 0.501 & 0.480\\
\hspace{1em}Feedback x Total correct & 0.130 & 1 & 0.075 & 0.784\\
\hspace{1em}Activity x Feedback x Total correct & 1.229 & 1 & 0.715 & 0.399\\
\hspace{1em}Residuals & 266.551 & 155 &  & \\
\bottomrule
\end{tabular}
\end{table*}

\hypertarget{notes}{%
\subsection{Notes}\label{notes}}

Plots created using \texttt{ggplot2} \citep{wickham_ggplot2:_2016}.
Bootstrap conducted using the \texttt{boot} package
\citep{canty_boot:_2017}. Methods and analyses written using
\texttt{rmarkdown} \citep{allaire_rmarkdown:_2019} and \texttt{knitr}
\citep{xie_knitr:_2019}. The package \texttt{car} \citep{fox_r_2011} was
used to obtain type III sums of squares. \texttt{compute.es}
\citep{re_compute.es:_2013} was used to obtain effect sizes for
contrasts. \texttt{kableExtra} was used to help generate tables
\citep{zhu_kableextra:_2019}. Other utilities used are
\texttt{tidyverse} \citep{wickham_tidyverse:_2017}, \texttt{magrittr}
\citep{bache_magrittr:_2014}, \texttt{here} \citep{muller_here:_2017},
\texttt{conflicted} \citep{wickham_conflicted:_2018}, \texttt{psych}
\citep{revelle_psych:_2018}.

\bibliographystyle{apacite}
\bibliography{../../paper/reference.bib}


\end{document}
