\documentclass[12pt]{article}
\usepackage[croatian]{babel}
\usepackage[utf8]{inputenc}
\usepackage[T1]{fontenc}
\usepackage[euler]{textgreek}
\usepackage[compact]{titlesec}
\usepackage{url}
\usepackage{longtable}
\usepackage{titletoc}
\usepackage{amsmath}
\usepackage[inline]{enumitem}
\usepackage{bookman}
\usepackage[left=2cm,right=2cm,bottom=2cm,top=2cm]{geometry}
\usepackage[onehalfspacing]{setspace}
\usepackage[left=\glqq, right=\grqq, leftsub=\glq, rightsub=\grq]{dirtytalk}

\newcommand{\periodafter}[1]{#1.}
\titleformat{\section}{\sffamily\scshape\Large\bfseries}{}{0pt}{}
\titleformat{\subsection}{\large\itshape}{}{0pt}{}
\titleformat{\paragraph}[runin]{\normalsize\itshape}{}{0pt}{\periodafter}
\titlespacing*{\paragraph}{0pt}{5pt}{5pt}[0pt]

%doc specific
\newcommand{\sub}[1]{\textsubscript{#1}}
\newcommand{\paro}{P\sub{O\sub{2}}}
\newcommand{\tl}{mm Hg}

\begin{document}

\thispagestyle{empty}
\vspace*{\fill}
\begin{center}
\bfseries
\sffamily
\scshape
\Large
Fiziologija letenja, putovanja u svemir i dubinskog ronjenja
\end{center}
\vspace*{\fill}
\clearpage
\setcounter{page}{1}

Kako se ljudi avionima, penjanjem na planine i svemirskim letjelicama uspinju na
sve veće visine, tako postaje sve važnije da se shvate učinci visine i niskih
tlakova plinova na ljudski organizam. Isto tako, ljudi se sve dublje spuštaju u
more, pa je potrebno da se shvate i učinci plinova pod visokim tlakom.

U ovom poglavlju razmotrit ćemo sve te probleme: prvo hipoksiju na velikim
visinama; drugo, ostale fizičke faktore koji djeluju na organizam u velikim
visinama; treće, ogromne sile akceleracije koje se javljaju i u letačkoj i u 
kozmičkoj fiziologiji; najzad, učinke plinova pod visokim tlakom u morskim
dubinama.

\section{Djelovanje niskog tlaka kisika na organizam}

\paragraph{Barometarski tlakovi na različitim visinama} Tablica 1 prikazuje
barometarske tlakove na različitim visinama, te se može vidjeti da je tlak na
razini mora 760 \tl{}, dok je na približno 3000 m samo 523 \tl{}, a na kojih
15000 m iznosi samo 87 \tl{}. To sniženje barometarskog tlaka osnovni je uzrok
svih problema što ih uzrokuje hipoksija na velikim visinama. Naime, kad god se
barometarski tlak snizuje, razmjerno se snizuje i tlak kisika, koji uvijek
iznosi manje od 21\% ukupnog barometarskog tlaka.

\paragraph{Parcijalni tlak kisika na različitim visinama} Tablica 1 pokazuje i
to da parcijalni tlak kisika (\paro{}) u suhom zraku na morskoj razini
iznosi približno 150 \tl{}. Kada u zraku ima mnogo vodene pare, ta se vrijednost
može smanjiti za 10 \tl{}. Na približno 3000 m \paro{} iznosi oko 110 \tl{},
na približno 6000 m kreće se oko 73 \tl{}, a na približno 15000 m iznosi 18
\tl{}.

\subsection{Alveolarni \paro{} na različitim visinama}

Kad se na većim visinama \paro{} u atmosferi snizuje, može se, naravno,
očekivati i pad \paro{} u alveolama. Na manjim visinama alveolarni \paro{} ne
snizuje se baš u tolikoj mjeri kao parcijalni tlak kisika u atmosferi, jer se
smanjenje kisika u atmosferi donekle nadoknađuje povećanom ventilacijom.
Međutim, na većim visinama \paro{} u alveolama se snizuje čak i više nego \paro{} u
atmosferi, i to zbog posebnih razloga koje ćemo protumačiti.

\paragraph{Učinak ugljičnog dioksida i vodene pare na kisik u alveolama}
Čak i na velikim nadmorskim visinama ugljični se dioksid stalno odstranjuje iz
krvi plućnih kapilara u alveole. Isto tako se i voda s respiracijskih površina
isparava u alveolarni prostor. Zbog toga ta dva plina razrjeđuju kisik i dušik
koji se već nalaze u alveolama, te tako snizuju koncentraciju kisika.

Nazočnost ugljičnog dioksida i vodene pare u alveolama postaje napose važna na
velikim visinama, jer ukupan barometarski tlak pada na niske vrijednosti, dok se
tlakovi ugljičnog dioksida i vodene pare ne snizuju razmjerno tome. Dokle god je
temperatura tijela normalna, tlak vodene pare zadržava vrijednost od 47 \tl{},
bez obzira na visinu; ali zbog pojačanog disanja na krajnje velikim visinama
tlak ugljičnog dioksida pada od približno 40 \tl{}, koliko iznosi na morskoj
razini, na približno 24 \tl{}.

Da razmotrimo na koji način tlakovi tih dvaju plinova djeluju na raspoloživi
prostor za kisik. Pretpostavimo li da ukupan barometarski tlak padne na 100 \tl{},
tada tada 47 \tl{} od toga mora otpasti na tlak vodene pare, pa će za sve
ostale plinove preostati samo 53 \tl{}. Od tih 53 \tl{} 24 \tl{} mora otpasti na
ugljični dioksid, te preostaje slobodno svega 29 \tl{}. Kada se kisik u tijelu
ne bi trošio, jedna petina od 29 \tl{}  bio bi kisik, a četiri petine bi otpadale
na dušik, tj. \paro{} bi u alveolama iznosio 6 \tl{}. Međutim, tkiva dotične osobe
postala bi do tog trenutka gotovo posve anoksična, pa bi se zbog toga čak i ovaj
zadnji trag kisika apsorbirao u krv, ne ostavljajući više od 1 \tl{} tlaka u
alveolama. Prema tome, pri barometarskom tlaku od 100 \tl{} čovjek ne bi mogao
preživjeti ako bi udisao zrak. Učinak je, međutim, bitno drukčiji ako čovjek
diše čisti kisik, kao što ćemo to opisati u daljnjem izlaganju.

\paragraph{Alveolarni \paro{} na različitim visinama} Tablica 1 prikazuje i \paro{}
u alveolama na različitim visinama kad čovjek diše zrak i kad diše čisti kisik.
Diše li zrak, alveolarni \paro{} iznosi 104 \tl{} na morskoj razini, na visini od
3000 m pada na 67 \tl{}, a na 15000 m snizuje se na svega 1 \tl{}.

\paragraph{Zasićenje hemoglobina kisikom na različitim visinama} Slika 1
prikazuje zasićenje arterijske krvi kisikom na različitim nadmorskim visinama,
kad čovjek diše zrak i kad diše kisik, a zasićenje u postocima na svakih 3000 m
prikazuje Tablica 1. Do visine od približno 3000 m, čak i kad se diše zrak,
zasićenje arterijske krvi kisikom održava se na vrijednosti od najmanje 90\%.
Međutim, kao što prikazuje lijeva krivulja na slici, iznad 3000 m zasićenje
arterijske krvi kisikom sve više pada, tako da na visini od 6000 m iznosi svega
70\%, a na većim visinama još manje.

\subsection{Utjecaj udisanja čistog kisika na alveolarni \paro{} na različitim
    visinama}

Osvrćući se još jednom na tablicu 1, zapažamo da u čovjeka koji na visini od
približno 9000 m udiše zrak alveolarni \paro{} iznosi samo 21 \tl{}, iako
barometarski tlak iznosi 226 \tl{}. Ta je razlika u prvom redu uzrokovana
činjenicom da znatan dio alveolarnog zraka čini dušik. Ako čovjek umjesto zraka
udiše čisti kisik, glavnina alveolarnog prostora, koju je prije zapremao dušik,
ispunja se kisikom. Teorijski, na približno 9000 m pilot bi mogao u alveolama
imati \paro{} od 139 \tl{}, a ne 21 \tl{}, koliko ima kad udiše zrak.

Druga krivulja na Slici 1 prikazuje zasićenje arterijske krvi kisikom na
različitim visinama kad čovjek udiše čist kisik. Može se zapaziti da zasićenje
ostaje iznad 90\% dok pilot ne dosegne približno 12000 m, a da se zatim naglo
smanjuje do oko 50\% na visini od 14000 m. To je otprilike granična vrijednost
koju pilot kroz duže vrijeme može podnijeti, pa se zbog toga ta visina ponekad
naziva \say{stropom}.

\subsection{Učinci hipoksije}

Plućna ventilacija obično se ne povećava znatnije dokle god se ne postigne
visina od približno 2500 m. Na toj visini zasićenje arterijske krvi kisikom pada
na približno 93\%, a to je razina na kojoj kemoreceptori izrazito reagiraju.
Iznad 2500 m mehanizam stimulacije preko kemoreceptora sve više povećava
ventilaciju, dokle god se ne dosegne visina od približno 5000 do 6000 m. Na toj
visini ventilacija postiže maksimalnu vrijednost od približno 65\% iznad
normalne. Daljnje povećanje visine ne pojačava više aktivnost kemoreceptora.

Ostali učinci hipoksije koji se počinju javljati na visini od približno 3500 m
jesu pospanost, tromost, mentalni umor, ponekad glavobolja i mučnina, a katkad i
euforija. Većina se tih simptoma pojačava na još većim visinama. Glavobolja je
često osobito jako izražena, a cerebralni simptomi ponekad se pogoršavaju sve do
stadija kad se jave trzaji i konvulzije, te u neaklimatizirane osobe završavaju
komom na visini od oko 7000 m.

Jedan je od najvažnijih učinaka hipoksije smanjenje mentalnih sposobnosti, zbog
čega je otežano rasuđivanje i pamćenje, te izvođenje finih motoričkih pokreta.
Obično te sposobnosti ostaju potpuno normalne do 2500 m, a za kratko vrijeme
mogu biti potpuno normalne do visine od 4500 m. Međutim, ako je avijatičar
izložen hipoksiji duže vrijeme, njegova se mentalna sposobnost, mjerena vremenom
reakcije, rukopisom i drugim psihološkim testovima, može smanjiti na 80\%
normalne vrijednosti, čak i na visini od 3300 m. Ako letač ostane na 4500 m jedan
sat, njegova će se mentalna sposobnost obično smanjiti na približno 50\%
normalne vrijednosti, a nakon 18 sati provedenih na toj visini na približno 20\%
normalne vrijednosti.

\subsection{Aklimatizacija na nizak \paro{}}

Ako čovjek ostane na velikim visinama nekoliko dana, tjedana ili godina,
postepeno će se aklimatizirati (prilagoditi) na nizak \paro{}. Tako će štetni
utjecati niskog \paro{} na organizam postajati sve slabiji, pa će čovjek moći
raditi teže poslove ili se uspinjati na još veće visine. Navest ćemo nekoliko
načina na koje se postiže aklimatizacija:
\begin{enumerate*}[label=\arabic*)]
\item povećanje plućne ventilacije, 
\item povećanje količine hemoglobina u krvi i
\item povećanje stupnja vaskularizacije tkiva.
\end{enumerate*}

\paragraph{Povećana plućna ventilacija} Prilikom iznenadnog izlaganja niskom
\paro{}, podraživanje kemoreceptora hipoksijom povećava alveolarnu ventilaciju za
maksimalno 65\%. To je trenutna kompenzacija učinka velike visine, koja sama
omogućuje da se čovjek popne nekoliko tisuća metara više nego što bi mogao da se
ventilacija nije povećala. Ako čovjek na veoma velikoj visini ostane nekoliko
dana, ventilacija se postepeno povećava na vrijednost koja je pet do sedam puta
veća od normalne. Sada ćemo opisati osnovni uzrok ovom postepenom povećanju
ventilacije.

Nagli porast plućne ventilacije za 65\%, kad se čovjek penje na veliku visinu,
odstranjuje velike količine ugljičnog dioksida. Zato se u tjelesnim tekućinama
smanjuje P\textsubscript{CO\textsubscript{2}} i povećava pH. Obje te promjene
\emph{koče} centar za disanje i na taj se način \emph{protive podraživanju
    hipoksijom}. Međutim, tokom narednih tri do pet dana ta inhibicija nestaje,
omogućujući respiracijskom centru da sada punom snagom odgovara na signale iz
kemoreceptora podraženih hipoksijom, pa ventilacija postane pet do sedam puta
veća od normalne.

\paragraph{Porast količine hemoglobina u toku aklimatizacije} Već smo u  5.
poglavlju iznijeli da je hipoksija glavni podražaj za pojačanu produkciju
eritrocita. Obično se pri potpunoj aklimatizaciji na nizak tlak kisika
hematokritska vrijednost povećava od normalnih 40--45 na prosječno 60--65, uz
prosječni porast koncentracije hemoglobina od normalnih 15 g\% na približno 22
g\%.

Osim toga, i volumen se krvi povećava, često za 20 do 30\%, što izaziva ukupan
porast količine hemoglobina u cirkulaciji za 50 do 90\%.

Međutim, to povećanje količine hemoglobina i volumena krvi ide sporo, tj.
praktički se uopće ne zapaža u prva dva do tri tjedna, razvije se do polovice
otprilike za mjesec dana ili nešto više, a posve se razvije tek nakon nekoliko
mjeseci.

\paragraph{Pojačana vaskularizacija} Histološka istraživanja na životinjama koje
su mjesecima ili godinama bile izložene niskim razinama kisika pokazuju da se
\emph{jako pojačava vaskularizacija} (povećan broj i veličina kapilara)
hipoksičnih tkiva. To pomaže da se protumači što se događa s onih 20--30\%
porasta volumena krvi. Krv, dakle, dolazi u mnogo tješnji kontakt s tkivnim
stanicama nego inače.

\subsection{Prirodna aklimatizacija domorodaca na visokim planinama}

Mnogi stanovnici Anda i Himalaja žive na visinama iznad 4000 m, a jedna skupina
u Peruanskim Andama živi zapravo na visini od približno 5500 m i radi u rudniku
na visini od 6000 m. Mnogi od ovih ljudi rođeni su na tim visinama i žive tu
čitava života. U pogledu svih prije navedenih aspekata aklimatizacije domoroci
su u prednosti i pred najbolje aklimatiziranim stanovnicima nizina, čak i onima
koji deset godina ili duže žive na velikoj razini. Proces aklimatizacije
domorodaca počinje u ranom djetinjstvu. U njih je prsni koš znatno povećan, dok
je veličina tijela nešto smanjena, pa je zbog toga omjer ventilacijskog
kapaciteta prema masi tijela velik. Osim toga, u usporedbi sa stanovnicima
nizina, u njih je veće i srce, naročito desno, koje u plućnoj arteriji osigurava
visok tlak potreban za tjeranje krvi kroz jako proširen sistem plućnih kapilara.

Predodžbu o važnosti aklimatizacije mogu nam dati podaci koje ćemo sada navesti.
Radni kapacitet, izražen u postocima maksimalno mogućeg za normalnu osobu na
morskoj razini, iznosi na visini od oko 5000 m približno:

\vspace*{1em}
\begin{tabular}{@{\hspace{-0.1em}}lr}
    u neaklimatiziranih osoba & 50\%\\
    nakon dvomjesečne aklimatizacije & 68\%\\
    \parbox{8cm}{\vspace{3pt}u gorštaka koji žive na 4000 m, \\\hspace*{2em} a
        rade na 5000 m}& \raisebox{7pt}{87\%}
\end{tabular}
\vspace*{1em}

Prirodno aklimatizirani gorštaci mogu, dakle, čak i na tim visinama postići
dnevni učinak gotovo jednak onome u normalnih osoba na morskoj razini, dok ljudi
iz nizina taj rezultat ne mogu gotovo nikad postići, čak ni kad su dobro
aklimatizirani.

\section{Fiziologija dubinskog ronjenja i rada pod visokim tlakom}

Kad se čovjek spušta u more, tlak se oko njega jako povisuje. Da bi se spriječio
kolaps pluća, zrak se čovjeku mora dostavljati također pod visokim tlakom, pa se
krv u plućima izlaže krajnje visokim alveolarnim tlakovima plinova. Iznad nekih
granica ti visoki tlakovi mogu izazvati velike promjene u funkcijama organizma,
što opravdava potrebu ovog razmatranja.

Visokim atmosferskim tlakovima izlažu se i radnici u kesonima, koji kopajući
tunele ispod rijeka ili na drugim mjestima često moraju raditi pod visokim
tlakom koji sprečava da se tunel ne uruši. I u ovom se slučaju javljaju isti
problemi zbog pretjerano visokih tlakova u alveolama.

Prije no što protumačimo učinke visokih tlakova plinova na organizam, treba da
ponovimo neke fizičke principe o promjenama tlaka i volumena u različitim
morskim dubinama.

\paragraph{Odnos dubine prema tlaku} Stupac morske vode od 10 m vrši na dno
jednak pritisak kao sva atmosfera iznad Zemlje. Zato će čovjek deset metara
ispod morske površine biti izložen tlaku od 2 atmosfere, jednoj atmosferi zbog
tlaka zraka iznad vode i drugoj atmosferi zbog težine same vode. Na dubini od 20
m tlak će iznositi 3 atmosfere itd., u skladu s tablicom 30-2.

\paragraph{Djelovanje dubine na volumen plinova} Drugi važan učinak dubine jest
kompresija plinova na sve manji volumen. Na morskoj dubini od 10 m, gdje tlak
iznosi 2 atmosfere, volumen koji na razini mora iznosi 1 litru stlačen je na
svega pola litre. Na dubini od 30 m, gdje vlada tlak od 4 atmosfere, volumen je
komprimiran na četvrtinu litre, a pod tlakom od 8 atmosfera (dubina od 70 m) na
jednu osminu litre. Taj je učinak izvanredno važan prilikom ronjenja, jer se u
tijelu ronioca šupljine ispunjene zrakom, uključivši pluća, mogu ponekad toliko
smanjiti da nastaju teška oštećenja.

\subsection{Djelovanje visokih parcijalnih tlakova plinova na organizam}

Ronilac koji udiše zrak izložen je normalno ovim plinovima: dušiku, kisiku i
ugljičnom dioksidu.

\paragraph{Narkotičko djelovanje dušika pod visokim tlakom} Približno četiri
petine zraka čini dušik. Pod tlakom koji vlada na morskoj razini dušik nema
poznatih učinaka na tjelesne funkcije, ali pod visokim tlakovima može uzrokovati
različite stupnjeve narkoze. Ako ronilac koji udiše komprimirani zrak ostane u
moru nekoliko sati, prvi će se znaci blage narkoze javiti na dubini od približno
40 do 45 m. U toj dubini ronilac počinje pokazivati veselost i gubiti mnogo od
svoje opreznosti. U dubini od 45 do 60 m postaje pospan. Između 60 i 75 m snaga
ronioca jako se smanjuje i on postaje tako nespretan da ne može obavljati rad
koji se od njega zahtijeva. Ispod 90 m (tlak od 10 atmosfera) ronilac obično
postaje zbog narkoze dušikom gotovo posve nesposoban. Međutim, da bi se u tijelu
otopilo toliko dušika da bi nastale ove pojave \emph{visok tlak obično treba da
    djeluje jedan sat ili duže}.

Narkoza dušikom pokazuje simptome veoma slične onima što se javljaju prilikom
otrovanja alkoholom, pa se zbog toga često naziva \say{dubinsko pijanstvo} ili
\say{opojenost dubinom}.

Smatra se da je mehanizam narkotičkog učinka jedna učincima svih drugih plinova
anestetika. To znači da se dušik lako otapa u mastima organizma, pa se,
vjerojatno, kao i većina ostalih plinovitih anestetika otapa u membranama ili
drugim lipidnim strukturama neurona. Svojim \emph{fizičkim} učinkom dušik
mijenja prijenos električnih naboja, pa tako smanjuje podražljivost neurona.

\paragraph{Toksičnost kisika pod visokim tlakom} Udisanje kisika pod vrlo
visokim parcijalnim tlakovima može pogubno djelovati na centralni nervni sistem,
izazivajući ponekad epileptičke grčeve nakon kojih nastupa koma. Izlaganje tlaku
kisika od 3 atmosfere (\paro = 2280 \tl{}) u većine ljudi izaziva konvulzije i
komu za približno jedan sat. Te konvulzije često nastaju bez ikakvog prethodnog
znaka, pa za ronioca u moru mogu, dakako, biti smrtonosne.

Uzrok ili uzroci toksičnosti kisika nisu još poznati, ali ćemo navesti jednu
pretpostavku. Nakon teškog otrovanja kisikom koncentracije nekih oksidacijskih
enzima u tkivima znatno su snižene. Zbog toga se smatra da višak kisika
inaktivira pojedine oksidacijske enzime i izaziva otrovanje na taj način što
smanjuje sposobnost tkiva da stvara visokoenergetske fosfatne vezove.

\paragraph{Toksičnost ugljičnog dioksida u velikim dubinama} Ako je ronilačka
oprema dobro konstruirana i ispravno radi, za ronioca nema opasnosti od
otrovanja ugljičnim dioksidom, jer sama dubina ne povećava P\sub{CO\sub{2}} u
alveolama. Razlog tome je taj što se ugljični dioksid stvara u organizmu, pa
dokle god ronilac udiše normalan respiracijski volumen izdisat će i dalje
onoliko ugljičnog dioksida koliko ga stvara, te će parcijalni tlak tog plina u
alveolama održati na normalnoj vrijednosti.

Ipak u nekim vrstama ronilačke opreme, kao npr. u ronilačkom šljemu i pojedinim
tipovima aparata za povratno disanje, ugljični se dioksid često može
nagomilavati u zraku mrtvog prostora samog aparata, odakle ga ronilac ponovno
udiše. Ronilac podnosi to nagomilavanje dok tlak ugljičnog dioksida
(P\sub{CO\sub2}}) u alveolama ne dosegne oko 80 \tl{}, što je dvostruko veća
vrijednost od normalne. Tada se minutni volumen disanja poveća do maksimalne
vrijednosti, koja je 6 do 10 puta veća od normalne, pa se time kompenzira
povišenje koncentracije ugljičnog dioksida. Međutim, iznad razine od 80 \tl{}
stanje postaje nepodnošljivo, te umjesto podraživanja konačno počinje depresija
centra za disanje, što znači da disanje počinje zatajivati umjesto da
kompenzira. Kao posljedica, u ronioca se razvija teška respiracijska acidoza,
različiti stupnjevi letargije, narkoza i, najzad, anestezija.

\end{document}
