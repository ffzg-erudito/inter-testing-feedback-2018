\documentclass[12pt]{article}
\usepackage[margin=1in]{geometry}
\usepackage[doublespacing]{setspace}
\usepackage[english]{babel}
\usepackage[euler]{textgreek}
\usepackage[htt]{hyphenat}
\usepackage[inline]{enumitem}
\usepackage[natbibapa, sectionbib, tocbib]{apacite}
\usepackage[normalem]{ulem}
\usepackage[page, toc]{appendix}
\usepackage[singlelinecheck = off]{caption}
\usepackage[utf8]{inputenc}
\usepackage{afterpage}
\usepackage{amsmath}
\usepackage{amssymb}
\usepackage{array}
\usepackage{booktabs}
\usepackage{colortbl}
\usepackage{environ}
\usepackage{etoolbox}
\usepackage{fancyhdr}
\usepackage{float}
\usepackage{gensymb}
\usepackage{graphicx}
\usepackage{ifxetex,ifluatex}
\usepackage{lmodern}
\usepackage{longtable}
\usepackage{lscape}
\usepackage{makecell}
\usepackage{microtype}
\usepackage{multirow}
\usepackage{pdflscape}
\usepackage{subfiles}
\usepackage{tabu}
\usepackage{textcomp}
\usepackage{threeparttablex}
\usepackage{threeparttable}
\usepackage{titletoc}
\usepackage{url}
\usepackage{wrapfig}
\usepackage{xcolor}
\usepackage{grffile}
%\usepackage{parskip}
%%% Change title format to be more compact
\usepackage{titling}
\usepackage{hyperref}

\setlength{\parskip}{0.38em}
\renewcommand{\baselinestretch}{1.5}
\setlength{\parindent}{2em}

\fancyhf{}
\fancyhead[RH]{\thepage}
\renewcommand{\headrulewidth}{0pt}
\pagestyle{fancy}

\hypersetup{colorlinks = true, linkcolor = blue, urlcolor = black, citecolor = blue}

\DeclareCaptionFormat{apa}{#1#2\\[1em]#3}
\captionsetup*[table]{labelsep = none, textfont = it, format = apa, width = .8\textwidth}
\captionsetup*[figure]{labelsep = period, labelfont = it, position = below}

%%%%%%%%%%%%%%%%%%%%%%% remove url's and date if there's a doi

% \newtoggle{bibdoi}
% \newtoggle{biburl}
% \makeatletter

% \undef{\APACrefURL}
% \undef{\endAPACrefURL}
% \undef{\APACrefDOI}
% \undef{\endAPACrefDOI}
% 
% \long\def\collect@url#1{\global\def\bib@url{#1}}
% \long\def\collect@doi#1{\global\def\bib@doi{#1}}
% \newenvironment{APACrefURL}{\global\toggletrue{biburl}\Collect@Body\collect@url}{\unskip\unskip}
% \newenvironment{APACrefDOI}{\global\toggletrue{bibdoi}\Collect@Body\collect@doi}{}
% 
% \AtBeginEnvironment{thebibliography}{
% 	\pretocmd{\PrintBackRefs}{%
% 		\iftoggle{bibdoi}
% 		{\iftoggle{biburl}{\unskip\unskip doi:\bib@doi}{}}
% 		{\iftoggle{biburl}{Retrieved from\bib@url}{}}
% 		\togglefalse{bibdoi}\togglefalse{biburl}%
% 	}{}{}
% }
% \makeatother
%%%%%%%%%%%%%%%%%%%%%%%


\renewcommand{\APACrefYearMonthDay}[3]{\APACrefYear{#1}}

% \makeatletter
% \renewcommand{\maketitle}{\bgroup\setlength{\parindent}{0pt}
% 	\begin{flushleft}
% 		\textbf{\@title}
% 		
% 		\@author
% 	\end{flushleft}\egroup
% }
% \makeatother

\def\biblio{\bibliographystyle{apacite}\bibliography{../reference.bib}}

% rmarkdown preamble
\ifnum 0\ifxetex 1\fi\ifluatex 1\fi=0 % if pdftex
  \usepackage[T1]{fontenc}
  \usepackage[utf8]{inputenc}
\else % if luatex or xelatex
  \ifxetex
    \usepackage{mathspec}
  \else
    \usepackage{fontspec}
  \fi
  \defaultfontfeatures{Ligatures=TeX,Scale=MatchLowercase}
\fi
% use upquote if available, for straight quotes in verbatim environments
\IfFileExists{upquote.sty}{\usepackage{upquote}}{}
% use microtype if available
\UseMicrotypeSet[protrusion]{basicmath} % disable protrusion for tt fonts

\makeatletter
\def\maxwidth{\ifdim\Gin@nat@width>\linewidth\linewidth\else\Gin@nat@width\fi}
\def\maxheight{\ifdim\Gin@nat@height>\textheight\textheight\else\Gin@nat@height\fi}
\makeatother
% Scale images if necessary, so that they will not overflow the page
% margins by default, and it is still possible to overwrite the defaults
% using explicit options in \includegraphics[width, height, ...]{}
\setkeys{Gin}{width=\maxwidth,height=\maxheight,keepaspectratio}
\IfFileExists{parskip.sty}{%
}{% else
\setlength{\parindent}{0pt}
\setlength{\parskip}{6pt plus 2pt minus 1pt}
}
\setlength{\emergencystretch}{3em}  % prevent overfull lines
\providecommand{\tightlist}{%
  \setlength{\itemsep}{0pt}\setlength{\parskip}{0pt}}
\setcounter{secnumdepth}{0}
% Redefines (sub)paragraphs to behave more like sections
\ifx\paragraph\undefined\else
\let\oldparagraph\paragraph
\renewcommand{\paragraph}[1]{\oldparagraph{#1}\mbox{}}
\fi
\ifx\subparagraph\undefined\else
\let\oldsubparagraph\subparagraph
\renewcommand{\subparagraph}[1]{\oldsubparagraph{#1}\mbox{}}
\fi

%%% Use protect on footnotes to avoid problems with footnotes in titles
\let\rmarkdownfootnote\footnote%
\def\footnote{\protect\rmarkdownfootnote}

% Create subtitle command for use in maketitle
% \providecommand{\subtitle}[1]{
%   \posttitle{
%     \begin{center}\large#1\end{center}
%     }
% }
% 
% \setlength{\droptitle}{-2em}
% 
%  \title{}
%    \pretitle{\vspace{\droptitle}}
%  \posttitle{}
%    \author{}
%    \preauthor{}\postauthor{}
%    \date{}
%    \predate{}\postdate{}

\begin{document}

\begin{titlepage}
    \begin{center}

        \vspace*{\fill}

        \Large
        \textbf{The role of retrieval type and feedback in test-potentiated new learning }
        \normalsize

        \bigskip
        Matej Pavlić, Denis Vlašiček \& Dragutin Ivanec

        University of Zagreb, Faculty of Humanities and Social Sciences,
        Department of Psychology

        \bigskip

        \raggedright
        Corresponding author:

        Denis Vlašiček

        Ivana Lučića 3, 10000 Zagreb, Croatia

        dvlasice@ffzg.hr

        \vspace*{\fill}

    \end{center}

\end{titlepage}

\def\biblio{}

\section{Abstract}

This study explored the effects of episodic and semantic retrieval, and
feedback presentation, on learning of new complex material. Participants
read a text divided into three parts, between which they engaged in
(i) content-related (episodic retrieval) or (ii) general-knowledge testing
(semantic retrieval), or (iii) reread the previous part. Participants in the 
two retrieval conditions were randomly assigned to either receive or not to receive
feedback on their interpolated test achievement. Learning was measured through 
multiple choice questions whose distractors were designed specifically to enable
capturing proactive interference. Planned analyses showed that participants in 
the episodic retrieval condition scored higher on the final test than participants 
in the other two groups. We found no evidence for an effect of feedback, nor 
for an interaction effect of interpolated activity type and feedback. Furthermore,
no effect of either independent variable on the number of proactive intrusions was 
found, although the existence of an effect cannot be rejected, which is reflected 
in exploratory Bayesian analyses. Results are interpreted in terms of integration 
and metacognitive frameworks that have previously been suggested as explanations
of the effect.

\bigskip

\noindent Keywords: test-potentiated new learning, interpolated activity, feedback,
semantic retrieval, episodic retrieval

\clearpage

\hypertarget{introduction}{%
\section{Introduction}}

\subfile{sections/intro.tex}

\hypertarget{methods}{%
\section{Methods}}

\subfile{sections/methods.tex}

\hypertarget{results}{%
\section{Results}}

\subfile{sections/analyses-main.tex}

\hypertarget{discussion}{%
\section{Discussion}}

\subfile{sections/discussion.tex}

\section{Author contributions}

All three authors participated in the experimental design and design of
the materials.
MP programmed the application used for data collection.
MP and DV collected the data and analyzed it.
MP and DV wrote the manuscript, DI served as editor.

\section{Acknowledgements}

We would like to thank every one of our participants for making this study
possible. Also, we would like to thank Marijana Glavica, our librarian,
who has been tremendously helpful around data archiving and preparations
for preprint publication.

\section{Notes}

Analyses conducted using the \textit{R} language
\citep{rcoreteamLanguageEnvironmentStatistical2019}.
Bootstrap conducted using the \textit{boot} package
\citep{cantyBootBootstrapSPlus2017}. Methods and analyses written using
\textit{rmarkdown} \citep{allaireRmarkdownDynamicDocuments2019} and
\textit{knitr} \citep{xieKnitrGeneralPurposePackage2019}. The package
\textit{car} \citep{foxCompanionAppliedRegression2011} was used to
obtain type III sums of squares. \textit{compute.es}
\citep{reComputeEsCompute2013} was used to obtain effect sizes for
contrasts. \textit{kableExtra} was used to help generate tables
\citep{zhuKableExtraConstructComplex2019}. Other utilities used are
\textit{tidyverse} \citep{wickhamTidyverseEasilyInstall2017},
\textit{magrittr} \citep{bacheMagrittrForwardPipeOperator2014},
\textit{here} \citep{mullerHereSimplerWay2017}, \textit{conflicted}
\citep{wickhamConflictedAlternativeConflict2018}, \textit{psych}
\citep{revellePsychProceduresPsychological2018}.

\clearpage

\section{Funding information}

This study was conducted under the project \textit{E-rudito: An
advanced online educational system for smart specialization and jobs of the
future} (KK.01.2.1.01.0009.), which is funded from the European Regional
Development Fund, Operational programme competitiveness and cohesion 2014--2020.

\section{Open Practices Statement}

The analysis plan was preregistered on GitHub (\texttt{analysis-plan.md};
first commit of analysis plan: \texttt{b101f42}; final relevant commit:
\texttt{16afea3}), as were the hypotheses (\texttt{design.md}; first commit of hypotheses:
\texttt{b101f42}; final commit: \texttt{dd0f863}).  The repository also serves all
project materials, data and analyses scripts, together with the whole project
history. It can be found at
\texttt{https://github.com/ffzg-erudito/inter-testing-feedback-2018}.
Materials are also available through \texttt{https://osf.io/gk9a3/}. The data
is also hosted on
\url{https://dataverse.ffzg.unizg.hr/dataset.xhtml?persistentId=doi:10.23669/JVNVNR}.

\phantomsection
\addcontentsline{toc}{section}{\refname}
\bibliographystyle{apacite}
\bibliography{reference}

\end{document}
