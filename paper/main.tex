\documentclass[12pt]{article}
\usepackage[margin=1in]{geometry}
\usepackage[doublespacing]{setspace}
\usepackage[english]{babel}
\usepackage[euler]{textgreek}
\usepackage[htt]{hyphenat}
\usepackage[inline]{enumitem}
\usepackage[natbibapa, sectionbib, tocbib]{apacite}
\usepackage[normalem]{ulem}
\usepackage[page, toc]{appendix}
\usepackage[singlelinecheck = off]{caption}
\usepackage[utf8]{inputenc}
\usepackage{afterpage}
\usepackage{amsmath}
\usepackage{amssymb}
\usepackage{array}
\usepackage{booktabs}
\usepackage{colortbl}
\usepackage{environ}
\usepackage{etoolbox}
\usepackage{fancyhdr}
\usepackage{float}
\usepackage{gensymb}
\usepackage{graphicx}
\usepackage{ifxetex,ifluatex}
\usepackage{lmodern}
\usepackage{longtable}
\usepackage{lscape}
\usepackage{makecell}
\usepackage{microtype}
\usepackage{multirow}
\usepackage{pdflscape}
\usepackage{subfiles}
\usepackage{tabu}
\usepackage{textcomp}
\usepackage{threeparttablex}
\usepackage{threeparttable}
\usepackage{titletoc}
\usepackage{titlesec}
\usepackage{url}
\usepackage{wrapfig}
\usepackage{xcolor}
\usepackage{grffile}
%\usepackage{parskip}
%%% Change title format to be more compact
\usepackage{titling}
\usepackage{hyperref}

\setlength{\parskip}{0.38em}
\renewcommand{\baselinestretch}{1.5}
\setlength{\parindent}{2em}

\fancyhf{}
\fancyhead[RH]{\thepage}
\renewcommand{\headrulewidth}{0pt}
\pagestyle{fancy}

\hypersetup{colorlinks = true, linkcolor = blue, urlcolor = black, citecolor = blue}

% section setup
\titleformat{\subsection}{\normalsize\rmfamily\bfseries}{}{0pt}{}

\newcommand\specialsection{%
    \titleformat{\section}{\centering\normalsize\rmfamily}{}{0pt}{}
}

\newcommand\regularsection{%
    \titleformat{\section}{\centering\normalsize\rmfamily\bfseries}{}{0pt}{}
}

\DeclareCaptionFormat{apa}{#1#2\\[1em]#3}
\captionsetup*[table]{labelsep = none, textfont = it, format = apa, width = .8\textwidth}
\captionsetup*[figure]{labelsep = period, labelfont = it, position = below}

\renewcommand{\APACrefYearMonthDay}[3]{\APACrefYear{#1}}

% \makeatletter
% \renewcommand{\maketitle}{\bgroup\setlength{\parindent}{0pt}
% 	\begin{flushleft}
% 		\textbf{\@title}
% 		
% 		\@author
% 	\end{flushleft}\egroup
% }
% \makeatother

\def\biblio{\bibliographystyle{apacite}\bibliography{../reference.bib}}

% rmarkdown preamble
\ifnum 0\ifxetex 1\fi\ifluatex 1\fi=0 % if pdftex
  \usepackage[T1]{fontenc}
  \usepackage[utf8]{inputenc}
\else % if luatex or xelatex
  \ifxetex
    \usepackage{mathspec}
  \else
    \usepackage{fontspec}
  \fi
  \defaultfontfeatures{Ligatures=TeX,Scale=MatchLowercase}
\fi
% use upquote if available, for straight quotes in verbatim environments
\IfFileExists{upquote.sty}{\usepackage{upquote}}{}
% use microtype if available
\UseMicrotypeSet[protrusion]{basicmath} % disable protrusion for tt fonts

\makeatletter
\def\maxwidth{\ifdim\Gin@nat@width>\linewidth\linewidth\else\Gin@nat@width\fi}
\def\maxheight{\ifdim\Gin@nat@height>\textheight\textheight\else\Gin@nat@height\fi}
\makeatother
% Scale images if necessary, so that they will not overflow the page
% margins by default, and it is still possible to overwrite the defaults
% using explicit options in \includegraphics[width, height, ...]{}
\setkeys{Gin}{width=\maxwidth,height=\maxheight,keepaspectratio}
\IfFileExists{parskip.sty}{%
}{% else
\setlength{\parindent}{0pt}
\setlength{\parskip}{6pt plus 2pt minus 1pt}
}
\setlength{\emergencystretch}{3em}  % prevent overfull lines
\providecommand{\tightlist}{%
  \setlength{\itemsep}{0pt}\setlength{\parskip}{0pt}}
\setcounter{secnumdepth}{0}
% Redefines (sub)paragraphs to behave more like sections
\ifx\paragraph\undefined\else
\let\oldparagraph\paragraph
\renewcommand{\paragraph}[1]{\oldparagraph{#1}\mbox{}}
\fi
\ifx\subparagraph\undefined\else
\let\oldsubparagraph\subparagraph
\renewcommand{\subparagraph}[1]{\oldsubparagraph{#1}\mbox{}}
\fi

%%% Use protect on footnotes to avoid problems with footnotes in titles
\let\rmarkdownfootnote\footnote%
\def\footnote{\protect\rmarkdownfootnote}

\begin{document}

\begin{titlepage}
    \begin{center}

        \vspace*{\fill}

        Title: The role of retrieval type and feedback in 
        test-potentiated new learning
        
        Running title: Retrieval type and feedback in TPNL

        \bigskip
        Matej Pavlić\textsuperscript{a}, Denis Vlašiček\textsuperscript{a} 
        \& Dragutin Ivanec\textsuperscript{a}

        a: Faculty of Humanities and Social Sciences, Department of 
        Psychology, University of Zagreb, Croatia

        \bigskip
		\bigskip
		\bigskip

	    \end{center}
		
		{\centering
			Conflict of Interest statement\par}
		We hereby declare that there were no conflicts of 
		interest with respect to the authorship or the 
		publication of this article
		
		\bigskip
		
		{\centering
		Data availability statement\par}

		The analysis plan was preregistered on GitHub 
		(\texttt{analysis-plan.md};
		first commit of analysis plan: \texttt{b101f42}; 
		final relevant commit:
		\texttt{16afea3}), as were the hypotheses 
		(\texttt{design.md}; first commit of hypotheses:
		\texttt{b101f42}; final commit: \texttt{dd0f863}).  
		The repository also serves all
		project materials, data and analyses scripts, 
		together with the whole project
		history. It can be found at
		\texttt{https://github.com/ffzg-erudito/inter-testing-feedback-2018}.
		Materials are also available through 
		\texttt{https://osf.io/gk9a3/}. The data
		is also hosted on
		\url{https://dataverse.ffzg.unizg.hr/dataset.xhtml?persistentId=doi:10.23669/JVNVNR}.

		\bigskip
		
		{\centering
			Acknowledgements\par}
		
		We would like to thank every one of our 
		participants for making this study
		possible. Also, we would like to thank Marijana 
		Glavica, our librarian,
		who has been tremendously helpful around data 
		archiving and preparations
		for preprint publication.	
		
                This study was conducted under the project \textit{E-rudito: An
                advanced online educational system for smart specialization and
                jobs of the future} (KK.01.2.1.01.0009.), which is funded from the
                European Regional Development Fund, Operational programme
                competitiveness and cohesion 2014--2020.
		
        \vspace*{\fill}

 

\end{titlepage}

\def\biblio{}

\specialsection


{\centering
	The role of retrieval type and feedback in 
	test-potentiated new learning\par}


\section{Abstract}

This study explored the effects of retrieval and feedback 
on test-potentiated new learning. Participants read a text 
divided into three parts, between which they engaged in 
either episodic retrieval, semantic retrieval, or 
rereading. Participants in the retrieval conditions were 
randomly assigned to either receive or not to receive 
feedback on their achievement. We administered multiple 
choice questions whose distractors were designed 
specifically to facilitate proactive interference. Planned 
analyses showed that participants in the episodic retrieval 
condition scored higher on the final test than participants 
in the other two groups. Feedback was found to have no 
bearing on new learning --- neither on its own, nor via 
interaction with the interpolated activity type. No effect 
regarding the number of proactive intrusions was found, 
although exploratory Bayesian analyses preclude rejecting 
an effect. Results are interpreted in terms of 
metacognitive theories that have previously been suggested 
as an explanation of the effect.

\bigskip

\noindent Keywords: test-potentiated new learning, interpolated activity, 
feedback, semantic retrieval, episodic retrieval

\newpage

\hypertarget{introduction}{%
}

\subfile{sections/intro.tex}

\regularsection

\hypertarget{methods}{%
\section{Methods}}

\subfile{sections/methods.tex}

\hypertarget{results}{%
\section{Results}}

\subfile{sections/analyses-main.tex}

\hypertarget{discussion}{%
\section{Discussion}}

\subfile{sections/discussion.tex}

\specialsection

\section{Notes}

Analyses conducted using the \textit{R} language
\citep{rcoreteamLanguageEnvironmentStatistical2019}.
Bootstrap conducted using the \textit{boot} package
\citep{cantyBootBootstrapSPlus2017}. Methods and analyses written using
\textit{rmarkdown} \citep{allaireRmarkdownDynamicDocuments2019} and
\textit{knitr} \citep{xieKnitrGeneralPurposePackage2019}. The package
\textit{car} \citep{foxCompanionAppliedRegression2011} was used to
obtain type III sums of squares. \textit{compute.es}
\citep{reComputeEsCompute2013} was used to obtain effect sizes for
contrasts. \textit{kableExtra} was used to help generate tables
\citep{zhuKableExtraConstructComplex2019}. Other utilities used are
\textit{tidyverse} \citep{wickhamTidyverseEasilyInstall2017},
\textit{magrittr} \citep{bacheMagrittrForwardPipeOperator2014},
\textit{here} \citep{mullerHereSimplerWay2017}, \textit{conflicted}
\citep{wickhamConflictedAlternativeConflict2018}, \textit{psych}
\citep{revellePsychProceduresPsychological2018}.

\section{Open Practices Statement}

The analysis plan was preregistered on GitHub (\texttt{analysis-plan.md};
first commit of analysis plan: \texttt{b101f42}; final relevant commit:
\texttt{16afea3}), as were the hypotheses (\texttt{design.md}; first commit of hypotheses:
\texttt{b101f42}; final commit: \texttt{dd0f863}).  The repository also serves all
project materials, data and analyses scripts, together with the whole project
history. It can be found at
\texttt{https://github.com/ffzg-erudito/inter-testing-feedback-2018}.
Materials are also available through \texttt{https://osf.io/gk9a3/}. The data
is also hosted on
\url{https://dataverse.ffzg.unizg.hr/dataset.xhtml?persistentId=doi:10.23669/JVNVNR}.

There were certain deviations from the original analysis plan.
Initially, we had planned to do a robustness check of our findings using
data with an additional exclusion criterion, based on the number of times
each participant had read each of the three parts of the main text. This analysis
was never conducted because (i) applying this criterion would have lead to
unacceptably low power and (ii) the participants' estimates of the number
of times they had read each part were similarly distributed across all conditions.
Further, we had planned to conduct a TOST procedure to test whether there is
no difference between the content-related and general-knowledge testing
groups. This analysis was not conducted because we did find a difference.
A Bayesian t-test was also considered for the same comparison, but was dropped
early on due to some conceptual concerns.

Also, our original second
hypothesis suggested a differential effect of 
activity type on PI rates. 
Specifically, we expected that participants engaging 
in episodic retrieval 
would display the lowest susceptibility to PI, 
followed by participants in 
the semantic retrieval condition, and finally by 
those tasked with 
rereading. Upon further examination of the literature, 
our expectations 
changed, but this had no bearing on the analyses, 
which remained the same 
nevertheless.

\clearpage

\phantomsection
\addcontentsline{toc}{section}{\refname}
\bibliographystyle{apacite}
\bibliography{reference,additional}

\clearpage

\begin{table*}

\caption{\label{tab:rb2Table}\label{rb2-table}ANOVA and ANCOVA models for the second Roy-Bargmann
                     procedure.}
\centering
\begin{tabular}[t]{lrrrr}
\toprule
Term & $SS$ & $df$ & $F$ & $p$\\
\midrule
\addlinespace[0.3em]
\multicolumn{5}{l}{\textbf{ANOVA}}\\
\hspace{1em}Activity & 109.393 & 1 & 11.200 & .001\\
\hspace{1em}Feedback & 3.904 & 1 & 0.400 & .528\\
\hspace{1em}Activity x Feedback & 0.045 & 1 & 0.005 & .946\\
\hspace{1em}Residuals & 1553.046 & 159 &  & \\
\addlinespace[0.3em]
\multicolumn{5}{l}{\textbf{ANCOVA}}\\
\hspace{1em}Activity & 0.301 & 1 & 0.175 & .676\\
\hspace{1em}Feedback & 0.173 & 1 & 0.100 & .752\\
\hspace{1em}Total correct & 63.216 & 1 & 36.760 & < .0001\\
\hspace{1em}Activity x Feedback & 0.813 & 1 & 0.473 & .493\\
\hspace{1em}Activity x Total correct & 0.862 & 1 & 0.501 & .480\\
\hspace{1em}Feedback x Total correct & 0.130 & 1 & 0.075 & .784\\
\hspace{1em}Activity x Feedback x Total correct & 1.229 & 1 & 0.715 & .399\\
\hspace{1em}Residuals & 266.551 & 155 &  & \\
\bottomrule
\end{tabular}
\end{table*}

\textbf{Figure legend:}
\begin{enumerate}[label=Figure \arabic*.]
    \item A flowchart depicting the experimental procedure.
    \item Mean number of correct answers and mean number of intrusive
        distractors chosen, broken down by experimental condition. Error bars
        show the 95\% confidence intervals around the means.

\end{document}
