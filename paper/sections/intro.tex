\documentclass[../main.tex]{subfiles}

\begin{document}

The term ``testing effect'' refers to the finding that, when it comes to 
long-term retention of a piece of information, retrieving it from memory 
trumps restudying it  
\citep{karpickeCriticalImportanceRetrieval2008,roedigeriiiPowerTestingMemory2006,
 roedigeriiiTestEnhancedLearningTaking2006, rowlandEffectTestingRestudy2014, 
adesopeRethinkingUseTests2017, roedigeriiiCriticalRoleRetrieval2011, 
gloverTestingPhenomenonNot1989}. Besides directly enhancing retention 
through repetition of \textit{successful} retrieval, testing effects can be 
brought about indirectly 
(\citealp{arnoldTestpotentiatedLearningDistinguishing2013, 
roedigeriiiPowerTestingMemory2006}; but for a different view, see 
\citealp{kornellRetrievalAttemptsEnhance2015}). For example, 
\textit{unsuccessful} retrieval attempts can, through subsequent repeated 
encoding, generate test-potentiated (re)learning (TPL), whereby taking more 
tests increases the proportion of \textit{newly} retrieved items in a test 
immediately following a restudy episode 
(\citealp{izawaReinforcementTestSequencesPairedAssociate1966,izawaOptimalPotentiatingEffects1970,
 kornellUnsuccessfulRetrievalAttempts2009, arnoldFreeRecallEnhances2013, 
arnoldTestpotentiatedLearningDistinguishing2013, 
wissmanTestpotentiatedLearningThree2018}).

More importantly, following an initial impetus provided by 
\cite{szpunarTestingStudyInsulates2008}, who built upon earlier findings 
\citep{darleyEffectsPriorFree1971, tulvingNegativeTransferEffects1974}, a 
decade of research has shown that 
retrieving previously studied information 
can even facilitate the acquisition of \textit{new} information 
\citep{chanRetrievalPotentiatesNew2018, 
	pastotterRetrievalPracticeEnhances2014, 
	yangEnhancingLearningRetrieval2018}. If, within a multi-list learning
paradigm, each subsequent study episode contains new items, testing the 
memory for those items after each learning episode still yields a greater 
number of correct responses and a decrease of proactive interference (PI) on 
a test pertaining to the final learning episode (e.g. 
\citealp{szpunarInterpolatedMemoryTests2013, 
szpunarTestingStudyInsulates2008, wissmanInterimTestEffect2011}). Following 
the reasoning of \cite{chanRetrievalPotentiatesNew2018}, we use the term 
``test-potentiated \textit{new} learning'' (TPNL) to denote this effect.
Having been investigated mainly using simple materials, a particularly 
important question for real-world applications is whether TPNL generalises 
to materials more complex than single words and word pairs. The research 
conducted in the preceding decade mostly points to a positive answer.

\cite{wissmanInterimTestEffect2011} were the first to ask whether the scope 
of TPNL extends to prose passages, and they did find a consistent effect 
using a free recall measure throughout their five experiments. As for its 
possible cause, evidence for an improvement via integration of related 
information or via PI reduction --- an explanation which figures prominently 
in the multi-list paradigm literature \citep{darleyEffectsPriorFree1971, 
szpunarTestingStudyInsulates2008} --- was rather unconvincing. The effect 
was observed regardless of the interrelatedness of passages, and, possibly 
owing to the novel methodological aspects of the study, overall intrusion 
rates were quite low. The authors subscribe to a view that retrieval 
attempts may promote the use of more effective encoding strategies. 

In a related study, \cite{divisRetrievalSpeedsContext2014} hypothesised that 
retrieval attempts enhance context fluctuations, which may lead to TPNL. 
Context changes have mostly been studied in terms of their role in 
\textit{directed forgetting}, where it has been proposed that their 
strategic use could reduce interference (e.g. 
\citealp{sahakyanContextualChangeAccount2002}). Since people store 
contextual alongside the target information during learning 
\citep{howardDistributedRepresentationTemporal2002,
	tulvingEncodingSpecificityRetrieval1973, 
	mensinkModelInterferenceForgetting1988}, perhaps the 
interplay between the two can affect the likelihood of successful target 
retrieval. If testing causes an internal context change relative to the one 
during study \citep{jangContextRetrievalContext2008, 
sahakyanContextualChangeAccount2002}, then items from separate study 
episodes might be updated with unique contextual information related to 
retrieval \citep{karpickeRetrievalBasedLearning2014, 
lehmanEpisodicContextAccount2014}, while newly encountered items remain 
associated only with the study context. In other words, additional 
contextual cues increase the disparity between pre- and post-retrieval 
items, providing an advantage to the latter by constraining the memory 
search set to items that are associated exclusively withthe study context 
\citep{szpunarTestingStudyInsulates2008}. However, if this is the case, then 
perhaps \textit{any} interpolated task causing an internal context change 
might also produce TPNL.

\hypertarget{nonepisodic}{%
	\subsection{Nonepisodic recall}}

One of the more curious findings in the field is that TPNL can indeed arise 
not only after retrieving the previously studied material (episodic 
retrieval), but also after retrieval of information unrelated to the studied 
material from semantic \citep{divisRetrievalSpeedsContext2014, 
	pastotterRetrievalLearningFacilitates2011}, or short-term memory 
	\citep{pastotterRetrievalLearningFacilitates2011}, although there have 
	been unsuccessful attempts at replication (e.g. 
\citealp{weinsteinNotAllRetrieval2015}). 

\cite{pastotterRetrievalLearningFacilitates2011} let their participants 
learn five lists of 20 words while engaging in varied interlist activities. 
They either restudied the lists, recalled the words from the list, generated 
as many words as they could from one of four semantic categories (e.g. 
professions), engaged in a 2-back short-term memory task, or counted 
backwards from a random three-digit number. They found that all three forms 
of retrieval induced TPNL. In their first experiment 
\cite{divisRetrievalSpeedsContext2014} adapted the procedure from 
\cite{pastotterRetrievalLearningFacilitates2011}. Participants learned five 
lists of 10 words, between which they either retrieved information from 
semantic memory (e.g. enumerated professions) or counted backwards by 3s 
from a three-digit number. Only semantic retrieval enhanced performance for 
final list recall. They replicated and extended these findings in their 
second experiment by using complex learning materials. Participants read 
four text passages in a self-paced manner, while engaging in one of the same 
two types of intervening tasks between reading. 

The argument these two groups of authors invoke to explain their results is 
that nonepisodic retrieval tasks sufficiently alter participant's internal 
context, and because the last study session is not affected by an additional 
context shift, a beneficial segregation of the final study context from the 
previous ones is produced. An additional prediction stemming from the 
context change hypothesis is that introducing a delay between the last study 
episode and the final test should annul the benefits conferred by the 
contextual segregation because the delay offers ample opportunity for a 
context change even without the retrieval 
\citep{chanRetrievalPotentiatesNew2018}. 
\cite{divisRetrievalSpeedsContext2014} confirmed this prediction in their 
final experiment, but \cite{chanTestingPotentiatesNew2018} provide evidence 
to the contrary, and side with the strategy change explanation suggested by 
\cite{wissmanInterimTestEffect2011}. They, however, did not include any 
other supposed context-changing interpolated tasks besides episodic 
retrieval, which prevented assessing other predictions of the hypothesis. 

Although \cite{divisRetrievalSpeedsContext2014} did find a reduction in PI 
in their first experiment, which they had predicted based on the purported 
isolating effect of context fluctuations, when their methodology precluded 
such effects in their second experiment, the results, resonating with those 
of \cite{wissmanInterimTestEffect2011}, still showed TPNL. They conceded 
that PI reduction may not be the sole basis for the observed enhanced 
performance, and mention optimisation of encoding strategy as a viable 
cause. Finally, it is worth noting that, apart from prose passages, positive 
evidence for TPNL with complex learning materials was also found using video 
lectures \citep{szpunarInterpolatedMemoryTests2013, 
	jingInterpolatedTestingInfluences2016}, but in order to clarify the
nature of TPNL within realistic settings, further inquiry is warranted.

\hypertarget{feedback}{%
\subsection{Feedback}}

Even though feedback is not necessary for the testing effect to emerge 
\citep{roedigeriiiTestEnhancedLearningTaking2006}, corrective feedback is 
known to augment the testing effect 
\citep{roedigeriiiCriticalRoleRetrieval2011}. When it is provided, 
corrective feedback increases learning because it promotes error correction 
\citep{pashlerWhenDoesFeedback2005} and reassurance 
\citep{butlerCorrectingMetacognitiveError2008}. Feedback is particularly 
important for recognition tests such as multiple-choice tests since the 
usual benefit testing provides might turn into a disadvantage in case the 
test-taker selects a lure 
\citep{roedigeriiiPositiveNegativeConsequences2005, 
	marshMemorialConsequencesMultiplechoice2007}. Further, evidence points 
	to the timing of feedback being a relevant variable when gauging its 
	influence on learning, with delayed feedback given in bulk showing 
	superior effects compared to immediate, piecemeal feedback 
\citep{metcalfeDelayedImmediateFeedback2009,butlerEffectTypeTiming2007, 
	butlerFeedbackEnhancesPositive2008,smithLearningFeedbackSpacing2010}.

Perhaps unsurprisingly, there is a paucity of research into the potential 
role of feedback in TPNL because the process by which it could affect new 
learning is less apparent. One possibility is that feedback may act 
indirectly by guiding future efforts and motivating a change in encoding 
strategy \citep{roedigeriiiPowerTestingMemory2006}. Indeed, even failed or 
unconfident retrieval can be conceptualised as feedback per se because it 
can provide insight into where additional cognitive resources should be 
allocated to improve performance (see e.g., 
\citealp{kornellUnsuccessfulRetrievalAttempts2009}). Moreover, providing 
feedback may engender additional processing 
\citep{bangert-drownsInstructionalEffectFeedback1991} which could increase 
the likelihood of intrusions during new learning 
\citep{chanRetrievalPotentiatesNew2018}. Thus, the variable of corrective 
feedback may be a fruitful avenue for research.

\subsection{Present study\label{present}}

Our study had two main goals. Firstly, we sought to replicate the TPNL 
effect in an ecologically valid setting, by using complex learning materials 
and standard multiple-choice items. Even though it has been shown that, in 
the standard multi-list procedure, substantially larger effect sizes follow 
after using free recall rather than recognition-level retrieval 
 \citep{chanRetrievalPotentiatesNew2018}, choosing to examine the impact of 
feedback on TPNL imposed constraints upon our choice of testing format; 
immediate provision of feedback would have been intractable had we chosen 
to use free recall. We used multiple-choice questions designed to assess 
memory both in terms of correct answers and susceptibility to intrusions. 
Secondly, there is a relative dearth of investigations using nonepisodic  
retrieval and recognition, and furthermore a lack of studies introducing 
corrective feedback \citep{chanRetrievalPotentiatesNew2018}. We therefore 
formed two memory tests, one of which tapped into episodic (assessing memory 
of the studied materials) while the other tapped into semantic memory 
(assessing general knowledge). Participants either were or were not given 
feedback upon completing an interpolated activity episode.

Based on the preceding discussion, if TPNL is mediated by context 
fluctuations, then we should find that both types of retrieval enhance new 
learning, whereas rereading, which does not bring about context changes 
\citep{chanRetrievalPotentiatesNew2018} does not. \textbf{We expected that 
participants engaging in episodic retrieval would display the lowest 
susceptibility to PI, followed by participants in the semantic retrieval 
condition, and finally by those in the rereading condition.} <- discuss.

We assumed that 
presenting feedback would have a positive effect on memory performance, but 
only for the participants engaging in episodic recall because only they 
could alter their encoding strategies accordingly. Finally, we expected to 
find an interaction effect of activity type and feedback presentation on the 
number of intrusions, but did not set a specific prediction regarding its 
pattern.
 
{
	\setstretch{1}
	\biblio
}

\end{document}
