\documentclass[../main.tex]{subfiles}

\begin{document}

The term ``testing effect'' refers to the finding that, when it comes to 
long-term retention of a piece of information, retrieving it from memory 
trumps restudying it 
\citep{karpickeCriticalImportanceRetrieval2008,roedigeriiiPowerTestingMemory2006,
 roedigeriiiTestEnhancedLearningTaking2006, rowlandEffectTestingRestudy2014, 
adesopeRethinkingUseTests2017, roedigeriiiCriticalRoleRetrieval2011, 
gloverTestingPhenomenonNot1989}. Besides directly enhancing retention 
through repetition of successful retrieval, testing effects can be brought 
about indirectly (\citealp{arnoldTestpotentiatedLearningDistinguishing2013, 
roedigeriiiPowerTestingMemory2006}; but for a different view, see 
\citealp{kornellRetrievalAttemptsEnhance2015}). For example, unsuccessful 
retrieval attempts can, through subsequent repeated encoding, generate 
test-potentiated (re)learning (TPL;
\citealp{izawaReinforcementTestSequencesPairedAssociate1966,izawaOptimalPotentiatingEffects1970,
 kornellUnsuccessfulRetrievalAttempts2009, arnoldFreeRecallEnhances2013, 
arnoldTestpotentiatedLearningDistinguishing2013, 
wissmanTestpotentiatedLearningThree2018}).

After an initial impetus provided by 
\cite{szpunarTestingStudyInsulates2008}, who built upon earlier findings 
\citep{darleyEffectsPriorFree1971, tulvingNegativeTransferEffects1974}, a 
decade of research has shown that retrieving previously studied information 
can even facilitate the acquisition of \textit{new} information 
\citep{chanRetrievalPotentiatesNew2018, 
pastotterRetrievalPracticeEnhances2014, yangEnhancingLearningRetrieval2018}. 
If each subsequent study episode in the paradigm used to demonstrate TPL 
contains new materials (giving a now standard blocked design; 
\citealp{chanTestingPotentiatesNew2018}), one still observes that testing 
the memory for those new materials after each learning episode yields a 
greater number of correct responses and a decrease of proactive interference 
(PI) on a test administered to all subjects after the final learning episode 
(e.g. \citealp{szpunarInterpolatedMemoryTests2013, 
szpunarTestingStudyInsulates2008, wissmanInterimTestEffect2011}). Following 
the reasoning of \cite{chanRetrievalPotentiatesNew2018}, we use the term 
``test-potentiated \textit{new} learning'' (TPNL) to denote this effect.
With studies mainly using the multilist learning paradigm to delineate the 
scope of TPNL, a particularly important question for real-world applications 
is whether these results generalise to materials more complex than word 
lists, and research conducted in the preceding decade mostly points to a 
positive answer (prose passages: \citealp{wissmanInterimTestEffect2011, 
divisRetrievalSpeedsContext2014}; video lectures: 
\citealp{szpunarInterpolatedMemoryTests2013, 
jingInterpolatedTestingInfluences2016}). Summarising the results of their 
metaregression, \cite{chanRetrievalPotentiatesNew2018} highlighted resource 
and integration theories as accounts which have thus far garnered more 
empirical support, giving a slight upper hand to integration theories, while 
stating that context theories are least supported by extant research. 
Therefore, we opted to align our study design with the goal of comparing 
resource and integration frameworks.

\hypertarget{nonepisodic}{%
\subsection{Nonepisodic recall and feedback}}

One of the more curious findings in the field is that TPNL can arise not 
only after retrieving the previously studied material (episodic retrieval), 
but also after retrieval of information unrelated to the studied material 
from semantic \citep{divisRetrievalSpeedsContext2014, 
pastotterRetrievalLearningFacilitates2011}, or short-term memory 
\citep{pastotterRetrievalLearningFacilitates2011}, although there have been 
unsuccessful attempts at replication (e.g. 
\citealp{weinsteinNotAllRetrieval2015}). 
\cite{pastotterRetrievalLearningFacilitates2011} demonstrated this using 
lists of words, while \cite{divisRetrievalSpeedsContext2014} replicated and 
extended these findings using prose passages.

Although corrective feedback is known to augment the testing effect 
\citep{roedigeriiiCriticalRoleRetrieval2011}, there is a paucity of research 
into the effect of feedback on TPNL. Feedback is particularly important for 
recognition tests such as multiple-choice tests since the usual benefit 
testing confers might turn into a disadvantage in case the test-taker 
selects a lure \citep{roedigerPositiveNegativeConsequences2005, 
marshMemorialConsequencesMultiplechoice2007}. Moreover, evidence points to 
the timing of feedback being a relevant variable when gauging its influence 
on learning, with delayed feedback given in bulk showing superior effects 
compared to immediate, piecemeal feedback 
\citep{metcalfeDelayedImmediateFeedback2009,butlerEffectTypeTiming2007, 
butlerFeedbackEnhancesPositive2008,smithLearningFeedbackSpacing2010}.
The variable of corrective feedback may be a fruitful avenue for research 
because resource and integration theories provide conflicting predictions 
regarding its effects on TPNL \citep{chanRetrievalPotentiatesNew2018}. 
Providing corrective feedback should increase the likelihood of intrusions 
during new learning, which are deemed beneficial from the standpoint of 
integration theories, but detrimental according to resource theories.


\subsection{Present study\label{present}}

Our study had two main goals. Firstly, we sought to replicate the TPNL 
effect in an ecologically valid setting, by using complex learning materials 
and standard multiple-choice items. Even though it has been shown that, in 
the standard TPNL procedure, substantially larger effect sizes follow after 
using free recall rather than recognition-level retrieval 
 \citep{chanRetrievalPotentiatesNew2018}, choosing to examine the impact of 
 feedback on TPNL imposed constraints upon our choice of testing format; 
 immediate provision of feedback would have been intractable had we chosen 
 to use free recall. We used multiple-choice questions designed to assess 
 memory both in terms of correct answers and susceptibility to intrusions. 
 Secondly, there is a relative dearth of investigations using nonepisodic 
 retrieval and recognition, and furthermore a lack of studies introducing 
 feedback in a blocked study design 
\citep{chanRetrievalPotentiatesNew2018}. We therefore formed two memory 
tests, one of which tapped into episodic (assessing memory of the studied 
materials) while the other tapped into semantic memory (assessing general 
knowledge). Participants either were or were not given feedback upon 
completing an interpolated activity episode.

Based on the preceding discussion, we predicted that participants in the 
retrieval groups would display TPNL, whereas a control rereading group would 
not. We expected that participants engaging in episodic retrieval would 
display the lowest susceptibility to PI, followed by participants in the 
semantic retrieval condition, and finally by those in the rereading 
condition. We assumed that presenting feedback would have a positive effect 
on memory performance, but only for the participants engaging in episodic 
recall. We also predicted receiving feedback would significantly increase 
interference. Finally, we expected to find an interaction effect of activity 
type and feedback presentation on the number of intrusions, but did not set 
a specific prediction regarding its pattern.
 
{
	\setstretch{1}
	\biblio
}

\end{document}
