\documentclass[../main.tex]{subfiles}

\begin{document}

Juxtaposed to the well established finding that recalling learned material, compared to merely restudying it, facilitates its long-term retention \citep{roedigeriiiPowerTestingMemory2006, roedigeriiiTestEnhancedLearningTaking2006, rowlandEffectTestingRestudy2014, adesopeRethinkingUseTests2017, roedigeriiiCriticalRoleRetrieval2011} stand the results of a number of studies showing that retrieving previously studied information can even facilitate the acquisition of new information \citep{chanRetrievalPotentiatesNew2018}. This has stimulated the use of various qualifiers such as \textit{interim} \citep{wissmanInterimTestEffect2011}, \textit{interpolated} \citep{szpunarInterpolatedMemoryTests2013}, and \textit{forward} \citep{pastotterRetrievalPracticeEnhances2014,yangEnhancingLearningRetrieval2018}, in order to distinguish the latter testing effect from the former, now sometimes referred to as the \textit{backward} testing effect (e.g \citealp{yangEnhancingLearningRetrieval2018}). 

A typical demonstration of this better-known testing effect entails an initial learning phase, followed by a period during which participants either restudy the same material, engage in a memory test involving the studied material, or are not exposed to the original material at all. Finally, after a retention interval, an additional administration of a memory test reveals that the group subjected to a memory test during the intervening period has a distinct advantage over the other two groups. A pioneering study by \cite{szpunarTestingStudyInsulates2008} introduced a difference prompted a surge of interest in the somewhat dif effect mentioned above.  wherein the recall of previously studied material seemingly potentiates the acquisition of information that is yet to be learned. Throughout this paper, we will use the term \textit{forward testing effect} to refer to the phenomenon under investigation. 

A recent meta-analysis termed 

{
    \setstretch{1}
    \biblio
}

\end{document}

one has attempted to commit to memory beforehand