\documentclass[../main.tex]{subfiles}

\begin{document}

Juxtaposed to the well established finding that recalling learned material, compared to merely restudying it, facilitates its long-term retention \citep{roedigeriiiPowerTestingMemory2006, roedigeriiiTestEnhancedLearningTaking2006, rowlandEffectTestingRestudy2014, adesopeRethinkingUseTests2017, roedigeriiiCriticalRoleRetrieval2011} stand the results of a study by \cite{szpunarTestingStudyInsulates2008}. The typical demonstration of the testing effect entails an initial learning phase, an intervening period during which participants either restudy the same material or   which prompted a surge of interest in a somewhat different type of testing effect, wherein the recall of previously studied material seemingly potentiates the acquisition of information that is yet to be learned. This has stimulated the use of various qualifiers such as \textit{interim} \citep{wissmanInterimTestEffect2011}, \textit{interpolated} \citep{szpunarInterpolatedMemoryTests2013}, and \textit{forward} \citep{pastotterRetrievalPracticeEnhances2014,yangEnhancingLearningRetrieval2018}, in order to distinguish this testing effect from the one described above, now sometimes referred to as the \textit{backward} testing effect (e.g \citealp{yangEnhancingLearningRetrieval2018}). Throughout this paper, we will use the term \textit{forward testing effect} to refer to the phenomenon under investigation. A recent meta-analysis termed 

{
    \setstretch{1}
    \biblio
}

\end{document}
