\documentclass[../main.tex]{subfiles}

\begin{document}

The protagonist in a standard educational setting is the student, who exerts effort to commit some information to memory, with the written and oral exams being the primary obstacles on their path to academic achievement. Be that as it may, current theories of testing look beyond the test's most dreaded function to appreciate the frequently overlooked beneficial effect it can have on learning and memory. In fact, the term "testing effect" refers to the finding that, when it comes to long-term retention of a piece of information, retrieving that information from memory trumps restudying it \citep{roedigeriiiPowerTestingMemory2006, roedigeriiiTestEnhancedLearningTaking2006, rowlandEffectTestingRestudy2014, adesopeRethinkingUseTests2017, roedigeriiiCriticalRoleRetrieval2011, gloverTestingPhenomenonNot1989}. 

It is generally recognised that testing can have two types of effects - \textit{direct} and \textit{indirect} \citep{arnoldTestpotentiatedLearningDistinguishing2013, roedigeriiiPowerTestingMemory2006}. Direct effects refer specifically to the increased retention that ensues from repeatedly \textit{successfully} retrieving the target information - a process which is, presumably, reactivated at the time of a later test. A typical demonstration of the testing effect entails an initial learning phase, followed by a period during which participants either restudy the same material, engage in a memory test involving the studied material, or are not exposed to the original material at all. Finally, after a retention interval, an additional memory test reveals that the group subjected to a memory test during the intervening period has a distinct advantage over the other two groups. 

On the other hand, indirect effects are brought about by some other process or processes besides the act of taking the test (\citealp{roedigeriiiPowerTestingMemory2006}; but for a different view, see \citealp{kornellRetrievalAttemptsEnhance2015}). One of those is the mediated effect of \textit{unsuccessful} retrieval attempts (which are not followed by feedback) through later encoding, namely \textit{test-potentiated (re)learning} \citep{izawaReinforcementTestSequencesPairedAssociate1966,izawaOptimalPotentiatingEffects1970, kornellUnsuccessfulRetrievalAttempts2009, arnoldFreeRecallEnhances2013, arnoldTestpotentiatedLearningDistinguishing2013}. In order to demonstrate test-potentiated learning, one would have to show that taking a test (or more tests) produces a greater increase in the proportion of \textit{newly} retrieved items in a test following a subsequent additional study episode, compared to a condition where the subjects engage in some other activity (or take fewer tests). \cite{arnoldTestpotentiatedLearningDistinguishing2013} did just that, thereby isolating the effect of unsuccessful from successful retrieval attempts.

\subsection*{Test-potentiated new learning}

However, juxtaposed to the well established finding that attempting to recall learned material, be it successful or unsuccessful, compared to merely restudying it, facilitates the long-term retention of \textit{that} material, stand the results of a decade of research showing that retrieving previously studied information can even facilitate the acquisition of \textit{new} information \citep{chanRetrievalPotentiatesNew2018, yangEnhancingLearningRetrieval2018}. If each additional study episode in the paradigm used to demonstrate test-potentiated learning contains \textit{new} materials, one still observes that taking a test on the new materials after each learning episode yields a greater number of correct responses and a decrease of proactive interference on a test administered to all subjects after the final learning episode. This has stimulated the use of various qualifiers such as \textit{interim} (e.g. \citealp{wissmanInterimTestEffect2011}), \textit{interpolated} (e.g. \citealp{szpunarInterpolatedMemoryTests2013}), and \textit{forward} (e.g.  \citealp{pastotterRetrievalPracticeEnhances2014,yangEnhancingLearningRetrieval2018}), in order to distinguish this effect from the other two better-known testing effects. Following the reasoning of \cite{chanRetrievalPotentiatesNew2018}, we will use the term "test-potentiated \textit{new} learning" (TPNL) to denote this effect.

In one of the earliest studies showing the effect of TPNL, \cite{darleyEffectsPriorFree1971} observed that, when recalling studied lists of words, participants systematically produce more prior-list intrusions when probed for a given list, if their memory of a prior list had not been tested before they proceeded to study the given list. The findings were soon corroborated by \cite{tulvingNegativeTransferEffects1974}. Building on these results, after a considerable period of silence on the topic, \cite{szpunarTestingStudyInsulates2008} told their subjects to study five lists of items in anticipation of a final cumulative test. All subjects were tested immediately after studying the final list, but they engaged in different intermittent activities between studying the first four lists. One group was tested on each list after studying it, another group restudied each list, and a third group completed a mathematical distractor task. Participants whose memory was tested after each list produced more correct responses and fewer prior-list intrusions on the immediate test administered after studying the last list, compared to the groups that were not tested. The authors explained the found benefit of testing in terms of a segregation mechanism that prevents overburdening of retrieval cues, which, in the absence of testing, causes a build-up of proactive interference. The following decade has seen a renewal of interest in TPNL \citep{chanRetrievalPotentiatesNew2018, pastotterRetrievalPracticeEnhances2014, yangEnhancingLearningRetrieval2018}, with studies mainly using the multilist learning paradigm to delineate the scope of the effect with respect to various moderating variables: the type of study materials, varieties of study designs, and populations, to name but a few. 

\subsection*{Theoretical overview}

Recently, \cite{chanRetrievalPotentiatesNew2018} provided a meta-analytic analysis and comprehensive overview of the literature, identifying four \textit{nonconflicting} theoretical frameworks which were put forth throughout the years as viable explanations for TPNL. \textit{Resource theories} generally posit that testing increases cognitive resources, but they propose different mechanisms by which this is achieved: (1) proactive interference reduction (e.g. \citealp{wahlheimTestingCanCounteract2015, weinsteinTestingProtectsProactive2011, szpunarTestingStudyInsulates2008, nunesTestingImprovesTrue2012}), (2) restoration of encoding/attentional resources (e.g. \citealp{pastotterRetrievalLearningFacilitates2011}), or (3) alteration of mind wandering patterns (e.g. \citealp{jingInterpolatedTestingInfluences2016,szpunarInterpolatedMemoryTests2013,szpunarMindWanderingEducation2013}). Whereas resource theories focus on the amount of deployable cognitive resources, \textit{metacognitive theories} emphasise the optimisation of encoding strategies induced by retrieval attempts (e.g. \citealp{choTestingEnhancesBoth2017, chanTestingPotentiatesNew2018}). For example, in a recent investigation, \cite{chanTestingPotentiatesNew2018} found that, compared to untested groups, the group whose memory for the first three word lists was subjected to interpolated testing displayed superior semantic organisation across lists. A similar result was found for the testing effect \citep{zarombTestingEffectFree2010}.

The key idea underlying \textit{context theories} is that, apart from storing the studied information per se, people store the related contextual information as well (e.g. \citealp{lehmanEpisodicContextAccount2014}). Afterwards, the accessibility of this contextual information can affect the likelihood of successful retrieval of target information. Furthermore, the claim is that, unlike restudying, attempting retrieval causes an internal context change relative to the study context \citep{jangContextRetrievalContext2008, sahakyanContextualChangeAccount2002}, and recalled items may be updated with contextual information from the retrieval attempt, while newly encountered information is still associated only with the study context. Therefore, recalling new-learning items is limited to only those items associated exclusively with the study context, providing them with the advantage observed upon testing. While this veritable circumscription of separate learning episodes is at the core of both resource and context accounts, its effect on learning is supposedly different. According to the former, isolating a learning episode through attempts at recall increases resources for subsequent learning by preventing \textit{encoding-based} proactive interference. On the other hand, the latter place the emphasis on later \textit{retrieval} processes, whereby isolating an earlier learning episode reduces the memory search set for retrieval. 

Finally, \textit{integration theories} advance the notion that interpolated testing facilitates the integration of the new-learning material either with its retrieval cues or with the original-learning material. On one account, testing increases the likelihood of spontaneous covert retrieval of original-learning items during the study of new items, fostering their integration, and, thus, increasing conceptual organisation (e.g. \citealp{jingInterpolatedTestingInfluences2016}) and the effectiveness of retrieval cues \citep{pycWhyTestingImproves2010}. For example, \cite{jingInterpolatedTestingInfluences2016} found that interpolated testing increased the clustering of related information that is acquired across different segments within a video-recorded lecture.

\subsection*{Nonepisodic recall}

One of the more curious findings in the field is that TPNL can arise not only after retrieving the previously studied material (episodic retrieval), but also after retrieval from semantic memory of information unrelated to the studied material \citep{divisRetrievalSpeedsContext2014, pastotterRetrievalLearningFacilitates2011}, or from short-term memory \citep{pastotterRetrievalLearningFacilitates2011}, although there have been unsuccessful attempts at replication (e.g. \citealp{weinsteinNotAllRetrieval2015}).

\cite{pastotterRetrievalLearningFacilitates2011} found that all three forms of retrieval induced TPNL. Their participants learned five lists of 20 words while engaging in varied interlist activities. They either restudied the lists, recalled the words from the list, generated as many words as they could from one of four semantic categories (e.g. professions), engaged in a 2-back short-term memory task, or counted backwards from a random three-digit number. In their first experiment, \cite{divisRetrievalSpeedsContext2014} adapted the procedure from \cite{pastotterRetrievalLearningFacilitates2011}, using only the semantic generation and distraction tasks, and found that interleaved semantic retrieval enhanced performance for final list recall. They replicated and extended these findings in their second experiment by using complex learning materials: lists of words were replaced by texts related to animals, while learning was assessed with short-answer and multiple-choice questions. However, their protocol didn't allow for assessing the contribution of proactive interference on TPNL.

The argument invoked to explain these results is that nonepisodic retrieval tasks sufficiently alter participant's internal context. Because the last study session is not affected by an additional context shift, a beneficial segregation of the final study context from the previous ones is produced.
	


\subsection{Feedback}


\subsection{Present study}

Our study had several aims. Firstly, we sought to replicate the effect in an ecologically valid setting, by using complex learning materials and standard multiple-choice items. Secondly, acknowledging the relative dearth of studies using nonepisodic retrieval as the interpolated activity, and furthermore a lack of studies using a blocked (vs. ) on the moderating influence of feedback in TPNL,    guided by the analysis of gaps in the field provided by \cite{chanRetrievalPotentiatesNew2018}. .  Secondly, we attempted to expand the existing body of literature.

Wanting to examine TPNL in a more ecologically valid setting, we...





{
	\setstretch{1}
	\biblio
}

\subsection{Notes}



Za feedback gle u Roediger i Butler (2011)




Chan et al. (2018). across a retention interval
In contrast to unrelated word lists, text passages and videos are
typically written/produced in a coherent manner, which should naturally
invite relational processing, so any relational processing advantage
induced by prior testing is likely to be modest relative to
%baseline (Einstein, McDaniel, Bowers, & Stevens, 1984; Einstein,
%McDaniel, Owen, & Cote, 1990; Masson & McDaniel, 1981). A version
of the strategy change account that is not tied strictly to relational
processing, however, may provide a reasonable explanation for the
TPNL effect with text passages and videos. In a broader sense, the
strategy change account specifies that performing retrieval practice
allows participants to discover the type of learning needed to ensure
satisfactory performance (or conversely, to realize the type of learning
that is inadequate to produce satisfactory performance, if participants
are performing poorly during retrieval practice), and participants can
then adjust their subsequent encoding strategy accordingly. If we take
this broader approach to strategy change, then this account can explain
the TPNL effect with prose/video materials. However, we realize that
the idea that “retrieval practice can improve later encoding strategies”
is perhaps vaguely defined. In fact, such a broad definition of strategy
change may render the account difficult to falsify. With this in mind, we
believe that the strategy change account, as we currently conceive,
should only be applied to explain the TPNL effect with word list type
materials, for which advantageous encoding strategies can be more
precisely defined (but see Jing et al., 2016 in which interspersed testing
improved conceptual integration of materials across sections of a video
lecture). In our opinion, application of this account to prose/video
material should only be done when one clearly outlines what is considered
an advantageous encoding strategy so that the hypothesis can
be adequately tested.


Možda ZV intruzori nije pokazala razlike između skupina jer smo koristili recognition, a ne free recall.
Context change account?


if taking more prior tests enhanced the proportion newly retrieved items,
the additional tests can be assumed to have potentiated learning during the restudy trial.


DISCUSSION:
Methodological concerns. The expectation of a final test ensured the
continued processing of materials across the study sequence.
Chan et al. (2018): % za istaknuti mogućnost da su se očekivanja ispitanika dinamički mijenjala tijekom mjerenja
For example, in a multilist learning environment, having taken a
recent memory test increases learners’ expectation that they will
again be tested in the immediate future, even when they are told
that whether a test will follow each study list is determined
randomly (Weinstein et al., 2014). Such test expectancies have
been shown to significantly influence how participants approach
%the encoding task (Balota & Neely, 1980; May & Thompson,
%1989; Szpunar, McDermott, & Roediger, 2007).
Weinstein et al. (2014):
The experiments reported here are based on the assumption that
participants in the tested group may be more likely to expect a test
after the fifth (last) list, having consistently received tests after
previous lists. Those in the untested group, having never received
a test during the experiment until the fifth list test, may therefore
pay less attention or engage in lower quality encoding strategies
during encoding of the fifth list. To test for this alternative explanation, 
we compared the two standard conditions (tested and
untested) with two novel conditions that were identical to the
standard conditions but included a warning before presentation of
the final list to alert participants that they would be tested on the
upcoming list. If attentional processes are mediating the observed
release from proactive interference, warning participants in the
untested group should produce the same benefit as the participants
taking a test after every list


Matej [11:05 PM]
E, sjetio sam se opet da bi možda bilo dobro da negdje navedemo razlike u učinku na testovima prije zadnjega...

Denis [11:09 PM]
Mda, neke te stvari su mi pale na pamet, i činilo mi se kao da bi bilo zgodno spomenuti ih u raspravi. Kao, ako se radi o nekim stvarima koje bi mogle ugroziti efekte ili objasniti neznačajne



Pastötter et al. (2011):
Whereas, relative to the two no-retrieval conditions,
both episodic and semantic memory retrieval effectively
eliminated List 1–4 intrusions during immediate List 5 recall,
short-term memory retrieval led to intrusion rates that were equivalent
in amount to the no-retrieval conditions. This difference in
intrusion errors may suggest that the effects of short-term and
long-term memory retrieval are not perfectly identical.

\end{document}

