\documentclass[../main.tex]{subfiles}

\begin{document}

Juxtaposed to the well established finding that recalling learned material, compared to merely restudying it, facilitates its long-term retention \citep{roedigeriiiPowerTestingMemory2006, roedigeriiiTestEnhancedLearningTaking2006, rowlandEffectTestingRestudy2014, adesopeRethinkingUseTests2017, roedigeriiiCriticalRoleRetrieval2011, gloverTestingPhenomenonNot1989} stand the results of a number of studies showing that retrieving previously studied information can even facilitate the acquisition of information that is yet to be learned \citep{chanRetrievalPotentiatesNew2018, yangEnhancingLearningRetrieval2018}. This has stimulated the use of various qualifiers such as \textit{interim} \citep{wissmanInterimTestEffect2011}, \textit{interpolated} \citep{szpunarInterpolatedMemoryTests2013}, and \textit{forward} \citep{pastotterRetrievalPracticeEnhances2014,yangEnhancingLearningRetrieval2018}, in order to distinguish the latter testing effect from the former, now sometimes referred to as the \textit{backward} testing effect (e.g \citealp{yangEnhancingLearningRetrieval2018}). Following the reasoning of \cite{chanRetrievalPotentiatesNew2018}, we will differentiate the two testing effects by using the terms "test-potentiated learning" and "test-potentiated \textit{new} learning" appropriately.

A typical demonstration of test-potentiated learning entails an initial learning phase, followed by a period during which participants either restudy the same material, engage in a memory test involving the studied material, or are not exposed to the original material at all. Finally, after a retention interval, an additional administration of a memory test reveals that the group subjected to a memory test during the intervening period has a distinct advantage over the other two groups. In one of the earliest studies showing the effect of test-potentiated new learning, \cite{darleyEffectsPriorFree1971} observed that, when recalling studied lists of words, participants systematically produce more prior-list intrusions when probed for a given list, if their memory of a prior list had not been tested before they proceeded to study the given list. 

Building on these results, after a considerable period of silence on the topic, \cite{szpunarTestingStudyInsulates2008} told their subjects to study five lists of items in anticipation of a final cumulative test. All subjects were tested immediately after studying the final list, but they engaged in different intermittent activities between studying the first four lists. One group was tested on each list after studying it, another group restudied each list, and a third group completed a mathematical distractor task. Participants whose memory was tested after each list produced more correct responses and fewer prior-list intrusions on the immediate test administered after studying the last list, compared to the groups that were not tested. The authors explained the found benefit of testing in terms of a segregation mechanism that prevents overburdening of retrieval cues, which, in the absence of testing, causes a build-up of proactive interference. The following decade has seen a renewal of interest in test-potentiated new learning \citep{chanRetrievalPotentiatesNew2018, pastotterRetrievalPracticeEnhances2014, yangEnhancingLearningRetrieval2018}, with studies attempting to delineate the scope of the effect with respect to various moderating variables: the type of study materials, varieties of study designs, and populations, to name but a few. 

\subsection{Theoretical overview}

Recently, \cite{chanRetrievalPotentiatesNew2018} provided a meta-analytic analysis and comprehensive overview of the literature, identifying four \textit{nonconflicting} theoretical frameworks which were put forth throughout the years as viable explanations for test-potentiated new learning. \textit{Resource theories} generally posit that testing increases cognitive resources, but they propose different mechanisms by which this is achieved: (1) proactive interference reduction (e.g. \citealp{wahlheimTestingCanCounteract2015, weinsteinTestingProtectsProactive2011, szpunarTestingStudyInsulates2008, nunesTestingImprovesTrue2012}), (2) restoration of encoding/attentional resources (e.g. \citealp{pastotterRetrievalLearningFacilitates2011}), or (3) alteration of mind wandering patterns (e.g. \citealp{jingInterpolatedTestingInfluences2016,szpunarInterpolatedMemoryTests2013,szpunarMindWanderingEducation2013}). Whereas resource theories focus on the amount of deployable cognitive resources, \textit{metacognitive theories} emphasise the optimisation of encoding strategies induced by retrieval attempts (e.g. \citealp{choTestingEnhancesBoth2017, chanTestingPotentiatesNew2018}). For example, in a recent investigation, \cite{chanTestingPotentiatesNew2018} found that, compared to untested groups, the group whose memory for the first three word lists was subjected to interpolated testing displayed superior semantic organisation across lists.

The key idea underlying \textit{context theories} is that, apart from storing the studied information per se, people store the related contextual information as well (e.g. \citealp{lehmanEpisodicContextAccount2014}). Afterwards, the accessibility of this contextual information can affect the likelihood of successful retrieval of target information. Furthermore, the claim is that, unlike restudying, attempting retrieval causes an internal context change relative to the study context (e.g. one may engage a retrieval mode), and recalled items may be updated with contextual information from the retrieval attempt, while newly encountered information is still associated only with the study context. Therefore, recalling new-learning items is limited to only those items associated exclusively with the study context, providing them with the advantage observed upon testing. While this veritable circumscription of separate learning episodes is at the core of both resource and context accounts, its effect on learning is supposedly different. According to the former, isolating a learning episode through attempts at recall increases resources for subsequent learning by preventing \textit{encoding-based} proactive interference. On the other hand, the latter place the emphasis on later \textit{retrieval} processes, whereby isolating an earlier learning episode reduces the memory search set for retrieval. 

Finally, \textit{integration theories} advance the notion that interpolated testing facilitates the integration of the new-learning material either with its retrieval cues or with the original-learning material. On one account, testing increases the likelihood of spontaneous covert retrieval of original-learning items during the study of new items, fostering their integration and conceptual organisation (e.g. \citealp{jingInterpolatedTestingInfluences2016}), and, thus, increasing the effectiveness of retrieval cues \citep{pycWhyTestingImproves2010}. For example, \cite{jingInterpolatedTestingInfluences2016} found that interpolated testing increased the clustering of related information that is acquired across different segments within a video-recorded lecture.

Guided by the analysis of gaps in the literature provided by \cite{chanRetrievalPotentiatesNew2018}, we aimed to expand on the 

Wanting to examine test-potentiated new learning in a more ecologically valid setting, we...

\subsection{Nonepisodic recall}

According to current theories of testing, an initial test will directly benefit retention to the extent that it engages those effortful retrieval processes that will likely be called upon at the time of a later test \cite{roedigeriiiPowerTestingMemory2006}
A number of investigations utilised nonepisodic recall as the format of the intermittent testing activity \cite{divisRetrievalSpeedsContext2014, pastotterRetrievalLearningFacilitates2011, }


\subsection{Feedback}


{
    \setstretch{1}
    \biblio
}

\subsection{Notes}




In this study we sought to examine the influence of one proposed moderating variables, and to pitch two contending theoretical overviews against each other.

Za feedback gle u Roediger i Butler (2011)




Chan et al. (2018). across a retention interval
In contrast to unrelated word lists, text passages and videos are
typically written/produced in a coherent manner, which should naturally
invite relational processing, so any relational processing advantage
induced by prior testing is likely to be modest relative to
%baseline (Einstein, McDaniel, Bowers, & Stevens, 1984; Einstein,
%McDaniel, Owen, & Cote, 1990; Masson & McDaniel, 1981). A version
of the strategy change account that is not tied strictly to relational
processing, however, may provide a reasonable explanation for the
TPNL effect with text passages and videos. In a broader sense, the
strategy change account specifies that performing retrieval practice
allows participants to discover the type of learning needed to ensure
satisfactory performance (or conversely, to realize the type of learning
that is inadequate to produce satisfactory performance, if participants
are performing poorly during retrieval practice), and participants can
then adjust their subsequent encoding strategy accordingly. If we take
this broader approach to strategy change, then this account can explain
the TPNL effect with prose/video materials. However, we realize that
the idea that “retrieval practice can improve later encoding strategies”
is perhaps vaguely defined. In fact, such a broad definition of strategy
change may render the account difficult to falsify. With this in mind, we
believe that the strategy change account, as we currently conceive,
should only be applied to explain the TPNL effect with word list type
materials, for which advantageous encoding strategies can be more
precisely defined (but see Jing et al., 2016 in which interspersed testing
improved conceptual integration of materials across sections of a video
lecture). In our opinion, application of this account to prose/video
material should only be done when one clearly outlines what is considered
an advantageous encoding strategy so that the hypothesis can
be adequately tested.


Možda ZV intruzori nije pokazala razlike između skupina jer smo koristili recognition, a ne free recall.
Context change account?

The expectation of a final test ensured the
continued processing of materials across the study sequence.
Chan et al. (2018): % za istaknuti mogućnost da su se očekivanja ispitanika dinamički mijenjala tijekom mjerenja
For example, in a multilist learning environment, having taken a
recent memory test increases learners’ expectation that they will
again be tested in the immediate future, even when they are told
that whether a test will follow each study list is determined
randomly (Weinstein et al., 2014). Such test expectancies have
been shown to significantly influence how participants approach
%the encoding task (Balota & Neely, 1980; May & Thompson,
%1989; Szpunar, McDermott, & Roediger, 2007).

\end{document}

