\documentclass[../main.tex]{subfiles}

\begin{document}

As opposed to the now well established finding that recalling learned material compared to merely restudying it facilitates its long-term retention \cite{roedigeriiiPowerTestingMemory2006, roedigeriiiTestEnhancedLearningTaking2006, rowlandEffectTestingRestudy2014, adesopeRethinkingUseTests2017}, the initial findings of \cite{szpunarTestingStudyInsulates2008} prompted a surge of interest in a somewhat different type of testing effect, wherein the recall of previously studied material seemingly potentiates the acquisition of information that is yet to be learned. This has stimulated the use of various qualifiers such as "interim" \citep{wissmanInterimTestEffect2011}, "interpolated" \citep{szpunarInterpolatedMemoryTests2013}, and "forward" \cite{pastaptterRetrievalPracticeEnhances2014, yangEnhancingLearningRetrieval2018}, in order to distinguish this testing effect from the one described above. 


\cite{}

{
    \setstretch{1}
    \biblio
}

\end{document}
