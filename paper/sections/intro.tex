\documentclass[../main.tex]{subfiles}

\begin{document}

Juxtaposed to the well established finding that recalling learned material, compared to merely restudying it, facilitates its long-term retention \citep{roedigeriiiPowerTestingMemory2006, roedigeriiiTestEnhancedLearningTaking2006, rowlandEffectTestingRestudy2014, adesopeRethinkingUseTests2017, roedigeriiiCriticalRoleRetrieval2011, gloverTestingPhenomenonNot1989} stand the results of a number of studies showing that retrieving previously studied information can even facilitate the acquisition of new information \citep{chanRetrievalPotentiatesNew2018a, yangEnhancingLearningRetrieval2018}. This has stimulated the use of various qualifiers such as \textit{interim} \citep{wissmanInterimTestEffect2011}, \textit{interpolated} \citep{szpunarInterpolatedMemoryTests2013}, and \textit{forward} \citep{pastotterRetrievalPracticeEnhances2014,yangEnhancingLearningRetrieval2018}, in order to distinguish the latter testing effect from the former, now sometimes referred to as the \textit{backward} testing effect (e.g \citealp{yangEnhancingLearningRetrieval2018}). 

A typical demonstration of the backward testing effect entails an initial learning phase, followed by a period during which participants either restudy the same material, engage in a memory test involving the studied material, or are not exposed to the original material at all. Finally, after a retention interval, an additional administration of a memory test reveals that the group subjected to a memory test during the intervening period has a distinct advantage over the other two groups. A pioneering study by \cite{darleyEffectsPriorFree1971} introduced a key difference which revealed a somewhat different effect. In particular, they observed that, when recalling studied lists of words, participants would systematically produce more prior-list intrusions when their memory for a given list was tested, if their memory of a prior list had not been tested before proceeding to study the given list. 

Building on these results, \cite{szpunarTestingStudyInsulates2008} told their subjects to study five lists of items in anticipation of a final cumulative test. All subjects were tested immediately after studying the final list, but they engaged in differing intermittent activities between studying the first four lists. One group was tested on each list after studying it, another group restudied each list, and the third group completed a mathematical distractor task.


 wherein the recall of previously studied material seemingly potentiates the acquisition of information that is yet to be learned. that testing can promote  prompted a surge of interest in the somewhat different effect mentioned above.   Throughout this paper, we will use the term \textit{forward testing effect} to refer to the phenomenon under investigation. 



{
    \setstretch{1}
    \biblio
}

\subsection{Notes}
one has attempted to commit to memory beforehand

According to current theories of testing, an initial test will directly benefit retention to the extent that it engages those effortful retrieval processes that will likely be called upon at the time of a later test \cite{roedigeriiiPowerTestingMemory2006}

The expectation of a final test ensured the
continued processing of materials across the study sequence.


A number of investigations employed nonepisodic recall as the format of the intermittent testing activity \cite{divisRetrievalSpeedsContext2014, pastotterRetrievalLearningFacilitates2011, }


Možda ZV intruzori nije pokazala razlike između skupina jer smo koristili recognition, a ne free recall.

\end{document}

