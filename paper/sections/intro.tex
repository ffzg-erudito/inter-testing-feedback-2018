\documentclass[../main.tex]{subfiles}

\begin{document}

Juxtaposed to the well established finding that recalling learned material, compared to merely restudying it, facilitates its long-term retention \citep{roedigeriiiPowerTestingMemory2006, roedigeriiiTestEnhancedLearningTaking2006, rowlandEffectTestingRestudy2014, adesopeRethinkingUseTests2017, roedigeriiiCriticalRoleRetrieval2011, gloverTestingPhenomenonNot1989} stand the results of a number of studies showing that retrieving previously studied information can even facilitate the acquisition of information that is yet to be learned \citep{chanRetrievalPotentiatesNew2018a, yangEnhancingLearningRetrieval2018}. This has stimulated the use of various qualifiers such as \textit{interim} \citep{wissmanInterimTestEffect2011}, \textit{interpolated} \citep{szpunarInterpolatedMemoryTests2013}, and \textit{forward} \citep{pastotterRetrievalPracticeEnhances2014,yangEnhancingLearningRetrieval2018}, in order to distinguish the latter testing effect from the former, now sometimes referred to as the \textit{backward} testing effect (e.g \citealp{yangEnhancingLearningRetrieval2018}). 

A typical demonstration of test-potentiated learning entails an initial learning phase, followed by a period during which participants either restudy the same material, engage in a memory test involving the studied material, or are not exposed to the original material at all. Finally, after a retention interval, an additional administration of a memory test reveals that the group subjected to a memory test during the intervening period has a distinct advantage over the other two groups. In one of the earliest studies showing the effect of test-potentiated new learning, \cite{darleyEffectsPriorFree1971} observed that, when recalling studied lists of words, participants systematically produce more prior-list intrusions when probed for a given list, if their memory of a prior list had not been tested before proceeding to study the given list. 

After a considerable period of silence on the topic, building on these results, \cite{szpunarTestingStudyInsulates2008} told their subjects to study five lists of items in anticipation of a final cumulative test. All subjects were tested immediately after studying the final list, but they engaged in differing intermittent activities between studying the first four lists. One group was tested on each list after studying it, another group restudied each list, and a third group completed a mathematical distractor task. Participants whose memory was tested after each list produced more correct responses and fewer prior-list intrusions on the immediate test administered after studying the last list. They explained the found benefit of testing in terms of a segregation mechanism that prevents overburdening of retrieval cues, which, in the absence of testing, causes a build-up of proactive interference.

Building on these results, \cite{szpunarTestingStudyInsulates2008} told their subjects to study five lists of items in anticipation of a final cumulative test. All subjects were tested immediately after studying the final list, but they engaged in differing intermittent activities between studying the first four lists. One group was tested on each list after studying it, another group restudied each list, and a third group completed a mathematical distractor task. Compared to the groups that were not tested after each list, the one that was tested recalled more items from the last list and showed fewer prior-list intrusions. The following decade has seen a rise in interest in test-potentiated new learning \citep{chanRetrievalPotentiatesNew2018, pastotterRetrievalPracticeEnhances2014, yangEnhancingLearningRetrieval2018}, with studies attempting to delineate the scope of the effect with respect to various moderating variables: the type of study materials, varieties of study designs, and populations, to name but a few. 

Recently, \cite{chanRetrievalPotentiatesNew2018} provided a meta-analytic analysis and comprehensive overview of the literature, identifying four \textit{nonconflicting} theoretical frameworks which were put forth throughout the years as viable explanations. \textit{Resource theories} generally posit that testing increases cognitive resources, but they propose different mechanisms by which this is achieved: (1) proactive interference reduction (e.g. \citealp{wahlheimTestingCanCounteract2015, weinsteinTestingProtectsProactive2011, szpunarTestingStudyInsulates2008, nunesTestingImprovesTrue2012}), (2) restoration of encoding/attentional resources (e.g. \citealp{pastotterRetrievalLearningFacilitates2011}), or (3) alteration of mind wandering patterns (e.g. \citealp{jingInterpolatedTestingInfluences2016,szpunarInterpolatedMemoryTests2013,szpunarMindWanderingEducation2013}). \textit{Metacognitive theories}


{
    \setstretch{1}
    \biblio
}

\subsection{Notes}
one has attempted to commit to memory beforehand

According to current theories of testing, an initial test will directly benefit retention to the extent that it engages those effortful retrieval processes that will likely be called upon at the time of a later test \cite{roedigeriiiPowerTestingMemory2006}

The expectation of a final test ensured the
continued processing of materials across the study sequence.


A number of investigations employed nonepisodic recall as the format of the intermittent testing activity \cite{divisRetrievalSpeedsContext2014, pastotterRetrievalLearningFacilitates2011, }

Za feedback gle u Roediger i Butler (2011)


Možda ZV intruzori nije pokazala razlike između skupina jer smo koristili recognition, a ne free recall.

\end{document}

