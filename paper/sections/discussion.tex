\documentclass[../main.tex]{subfiles}

\begin{document}
	
The aim of this study was to explore the effects of various interpolated activities
and feedback reception on learning complex materials. Participants read three prose
passages, and engaged in interpolated activities between reading episodes. Two 
testing activities were chosen so as to tap into episodic (content-related testing) 
or semantic (general-knowledge testing) memory, while a rereading condition served 
as a control. Participants in the episodic and semantic retrieval conditions were 
also randomly assigned to receive or not to receive feedback. Learning was measured 
through the total number of correct answers and the number of intrusive distractors 
chosen on a test assessing memory of the final part.

We found evidence for an effect of interpolated activity type on TPNL --- participants 
engaging in episodic retrieval exhibited greater TPNL than both participants who engaged in 
semantic retrieval and those in the control condition. Moreover, post hoc analyses
showed that observed differences were driven primarily by the number of correct
responses, while the extent of proactive interference had no bearing on the observed
effect. Notably, \cite{pastotterRetrievalLearningFacilitates2011} did find that the 
effect of various types of retrieval depends on dependent variable under examination. 
In their study, when considering the number of correct recalls, three different types of recall produced a comparable level of TPNL. However, while both episodic and semantic
retrieval effectively eliminated intrusions during final recall, short-term memory
retrieval led to intrusion rates that were equivalent to those observed in the 
no-retrieval conditions, suggesting a possible substantive difference between types 
of retrieval.

On the other hand, we found no evidence for an effect of feedback on TPNL. However, 
the exploratory Bayesian analysis does not exclude the possibility of a feedback effect. 
Still, the estimates we have obtained point to an effect which could be practically 
equivalent to zero, i.e. insignificantly small for real-world purposes.

Taking all this into account, the gathered data provide more evidence in
favour of integration theories than resource theories. Interpreting these results
warrants caution, though, since a more precise estimate of the effect is desirable.
The few studies that have suggested that nonepisodic forms of recall may serve
as effective methods of learning potentiation have drawn on context theories to
explain their results \citep{divisRetrievalSpeedsContext2014, pastotterRetrievalLearningFacilitates2011}.
The lack of evidence for an effect of semantic retrieval on learning
in our study may be taken as an argument against context change accounts of
TPNL. It is important to note that, according to our power analysis, we should
have had enough power to detect small to medium effects (around 85\% power for
\(f^2 \gtrapprox\) 0.06). Furthermore, the Bayesian estimate of the effect of
nonepisodic recall lends additional support for the claim that it does not
enhance learning.

Importantly, our results may confound the lack of evidence in favour of context
or resource accounts with our choice of learning materials. To our knowledge,
only one study (\citealp{divisRetrievalSpeedsContext2014}, Experiment 2) 
investigating the impact of types of retrieval on TPNL used complex learning 
materials, but it did not allow for a direct assessment of the level of 
proactive interference. Previous work has shown that release from proactive
interference may play basically no role when it comes to learning complex
materials. Aiming primariliy to replicate the basic pattern of findings of 
\cite{szpunarTestingStudyInsulates2008}, \cite{wissmanInterimTestEffect2011} used
prose passages as learning materials, and found a very low baseline level of 
proactive intereference in the control group, which effectively excluded
an explanation based on release from proactive interference. They fashioned an 
account very much in line with the general ideas behind integration theories. 
Specifically, they proposed that interpolated testing induces a stronger 
activation and retention of learned information, whose accessibility further 
facilitates comprehension and encoding of new related materials, 
especially connected discourse and lecture videos. They provide an additional
nonconflicting metacognitive explanation based on encoding strategy changes, 
mediated by possible failures of retrieval \citep{bahrickImportanceRetrievalFailures2005}. 
In line with results \cite{wissmanInterimTestEffect2011} obtained using free recall as the 
testing format, the recognition-level method which we brought to bear in order to 
examine possible interference effects showed no signs of proactive interference 
beyond those one could have expected to occur by chance alone.

Another metacognitive explanation, possible reference:
%Soderstrom, N. C. & Bjork, R. A. Testing facilitates the regulation of subsequent
study time. J. Mem. Lang. 73, 99–115 (2014).

iz Chan 2018: testing potentiates new learning across a retention lag
In contrast to unrelated word lists, text passages and videos are
typically written/produced in a coherent manner, which should naturally
invite relational processing, so any relational processing advantage
induced by prior testing is likely to be modest relative to
%baseline (Einstein, McDaniel, Bowers, & Stevens, 1984; Einstein,
%McDaniel, Owen, & Cote, 1990; Masson & McDaniel, 1981). A version
of the strategy change account that is not tied strictly to relational
processing, however, may provide a reasonable explanation for the
TPNL effect with text passages and videos. In a broader sense, the
strategy change account specifies that performing retrieval practice
allows participants to discover the type of learning needed to ensure
satisfactory performance (or conversely, to realize the type of learning
that is inadequate to produce satisfactory performance, if participants
are performing poorly during retrieval practice), and participants can
then adjust their subsequent encoding strategy accordingly. If we take
this broader approach to strategy change, then this account can explain
the TPNL effect with prose/video materials. However, we realize that
the idea that “retrieval practice can improve later encoding strategies”
is perhaps vaguely defined. In fact, such a broad definition of strategy
change may render the account difficult to falsify. With this in mind, we
believe that the strategy change account, as we currently conceive,
should only be applied to explain the TPNL effect with word list type
materials, for which advantageous encoding strategies can be more
precisely defined (but see Jing et al., 2016 in which interspersed testing
improved conceptual integration of materials across sections of a video
lecture). In our opinion, application of this account to prose/video
material should only be done when one clearly outlines what is considered
an advantageous encoding strategy so that the hypothesis can
be adequately tested.

Methodological concerns. In our study participants were thoroughly informed
regarding the activities they would encounter during the procedure, including
the final test that followed the last reading episode. The typical instruction
given to participants in the TPNL paradigm is that interpolated activities 
will be determined randomly by the computer \cite{Yang}. Thus, an attempt is 
made to equalise the expectations of a final test across conditions, and to 
ensure continued processing of materials across the study sequences. Nevertheless,
learners dynamically adjust their expectations based on their experiences of 
the experimental procedure, regardless of being told that the acitivities they
are given are determined randomly \citep{weinsteinRoleTestExpectancy2014}. If
they take a test, they will more likely expect another one, and such expectations
have been shown to make a significant difference for how one approaches
the encoding task (e.g. \citealp{szpunarExpectationFinalCumulative2007}).
Moreover, a basic assumption we have made is that the interpolated activity that
served the function of activating retrieval from semantic memory was effective.
Comparisons of interpolated test achievement between groups engaging in episodic
and semantic retrieval revealed ...



\subsection{Concluding remarks}









\subsection{notes}

Weinstein et al. (2014):
The experiments reported here are based on the assumption that
participants in the tested group may be more likely to expect a test
after the fifth (last) list, having consistently received tests after
previous lists. Those in the untested group, having never received
a test during the experiment until the fifth list test, may therefore
pay less attention or engage in lower quality encoding strategies
during encoding of the fifth list. To test for this alternative explanation, 
we compared the two standard conditions (tested and
untested) with two novel conditions that were identical to the
standard conditions but included a warning before presentation of
the final list to alert participants that they would be tested on the
upcoming list. If attentional processes are mediating the observed
release from proactive interference, warning participants in the
untested group should produce the same benefit as the participants
taking a test after every list


Matej [11:05 PM]
E, sjetio sam se opet da bi možda bilo dobro da negdje navedemo razlike u učinku na testovima prije zadnjega...

Denis [11:09 PM]
Mda, neke te stvari su mi pale na pamet, i činilo mi se kao da bi bilo zgodno spomenuti ih u raspravi. Kao, ako se radi o nekim stvarima koje bi mogle ugroziti efekte ili objasniti neznačajne

	
\end{document}

