\documentclass[../main.tex]{subfiles}

\begin{document}

The aim of this study was to explore the effects of different interpolated 
activities and feedback reception on learning complex materials. Participants 
read three prose passages, and engaged in interpolated activities between 
reading episodes. Two testing activities were chosen so as to tap into episodic 
(content related testing) or semantic (general knowledge testing) memory, while 
a rereading condition served as a control. Participants in the episodic and 
semantic retrieval conditions were also randomly assigned to receive or not to 
receive feedback. Learning was measured through the total number of correct 
answers and the number of intrusive distractors chosen on a test assessing 
memory of the final part.

We found evidence for an effect of interpolated activity type on TPNL --- 
treating the two dependent variables as manifestations of TPNL, we conducted a 
MANOVA, revealing that participants engaging in episodic retrieval exhibited 
greater TPNL than both participants who engaged in semantic retrieval and those 
in the control condition. Moreover, a Roy-Bargmann stepdown analysis was used 
to tease apart the contributions of the two dependent variables to the found 
effect, and it showed that observed differences were driven primarily by the 
number of correct responses, while we found no evidence for the contribution of 
proactive interference. These results are not entirely in line with extant 
research. While our results point to an exclusive role of episodic retrieval
in TPNL, data from \cite{pastotterRetrievalLearningFacilitates2011} 
suggest that retrieval from both long-term and short-term memory
can generate the effect. Notably, 
\cite{pastotterRetrievalLearningFacilitates2011} did find that the effect of 
various types of retrieval depends on the performance measure under 
examination. In their study, when considering the number of correct recalls, 
three different types of recall produced a comparable level of TPNL. However, 
while both episodic and semantic retrieval effectively eliminated intrusions 
during final recall, short-term memory retrieval led to intrusion rates that 
were equivalent to those observed in the no-retrieval conditions, suggesting a 
possible substantive difference between types of retrieval. It is important to 
point out that the effects observed by these authors were obtained by using 
learning materials simpler than the ones we used, which was shown to be an 
important moderating variable \citep{chanRetrievalPotentiatesNew2018}.

The few studies that have suggested that nonepisodic forms of recall may serve 
as effective methods of learning potentiation have drawn on context and resource
theories to explain their results \citep{divisRetrievalSpeedsContext2014, 
pastotterRetrievalLearningFacilitates2011}. 
\cite{divisRetrievalSpeedsContext2014} proposed that retrieval processes 
enhance context fluctuation, associating specific context cues with specific 
study episodes, thereby increasing the contextual disparity between information 
acquired across study sessions. This, in turn, isolates individual learning 
episodes, and reduces the memory search set and proactive interference.
The lack of evidence for an effect of semantic retrieval 
on learning in our study may be taken as an argument against this encoding 
``resetting'' and context change accounts of TPNL because, presumably, semantic 
retrieval should have produced the internal context change required for 
resetting the encoding process 
\citep{pastotterRetrievalLearningFacilitates2011}. Finally, we will mention that
the Bayesian estimate of the effect of nonepisodic recall lends support for the
claim that it does not enhance learning.

While we found no evidence for an effect of feedback on TPNL, the exploratory 
Bayesian analysis does not exclude the possibility of a feedback effect. Still, 
the estimates we have obtained point to an effect which could be practically 
equivalent to zero, i.e. insignificantly small for real-world purposes.
From this we gather that the collected data provide no evidence that a 
proactive interference reduction mechanism underpins TPNL. Interpreting these 
results warrants caution, though, since a more precise estimate of the effect 
is desirable.

Importantly, our choice of learning materials could have prevented us from 
finding evidence in favour of context theories and an account based on the 
reduction of proactive interference. To our knowledge, only one study 
(\citealp{divisRetrievalSpeedsContext2014}, Experiment 2) investigating the 
impact of different types of retrieval on TPNL used complex learning 
materials, but it did not allow for a direct assessment of the level of 
proactive interference. Previous work has shown that release from proactive
interference may play basically no role when it comes to learning complex
materials. Aiming primarily to replicate the basic pattern of findings of 
\cite{szpunarTestingStudyInsulates2008}, \cite{wissmanInterimTestEffect2011} 
used prose passages as learning materials, and found a very low baseline level 
of proactive interference in the control group, which effectively excluded
an explanation based on release from proactive interference. In line with 
results obtained by \cite{wissmanInterimTestEffect2011}, who used free recall 
as the testing format, the recognition-level method which we employed in order 
to examine possible interference effects showed no signs of proactive 
interference beyond those one could have expected to occur by chance alone.

Although our results do permit setting aside some accounts within the 
resource theoretical and context change frameworks, other theories still 
provide viable explanations of TPNL. \cite{wissmanInterimTestEffect2011} 
fashioned an account very much in line with the general ideas behind 
integration theories. Specifically, they proposed that interpolated testing 
induces a stronger activation and retention of learned information, whose 
accessibility further facilitates comprehension and encoding of new related 
materials, especially connected discourse and lecture videos. More recent 
studies provided supportive evidence for explanations relying on changes in 
patterns of mind wandering \citep{szpunarInterpolatedMemoryTests2013}, whereby 
testing increases mind wandering related to the acquired information
\citep{jingInterpolatedTestingInfluences2016}. 
\cite{wissmanInterimTestEffect2011} suggested an additional nonconflicting 
metacognitive explanation based on encoding strategy changes, mediated by 
possible failures of retrieval \citep{bahrickImportanceRetrievalFailures2005}, 
whereby subjects use immediate feedback on recall to adjust their encoding 
strategy. In line with these proposals, recent studies have shown that 
performing retrieval modifies the learner's approach to new
information \citep{choTestingEnhancesBoth2017, 
soderstromTestingFacilitatesRegulation2014}, which may lead to superior 
semantic organisation of acquired knowledge 
\citep{chanTestingPotentiatesNew2018, jingInterpolatedTestingInfluences2016}. 
However, our opinion is aligned with that of 
\cite{chanTestingPotentiatesNew2018}, who caution that the application 
of a strategy change account to TPNL of complex learning materials should be 
preceded by an adequate description and operationalisation of an
``advantageous'' encoding strategy, which would enable precise tests of the
proposed mechanism.

Finally, we have to address certain methodological concerns. In our study 
participants were thoroughly informed regarding the activities they would
encounter during the procedure, including the final test that followed the last 
reading episode. The typical instruction given to participants in the TPNL 
paradigm is that interpolated activities will be determined randomly by the 
computer \cite{yangEnhancingLearningRetrieval2018}. Thus, an attempt is made to 
equalise the expectations of a final test across conditions, and to ensure 
continued processing of materials across the study sequences. Nevertheless,
learners dynamically adjust their expectations based on their experiences of 
the experimental procedure, regardless of being told that the activities they
are given are determined randomly \citep{weinsteinRoleTestExpectancy2014}. If
they take a test, they will more likely expect another one, and such 
expectations have been shown to make a significant difference for how one 
approaches the encoding task (e.g. 
\citealp{szpunarExpectationFinalCumulative2007}). Moreover, a basic assumption 
we have made is that the interpolated activity that served the function of 
activating retrieval from semantic memory was effective.

Further, we cannot exclude the possibility that the nature of our interpolated
activities had differential effects on our participants' motivation.
Several participants did remark that the text was tedious, and it is possible 
that the motivation of participants in the episodic retrieval condition 
persisted throughout the procedure, while that of the other participants waned 
as the procedure progressed. However, if this were true, we can easily imagine 
that participants in the rereading condition should have obtained the lowest 
scores, given that they were instructed to read the texts twice. However, this 
is not the case. Furthemore, we believe that most people would agree that 
answering general knowledge questions is more interesting than answering 
questions about weeds, and that this activity would, therefore, help 
sustain the motivation level. On the other hand, differences in 
motivation could have been caused by unequal task difficulties --- the mean 
proportions of correctly answered questions are larger in the content-related 
than in the general-knowledge testing conditions. Importantly,
the mean proportion of correct answers on the first interpolated 
content-related test is higher than on the second. If the tests were equally 
difficult, and if there were a TPNL effect, we would expect higher scores on 
the second test. This points to the tentative conclusion that the interpolated 
tests themselves differ in difficulty. Thus, we cannot reject the possibility 
that differing difficulties have had an effect on our participants' 
achievement. However, \citet{divisRetrievalSpeedsContext2014} argue that the 
difficulty of the interpolated tasks is irrelevant. Still, such claims are yet 
to be corroborated by experimental data.

To conclude, our findings confirm the effect of episodic recall on TPNL, but we
fail to find evidence for an effect of semantic recall. Further, evidence for an
effect of feedback is also lacking. Our data are generally aligned with
predictions stemming from metacognitive and integration theories of TPNL, and
speak against proactive interference reduction accounts within the wider
framework of resource theories.

\end{document}
