\documentclass[../main.tex]{subfiles}

\begin{document}

The aim of this study was to explore the effects of different interpolated activities
and feedback reception on learning complex material. Participants read a text
about weeds divided into three parts, and engaged in interpolated activities between
the parts. The activities were chosen so as to tap into episodic (content-related testing)
or semantic (general-knowledge testing) memory. A rereading condition served as a control.
Participants in the content-related and general-knowledge conditions were also randomly
assigned to receive or not to receive feedback. Learning was measured through the total
number of correct answers on a final test, and through the number of intrusive
distractors chosen.

We found evidence for an effect of interpolated activity type on the total number
of correct answers --- participants engaging in episodic retrieval scored higher
than both the participants engaging in semantic retrieval, and the participants
in the control condition.

\end{document}
