\documentclass[../main.tex]{subfiles}

\begin{document}

The aim of this study was to explore the effects of different interpolated  
activities and feedback reception on learning complex materials. We found 
evidence for an effect of interpolated activity type on TPNL — treating the 
two dependent variables as manifestations of TPNL, we conducted a MANOVA, 
revealing that participants engaging in episodic retrieval exhibited greater 
TPNL than both participants who engaged in semantic retrieval and those in 
the control condition. Moreover, a Roy-Bargmann stepdown analysis showed 
that observed differences were driven primarily by the number of correct 
responses, while finding no evidence for the contribution of PI.

The fact that we observed the effect of interest while employing a testing 
format which is known to produce the smallest effects is interesting in and 
of itself, and suggests that the effect should hold in conditions that are 
arguably the most prevalent in western educational systems. Nevertheless, 
our results are not entirely in line with extant research. For example, 
while our results point to an exclusive role of episodic retrieval, 
\cite{pastotterRetrievalLearningFacilitates2011} suggest 
that both types of retrieval can generate TPNL. Notably, these authors used 
simpler learning materials and free recall — both learning material 
complexity and testing format are known moderators of TPNL 
\citep{chanRetrievalPotentiatesNew2018}. However, the few studies that 
examined the effects of nonepisodic recall on TPNL have produced equivocal 
results. While two studies suggested that nonepisodic and episodic recall 
have comparable effects \citep{divisRetrievalSpeedsContext2014, 
pastotterRetrievalLearningFacilitates2011}, 
\cite{weinsteinNotAllRetrieval2015} failed to show this. Among a number of 
methodological differences between these studies, the specific type of 
nonepisodic recall stands out as a possible reason behind the diverging 
results. While \cite{pastotterRetrievalLearningFacilitates2011} and 
\cite{divisRetrievalSpeedsContext2014} both used semantic generation, 
\cite{weinsteinNotAllRetrieval2015} used recall from autobiographical 
memory. Delineating the potential distinctive effects various forms of 
non-episodic recall could have in the TPNL paradigm is a goal future studies 
may pursue.

Studies that have suggested that nonepisodic recall may serve as an effective
method of learning potentiation have drawn on context theories to explain 
their results \citep{divisRetrievalSpeedsContext2014, 
pastotterRetrievalLearningFacilitates2011}. 
\cite{divisRetrievalSpeedsContext2014} proposed that retrieval processes 
enhance context fluctuation, thereby increasing the contextual disparity 
between information acquired across study sessions. This, in turn, reduces 
the memory search set and PI. The absence of an effect of semantic retrieval 
on learning in our study may be taken as evidence against this context 
change account of TPNL because, presumably, semantic retrieval should have 
produced the internal context change required for resetting the encoding 
process \citep{pastotterRetrievalLearningFacilitates2011}. A Bayesian 
estimate of the effect of nonepisodic recall also lends support for the 
claim that it does not enhance learning (see supplementary materials). 
Still, a basic assumption we have made is that the interpolated activity 
that served the function of activating retrieval from semantic memory in our 
study was effective. 

While we found no evidence for an effect of feedback on TPNL, exploratory 
Bayesian analyses do not exclude the possibility of a feedback effect, but 
the obtained estimates point to an effect which could be practically 
equivalent to zero. From this we gather that the collected data provide no 
evidence that a PI reduction mechanism underpins TPNL. Interpreting these 
results warrants caution, though, since a more precise estimate of the 
effect is desirable. Assuming that the effect of feedback was in fact 
insignificant, this may have occurred because episodic recall in and of 
itself gave subject the information they required in order to adjust their 
encoding strategy \citep{kornellUnsuccessfulRetrievalAttempts2009}. If this 
were the case, additional explicit feedback might have been wholly redundant.

Importantly, our choice of learning materials could have prevented us from 
finding evidence in favour of context theories and an account based on the 
reduction of PI. However, previous work has shown that release from PI may 
play basically no role when it comes to learning complex materials 
\citep{divisRetrievalSpeedsContext2014, wissmanInterimTestEffect2011}. The 
recognition-level method which we employed showed no signs of PI beyond 
those expected to occur by chance alone. Admittedly, limiting the number of 
choices by displaying possible answers may have diluted the interference 
effect other unwritten pieces of information might have had, if we had used 
free recall instead.

Recently, \cite{chanRetrievalPotentiatesNew2018} provided a meta-analytic 
analysis and comprehensive overview of the literature, identifying four 
theoretical frameworks that figure prominently as explanations for TPNL. 
Besides context theories, whose main points are outlined above, 
\textit{resource theories} generally posit that testing increases cognitive 
resources either by (i) reducing encoding-based PI (e.g. 
\citealp{wahlheimTestingCanCounteract2015, 
	weinsteinTestingProtectsProactive2011, szpunarTestingStudyInsulates2008, 
	nunesTestingImprovesTrue2012}), (ii) restoring encoding/attentional 
resources (e.g. \citealp{pastotterRetrievalLearningFacilitates2011}), or by 
(iii) altering mind wandering patterns to focus them more on the learning 
material (e.g. \citealp{jingInterpolatedTestingInfluences2016, 
	szpunarInterpolatedMemoryTests2013,szpunarMindWanderingEducation2013}).
A circumscription of separate learning episodes is at the core of both 
resource and context accounts, but its effect on learning is different 
depending on the outlook. According to the former, isolating a learning 
episode through attempts at recall increases resources for subsequent 
learning by preventing \textit{encoding-based} PI. On the other hand, 
context theories emphasise later \textit{retrieval} processes, whereby 
isolating an earlier learning episode reduces the memory search set for 
retrieval. 

While resource theories revolve around the amount of deployable cognitive 
resources, \textit{metacognitive theories} emphasise the optimisation of 
their deployment, regardless of their amount. Thus, using metacognitive 
knowledge gained through retrieval attempts (e.g. 
\citealp{choTestingEnhancesBoth2017, 
	chanTestingPotentiatesNew2018}), and perhaps even failures of retrieval 
\citep{bahrickImportanceRetrievalFailures2005}, learners can adjust their 
strategies. In line with these proposals, recent studies have shown that 
retrieval modifies the learner's approach to new information 
\citep{choTestingEnhancesBoth2017, 
	soderstromTestingFacilitatesRegulation2014}, which may lead to superior 
semantic organisation of acquired knowledge 
\citep{chanTestingPotentiatesNew2018, 
jingInterpolatedTestingInfluences2016}. For example, in a recent 
investigation using multi-list learning, 
\cite{chanTestingPotentiatesNew2018} found that, compared to untested 
groups, the group subjected to interpolated testing displayed an increase in 
category-based clustering during free recall. They ascribe this to an 
adoption of more efficient encoding/retrieval strategies. A similar pattern 
is often observed in studies exploring the testing effect, where a greater 
number of tests is associated with improved organisation of output displayed 
upon testing \citep{zarombTestingEffectFree2010, 
karpickeRetrievalBasedLearningActive2012}.

Finally, \textit{integration theories} advance the notion that interpolated 
testing facilitates the integration of the new-learning material either with 
its retrieval cues or with the prior-learning material. On one account, 
testing increases the likelihood of spontaneous covert retrieval of 
older items during the study of new items, fostering their 
integration, thereby increasing conceptual organisation (e.g. 
\citealp{jingInterpolatedTestingInfluences2016}) and the effectiveness of 
retrieval cues \citep{pycWhyTestingImproves2010}. For example, 
\cite{jingInterpolatedTestingInfluences2016} found that interpolated testing 
increased the clustering of related information that is acquired 
\textit{across} different segments within a video-recorded lecture.

The accounts mentioned above are not mutually exclusive. For example, one 
can imagine how a strategy shift could reduce PI and foster integration. 
Still, we believe our results are mostly aligned with a metacognitive 
explanation of the effect \citep{wissmanInterimTestEffect2011, 
chanTestingPotentiatesNew2018}. 

Finally, we have to address certain methodological concerns.  In our study, 
participants were thoroughly informed regarding the activities they would 
encounter during the procedure, including the final test following the last 
reading episode. The typical instruction given to participants in the TPNL 
paradigm is that interpolated activities will be determined randomly
\citep{yangEnhancingLearningRetrieval2018}. Thus, an attempt is made to 
equalise expectations of a final test across conditions, and to ensure 
continued processing of materials across the study sequences. Nevertheless, 
learners dynamically adjust their expectations based on their experiences of 
the procedure, regardless of the instructions they are given 
\citep{weinsteinRoleTestExpectancy2014}.  If they take a test, they will 
more likely expect another one, and such expectations are known to influence 
encoding (e.g. \citealp{szpunarExpectationFinalCumulative2007}). 

Further, we cannot exclude the possibility that our interpolated activities 
had differential effects on our participants’ motivation. Several  
participants remarked that the text was tedious, and it is possible that the 
motivation of participants in the episodic retrieval condition persisted 
throughout the procedure, while that of the other participants waned as the 
procedure progressed. However, if this were true, we should have observed 
the lowest scores in the rereading group. Yet, this is not the case. 
Furthermore, we find no reason to believe that answering general knowledge 
questions is less interesting than answering questions about weeds, and 
presume that the former would, therefore, help sustain motivation. On the 
other hand, differences in engagement could have been caused by unequal task 
difficulties — the mean proportions of correctly answered questions are 
larger in the content-related than in the general-knowledge testing 
conditions. Importantly, the mean proportion of correct answers on the first 
interpolated content-related test is higher than on the second. If the tests 
were equally difficult, and if there had been a TPNL effect, we would expect 
higher scores on the second test. This points to the tentative conclusion 
that the interpolated tests themselves differed in difficulty. Thus, we 
cannot reject the possibility that differing difficulties affected our 
participants’ achievement. However, \citet{divisRetrievalSpeedsContext2014} 
argue that the difficulty of the interpolated tasks is irrelevant. Still, 
such claims are yet to be corroborated by experimental data.

To conclude, our findings confirm the effect of episodic recall on TPNL, but 
we fail to find evidence for an effect of semantic recall. Further, evidence 
for an effect of feedback is also lacking. Our data are generally aligned 
with predictions stemming from metacognitive and integration theories of 
TPNL, and speak against PI reduction accounts within the wider framework of 
resource theories.

\end{document}





%Szpunar (2008)
%In each of our experiments, participants expected a final test.
%Accordingly, they were likely to hold words in mind (across lists) as they 
%studied (Masson & McDaniel, 1981; Szpunar et al., 2007).
%Participants also knew that a test might follow the presentation of
%any list in the study sequence. Thus, although maintenance of all
%lists was important in the long term, subsequent lists in the study
%sequence had to be discriminated in anticipation of initial testing.
