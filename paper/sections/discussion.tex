\documentclass[../main.tex]{subfiles}

\begin{document}

The aim of this study was to explore the effects of various interpolated activities
and feedback reception on learning complex materials. Participants read a text
about weeds divided into three parts, and engaged in interpolated activities between
reading episodes. Two testing activities were chosen so as to tap into episodic 
(content-related testing) or semantic (general-knowledge testing) memory, while a 
rereading condition served as a control. Participants in the episodic and semantic
retrieval conditions were also randomly assigned to receive or not to receive 
feedback. Learning was measured through the total number of correct answers 
and the number of intrusive distractors chosen on a test assessing memory of the
final part.

We found evidence for an effect of interpolated activity type on TPNL --- participants 
engaging in episodic retrieval scored higher than both participants who engaged in 
semantic retrieval and those in the control condition. On the other hand, we
have found no evidence for an effect of feedback on TPNL. However, the
exploratory Bayesian analysis does not exclude the possibility of a feedback
effect. Still, the estimates we have obtained point to an effect which could be
practically equivalent to zero, i.e. insignificantly small for real-world
purposes. Taking all this into account, the gathered data provide more evidence in
favour of integration theories than resource theories. 

% ovo je trenutno nepovezano s prethodnim. tamo pričamo o feedbacku i broju
% točnih, a tu najednom skačemo na intruzore
Moreover, analyses showed
that the observed differences were driven entirely by differences in the number of
chosen correct answers.

% ovo bih maknio jer se poziva na neku analizu koju u biti nismo radili, a to je
% anova s aktivnosti i intruzorima
% After controlling for this statistically, the type of 
% interpolated activity had no bearing on the extent of proactive interference.

The few studies that have suggested that nonepisodic forms of recall may serve
as effective methods of learning potentiation have drawn on context theories to
explain their results \citep{divisRetrievalSpeedsContext2014, pastotterRetrievalLearningFacilitates2011}.
The lack of evidence for an effect of semantic retrieval on learning
in our study may be taken as an argument against context change accounts of
TPNL. It is important to note that, according to our power analysis, we should
have had enough power to detect small to medium effects (around 85\% power for
\(f^2 \gtrapprox\) 0.06). Furthermore, the Bayesian estimate of the effect of
nonepisodic recall lends additional support for the claim that it does not
enhance learning.
, only \cite{divisRetrievalSpeedsContext2014} used complex learning materials,
but they 

Divis i Benjamin (2014)
However, their protocol didn't allow for assessing the contribution of proactive interference to TPNL. Furthermore, the authors argue for the irrelevance of the level of difficulty of alternative activities (i.e. perhaps the semantic retrieval and the distractor tasks were not equal with regards to difficulty) whilst referring to the variety of tasks used by \cite{pastotterRetrievalLearningFacilitates2011}, who found effects of similar magnitude for the retrieval tasks and a lower but approximately equal magnitude for the distractor task and restudy. But it does not follow that these patterns in the data refute an explanation based on task difficulty, i.e. the distractor task and restudy may have been easier than the retrieval tasks. 



Yang et al. (2018): za neznačajan efekt na intruzore!
the activation facilitation and enhanced encoding effort mechanisms may play
important roles for complex materials whereas the release from PI
mechanism is likely to play little role.29


Notably, \cite{pastotterRetrievalLearningFacilitates2011} did find that the effect
of various types of retrieval depends on the examined dependent variable. In their
study, both episodic and semantic memory retrieval effectively eliminated intrusions
during final recall. Short-term memory retrieval, however, led to intrusion rates 
that were equivalent to those observed in the no-retrieval conditions, suggesting
a possible substantive difference between types of retrieval.


\subsection{Notes for discussion}


Chan et al. (2018). across a retention interval
In contrast to unrelated word lists, text passages and videos are
typically written/produced in a coherent manner, which should naturally
invite relational processing, so any relational processing advantage
induced by prior testing is likely to be modest relative to
%baseline (Einstein, McDaniel, Bowers, & Stevens, 1984; Einstein,
%McDaniel, Owen, & Cote, 1990; Masson & McDaniel, 1981). A version
of the strategy change account that is not tied strictly to relational
processing, however, may provide a reasonable explanation for the
TPNL effect with text passages and videos. In a broader sense, the
strategy change account specifies that performing retrieval practice
allows participants to discover the type of learning needed to ensure
satisfactory performance (or conversely, to realize the type of learning
that is inadequate to produce satisfactory performance, if participants
are performing poorly during retrieval practice), and participants can
then adjust their subsequent encoding strategy accordingly. If we take
this broader approach to strategy change, then this account can explain
the TPNL effect with prose/video materials. However, we realize that
the idea that “retrieval practice can improve later encoding strategies”
is perhaps vaguely defined. In fact, such a broad definition of strategy
change may render the account difficult to falsify. With this in mind, we
believe that the strategy change account, as we currently conceive,
should only be applied to explain the TPNL effect with word list type
materials, for which advantageous encoding strategies can be more
precisely defined (but see Jing et al., 2016 in which interspersed testing
improved conceptual integration of materials across sections of a video
lecture). In our opinion, application of this account to prose/video
material should only be done when one clearly outlines what is considered
an advantageous encoding strategy so that the hypothesis can
be adequately tested.


Možda ZV intruzori nije pokazala razlike između skupina jer smo koristili 
recognition, a ne free recall.
Context change account?






Methodological concerns. The expectation of a final test ensured the
continued processing of materials across the study sequence.
Chan et al. (2018): % za istaknuti mogućnost da su se očekivanja ispitanika dinamički mijenjala tijekom mjerenja
For example, in a multilist learning environment, having taken a
recent memory test increases learners’ expectation that they will
again be tested in the immediate future, even when they are told
that whether a test will follow each study list is determined
randomly (Weinstein et al., 2014). Such test expectancies have
been shown to significantly influence how participants approach
%the encoding task (Balota & Neely, 1980; May & Thompson,
%1989; Szpunar, McDermott, & Roediger, 2007).
Weinstein et al. (2014):
The experiments reported here are based on the assumption that
participants in the tested group may be more likely to expect a test
after the fifth (last) list, having consistently received tests after
previous lists. Those in the untested group, having never received
a test during the experiment until the fifth list test, may therefore
pay less attention or engage in lower quality encoding strategies
during encoding of the fifth list. To test for this alternative explanation, 
we compared the two standard conditions (tested and
untested) with two novel conditions that were identical to the
standard conditions but included a warning before presentation of
the final list to alert participants that they would be tested on the
upcoming list. If attentional processes are mediating the observed
release from proactive interference, warning participants in the
untested group should produce the same benefit as the participants
taking a test after every list


Matej [11:05 PM]
E, sjetio sam se opet da bi možda bilo dobro da negdje navedemo razlike u učinku na testovima prije zadnjega...

Denis [11:09 PM]
Mda, neke te stvari su mi pale na pamet, i činilo mi se kao da bi bilo zgodno spomenuti ih u raspravi. Kao, ako se radi o nekim stvarima koje bi mogle ugroziti efekte ili objasniti neznačajne











\end{document}
